\documentclass[main.tex]{subfiles}
\begin{document}

One important conclusion to draw from the models is that spherically symmetric accretion is generally a very \emph{dim} process: in non-radial motion around a Schwarzschild black hole the efficiency can be as high as \(e \sim \num{6e-2}\) \cite[eq. 2.8.5]{Nobili:2000}, which is almost three orders of magnitude more than \(e\sim \num{9e-5}\), the maximum efficiency obtained in the models of spherical accretion considered here.

Further, the bimodal behaviour of the efficiency does not have an apparent theoretical justification.

The research explored in this thesis is pioneering in the numerical study of accretion using a fully relativistic formulation, but there are many aspects of this problem to be explored beyond what was treated here.
The problem may be treated introducing viscosity (which is done in \cite[]{TurollaNobili:1989}), time dependence, frequency dependence, considering more of the moment equations, relaxing the assumption of spherical symmetry.

\end{document}
