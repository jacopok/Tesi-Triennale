\documentclass[main.tex]{subfiles}
\begin{document}


I will use Greek indices ($\mu$, $\nu$, $\rho$\dots) to denote 4-dimensional indices ranging from 0 to 3, and Latin indices ($i$, $j$, $k$\dots) to denote 3-dimensional indices ranging from 1 to 3.

I will use the ``mostly plus'' metric for flat Minkowski space-time, $\eta_{\mu \nu} = \diag(-, +, +, +)$: therefore four-velocities will have square norm \(u^\mu u_\mu = -1\).
I will use Einstein summation convention: if an index appears multiple times in the same monomial, it is meant to be summed over.

I will always use the abuse of notation in which a vector is denoted by its components, and a free index meaning all of the possible values it can take, such as \(x^\mu\) denoting a point in spacetime.

Take a diffeomorphism $x \rightarrow y$, with Jacobian matrix $\pdv*{y^\mu}{x^\nu}$.
The indices of contravariant vectors, transforming as

\begin{equation}
    V^\mu \rightarrow \qty( \pdv{y^\mu}{x^\nu})  V^\nu
\end{equation}

will be denoted as upper indices, while the indices of covariant vectors, transforming as

\begin{equation}
V_\mu \rightarrow \qty( \pdv{x^\nu}{y^\mu})  V_\nu
\end{equation}

will be denoted as lower indices; the same applies to higher rank tensors.

Unless otherwise specified, I will work in geometrized units, where $c = G = 1$.

In section \ref{sec:radiative-effects} I will use the notation from \textcite[]{Thorne:1981feb}: \(A_k\) will represent a sequence of \(k\) indices labelled as \(\alpha_i\), for \(i\) between 1 and \(k\). The same will hold for \(B_k \rightarrow \qty{\beta_i}\) etc.

Take a tensor with many indices, \(T_{A_k B_j}\). These indices can be symmetrized and antisymmetrized, and I will use the following conventions:

\begin{align}
    T_{(A_k) B_j } &= \frac{1}{k!} \sum_{\sigma \in S_k} T_{\sigma(A_k)J} \\
    T_{[A_k]B_j} &= \frac{1}{k!} \sum_{\sigma \in S_k} \sign{\sigma} T_{\sigma(A_k)B_j}
\end{align}

where \(S_k \ni \sigma \) is the group of permutations of $k$ elements, and $\sign{\sigma}$ is 1 if $\sigma$ is an even permutation (it can be obtained in an even number of pair swaps) and -1 otherwise.

In the case a set of indices that are not nearby need to be (anti)symmetrized, I will use vertical bars: for example, \(R_{(\mu | \nu| \rho \sigma)}\) means that we take the permutations of the indices \(\mu\), \(\rho\) and \(\sigma\).

\end{document}
