\documentclass[main.tex]{subfiles}
\begin{document}

\subsection{Basics}

\paragraph{Metric}

The metric tensor \(g_{\mu\nu}\) is a symmetric \((0,2)\) tensor which gives us a scalar product at every point in our manifold: \(x \cdot y = g_{\mu\nu} x^\mu y^\nu\).
It is not intrinsic to the manifold.
By integrating the velocity vector we can find the lengths of curves \(x^\mu(\lambda)\):

\begin{equation}
    L = \int \sqrt{g_{\mu\nu} \dv{x^\mu}{\lambda} \dv{x^\nu}{\lambda} }  \dd{\lambda}
\end{equation}

For a flat spacetime we use the Minkowski metric \(\eta_{\mu\nu}\). In general, in the presence of matter the manifold will be curved, so there will not be a coordinate transformation to cast \(g_{\mu\nu}\) in the form \(\eta_{\mu\nu}\). If we choose a certain point \(P\), however, it is possible to find a transformation in order to impose the conditions \(g_{\mu\nu}(P)=\eta_{\mu\nu}(P)\), \(\partial_\rho g_{\mu\nu}(P) = 0\) \cite[pages 49--50]{Carroll:1997ar}.

\paragraph{Tensor calculus}

The covariant derivative keeps account of the shifting of the basis vectors:

\begin{equation}
    \nabla_\mu A^\nu = \partial_\mu A^\nu + \Gamma^\nu _{\alpha \mu}  A^\alpha
\end{equation}

The rank-3 objects $\Gamma$ are called Christoffel symbols. They are not tensors! they depend on the choice of basis $e_\alpha$, and they satisfy $\nabla _\mu e_\alpha = \Gamma ^\nu _{\mu \alpha} e_\nu$. They are not intrinsic to the manifold.

If we have the metric (and make reasonable assumptions of the connection being torsion-free), they can be calculated as:

\begin{equation}
    \Gamma^\mu_{\nu \rho} = \frac{1}{2} g^{\mu \alpha} \qty(
    \partial_\rho g_{\alpha \nu} +
    \partial_\nu g_{\alpha \rho} -
    \partial_\alpha g_{\nu \rho}
    ) \label{eq:christoffel-symbols-from-metric}
\end{equation}

This also tells us that they are symmetric in the lower two indices: $\Gamma ^\mu _{\nu \rho} = \Gamma ^\mu _{\rho \nu}$.

The divergence of a vector field $A^\mu$ can be calculated as:

\begin{equation}
    \nabla_\mu A^\mu = \frac{1}{\sqrt{-g}}\partial_\mu \qty(\sqrt{-g}A^\mu) \label{eq:covariant-divergence}
\end{equation}

where $g$ is the determinant of the metric.

We can also define the Dalambertian operator $\square = \nabla_\mu \nabla^\mu = \nabla_\mu \partial^\mu$, which can only act on scalars, and it does so like:

\begin{equation}
    \square A = \nabla_\mu (\partial^\mu A) =  \frac{1}{\sqrt{-g}}\partial_\mu \qty(\sqrt{-g}\partial^\mu A)
\end{equation}

If we differentiate and antisymmetrize (so, take the rotor of) an antisymmetric tensor $F_{[\mu \nu]}$, the Christoffel symbols cancel:

\begin{equation}
    \nabla_{[\mu} F_{\nu\rho]} = \partial_{[\mu} F_{\nu\rho]}
\end{equation}

The derivative with respect to proper time is $\dv{}{\tau} = u^\mu \nabla_\mu$.

\index{covariant acceleration}Covariant acceleration is defined as:

\begin{equation} \label{eq:covariant-acceleration-def}
    a^\nu = u^\mu \nabla_\mu u^\nu
\end{equation}

\paragraph{Stokes' theorem}

Following \cite[]{Unger:2016}.
If we have a manifold of dimension \(n\), and an \(n\)-dimensional region \(V\) of this manifold equipped with coordinates \(x^\mu\) a metric \(g_{\mu\nu}\), with a submanifold boundary \(\partial V\) equipped with coordinates \(y^\mu\) and the induced metric \(h_{\alpha\beta} = \qty(\pdv*{x^\mu}{y^\alpha}) \qty(\pdv*{x^\nu}{y^\beta}) g_{\mu\nu}\), and for which we have a properly oriented normal vector \(n^\mu (y)\); then for any vector \(f^\mu\) we have:

\begin{equation} \label{eq:stokes-theorem}
    \int _{V} \nabla_\mu f^\mu \sqrt{\abs{\det g}}  \dd[n]{x}  = \int _{\partial V} f^\mu  n_\mu \sqrt{\abs{\det h}} \dd[n-1]{y}
\end{equation}

\paragraph{Curvature}

The curvature of spacetime is fully described by the Riemann curvature tensor, which is a fourth rank tensor: for any generic vector \(V^\mu\),

\begin{equation} \label{eq:riemann-tensor-def}
    R ^{\mu} _{\nu \rho \sigma} V^\nu \defeq [\nabla_\rho, \nabla_\sigma]   V^\mu
\end{equation}

It can be calculated using the Christoffel symbols, and while they are not tensors \(R ^{\mu} _{\nu \rho \sigma}\) is one. This result follows by direct computation from formula \eqref{eq:riemann-tensor-def}.

\begin{equation}
    R ^{\mu} _{\nu \rho \sigma} =
     \partial_\rho \Gamma^\mu_{\nu \sigma}
    -\partial_\sigma \Gamma^\mu_{\nu \rho}
    +\Gamma^\mu_{\rho \lambda} \Gamma ^{\lambda} _{\sigma \nu}
    -\Gamma^\mu_{\sigma \lambda} \Gamma ^{\lambda} _{\rho \nu}
\end{equation}

The Christoffel symbols can be nonzero if we choose certain coordinates even for flat spacetime, but the Riemann tensor is zero iff the spacetime is flat.

The Riemann tensor satisfies the following identities \cite[eqs. 8.45 and 8.76]{MisnerThorneWheeler:1973}:

\begin{subequations}
\begin{align}
  \nabla _{[\lambda} R_{\mu\nu]\rho \sigma} &= 0 \label{eq:bianchi-identities}  \\
  R_{\mu\nu\rho\sigma} &= R_{[\mu\nu][\rho\sigma]} = R_{[\rho\sigma][\mu\nu]}  \\
  R_{[\mu\nu\rho\sigma]} &= 0 = R_{\mu[\nu\rho\sigma]}
\end{align}
\end{subequations}

If we define the Ricci tensor \(R_{\mu\nu} = R^\rho_{\mu \rho \nu}\) and the curvature scalar \(R = R_{\mu\nu}g^{\mu\nu}\), we can rewrite \eqref{eq:bianchi-identities}  as \(\nabla_\mu R = 2 \nabla_\nu R^{\nu}_{\mu}\).

\paragraph{Geodesics}

If we have a path $x^\mu(\lambda)$, we would like to see if it is a geodesic, that is, if it is stationary with respect to path length. To do this we can stationarize the action corresponding to the lagrangian $\Lagr (x, \Dot{x}) = g_{\mu\nu} \Dot{x}^\mu \Dot{x}^\nu$ (where we use $\Dot{x} = \dv*{x}{\lambda}$). The Lagrange equations then are:

\begin{equation}
    \Ddot{x}^\mu + \Gamma^\mu_{\nu\rho} \Dot{x}^\nu \Dot{x}^\rho = 0
\end{equation}

Where $\Gamma$ are the Christoffel symbols, which can be calculated by differentiating the metric, as shown in \eqref{eq:christoffel-symbols-from-metric}. $\Lagr$ is an integral of these Lagrange equations.

If the parameter $\lambda$ is taken to be the proper time $s$, then the equation is

\begin{equation}
    \dv{u^\mu}{s} + \Gamma^\mu _{\nu\rho} u^\nu u^\rho = 0
\end{equation}

Notice that this is equivalent to the covariant acceleration \eqref{eq:covariant-acceleration-def} being zero.

\paragraph{Fermi-Walker transport}

Take a general vector field \(V ^{\mu} (s)\) defined along a curve, with its tangent vector \(u^\mu\) whose covariant acceleration is \(a^\mu\).
Then we say that \(V^\mu\) is transported according to Fermi-Walker iff it satisfies

\begin{equation}
    \dot{V}^\mu  = u^\nu \nabla_\nu V^\mu
    = 2 V_\rho u^{[\mu} a^{\rho]}
\end{equation}

This condition is always satisfied by \(V^\mu = u^\mu\), since \(a^\mu u_\mu = 0\), whether or not the curve is a geodesic. The tangent vector is \emph{parallel} transported only for geodesics.

The justification of this definition is the fact that we want the transformations of our our tetrad to be infinitesimal Lorentz boosts, which are generated by antisymmetric tensors, and we want to prohibit any rotations in the plane orthogonal to \(a^\mu\) and \(u^\mu\)

\paragraph{Tetrads and projectors} \label{par:tetrads}

We want to work in a reference in which the velocity $u^\mu$ is purely timelike. This can always be found by the equivalence principle. Such a reference can be completed into what  is called a tetrad, for which the metric becomes the Minkowski metric in a neighbourhood of the point we consider.

We call the velocity \(u^\mu = V^\mu _{(0)}\), and add to it three other vectors \(V^\mu_{(i)}\) such that

\begin{equation}
    g_{\mu\nu} V^\mu _{(\alpha)} V^\nu _{(\beta)} = \eta_{(\alpha) (\beta)}
\end{equation}

where the brackets around the indices denote the fact that they label four vectors, not the components of a tensor.

We can choose the vectors \(V_{(i)}^\mu\) so that they are Fermi-Walker transported along the worldline defined by \(u^\mu\): this allows us to find the relativistic equivalent of a nonrotating frame of reference.

It is useful to project tensors onto the space-like and time-like subspaces defined by our tetrad (and we wish to do so in a coordinate-independent manner,  so just taking the 0th and $i $-th components in the tetrad will not suffice). We therefore define the projectors:

\begin{equation}
    h_{\mu \nu} = u_\mu u_\nu + g_{\mu \nu} \qquad \pi_{\mu\nu} = -u_\mu u_\nu
\end{equation}

respectively onto the space- and time-like subspaces.

\paragraph{Killing vector fields}

Following \cite[section 25.2, page 650]{MisnerThorneWheeler:1973}.
Say there is a certain direction (for simplicity, along one of our coordinate axes) along which the metric is preserved: an \(\widetilde{\alpha}\) such that \(\partial_{\widetilde{\alpha} g_{\mu\nu}}\).

Then the metric properties of curves along the manifold are unchanged if we shift their coordinate representation by a constant along the \(\widetilde{\alpha}\) coordinate axis.

Let us call the direction of this translation \(\xi^\mu = \delta^\mu_{\widetilde{\alpha}}\) if we use this coordinate system. It can be shown by direct computation that

\begin{equation} \label{eq:killing-vector-identity}
    \nabla_{\nu} \xi_\mu = \frac{1}{2} \qty(\partial_{\widetilde{\alpha}}
    g_{\mu\nu} + \partial_\nu g_{\mu\widetilde{\alpha}} -
    \partial_\mu g_{\nu \widetilde{\alpha}})
\end{equation}

but by hypothesis the first term on the RHS of \eqref{eq:killing-vector-identity} is zero, therefore we have shown that \(\nabla_{\nu} \xi_\mu = \nabla_{[\nu} \xi_{\mu]}\) in this coordinate frame, but since this is a covariant equation it extends to every other one.

It can also be seen this way that \(\nabla_{(\nu} \xi_{\mu)}=0\): this is called \emph{Killing's equation}. This is useful since: given a geodesic \(x^\mu(\lambda)\), for which we define \(u^\mu = \dv*{x^\mu}{\lambda} \), it must be the case that \(u^\nu \nabla_\nu u^\mu = 0 \). Then, the component of \(u^\mu\) along \(\xi^\mu\) (\(u^{\widetilde{\alpha}} = u^\mu \xi_\mu\)) is conserved:

\begin{equation}
    \dv{}{\lambda} \qty(u^\mu \xi_\mu) = u^\nu \nabla_\nu \qty(u^\mu \xi_\mu)
    = \cancelto{0}{\xi^\mu u^\nu \nabla_\nu u_\mu} + \cancelto{0}{u^\nu u^\mu \nabla_\nu \xi_\mu} \equiv 0
\end{equation}

\paragraph{Lie derivative}

Following \cite[section 6]{Taub:1978}. The Lie derivative of a generic tensor \(T^{\mu_1 \dots \mu_n} _{\nu_1 \dots \nu_m}\) along a vector \(\xi^\mu\) is defined as:

\begin{equation}
    \Lie_{\xi} T^{\mu_1 \dots \mu_n} _{\nu_1 \dots \nu_m} \defeq
    \xi^\rho \nabla_\rho T^{\mu_1 \dots \mu_n} _{\nu_1 \dots \nu_m}
    + \sum_{i=1}^{m} T^{\mu_1 \dots \mu_n} _{\nu_1 \dots \nu_{i-1} \rho \nu_{i+1} \dots \nu_m} \nabla_{\nu_i} \xi^\rho
    - \sum_{i=1}^{n} T^{\mu_1 \dots \nu_{i-1} \rho \nu_{i+1} \dots \mu_n} _{\nu_1 \dots  \nu_m} \nabla_{\rho} \xi^{\mu_i}
\end{equation}

Some special cases are: a scalar \(\Lie _\xi f = \xi^\rho \partial_\rho f \), a vector \(\Lie _\xi u^\mu = \xi^\rho \nabla_\rho u^\mu -u^\rho \nabla_\rho \xi^\mu\), and  an antisymmetric covariant two-tensor \(\omega_{\mu\nu} = \omega_{[\mu \nu]}\) which represents a closed form \(\nabla_{[\mu} \omega_{\nu \rho]} = 0\) we have \(\Lie _\xi \omega_{\mu\nu} =2\nabla_{[\nu} \qty( \omega_{\mu] \rho} \xi^\rho )\)

\paragraph{Useful metrics}

The simplest physically relevant one is the Schwarzschild metric. It describes a spherically symmetric object of mass $M$, in spherical coordinates. Defining $\Phi = -M/r$, we have:

\begin{equation}
    \dd{s}^2 = -(1+2\Phi)\dd{t}^2 + \frac{1}{1+2\Phi} \dd{r}^2
    + r^2 \qty(\dd{\theta}^2 + \sin^2\theta \dd{\varphi}) \label{eq:schwartzshild-line-element}
\end{equation}

or, equivalently,

\begin{equation}
    g_{\mu\nu} =  \diag\qty(-(1+2\Phi),\, \frac{1}{1+2\Phi},\, r^2,\, r^2 \sin ^2 \theta )
\end{equation}

We can see that it approaches the flat metric $\eta_{\mu\nu} = \diag\qty(-, +, +, +)$ in the limit $M\rightarrow 0$. Its determinant is $g = -r^4 \sin^2 \theta$.

\subsection{Fluid mechanics}

In usual relativistic single-body mechanics, we use the 4-velocity $u^\mu$ and the corresponding 4-momentum $p^\mu = m u^\mu$. The 0-th component of this vector is the energy of the body, while the $i$-th components are its momentum: we then have $p^\mu p_\mu = m^2 = E^2 - \abs{p}^2$.

When dealing with a continuum, we will have a certain density of particles per unit of volume, we call this $n$. The current of particles is then $N^\mu = n u^\mu$. If these particles have a certain rest mass $m_0$, we can then define the vector $\rho_0 u^\mu = m_0 n u^\mu = m_0 N^\mu$.

This satisfies a conservation equation: $\nabla_\mu(\rho_0 u^\mu) = 0$.

Particles in a fluid can have three kinds of energy we concern ourselves with: mass, kinetic energy and other forms of energy (thermal, chemical, nuclear\dots).
We can always perform a change of coordinates to bring us to a frame in which the kinetic energy is zero. We write the sum of the other two forms of energy as $\rho = \rho_0 (1+\epsilon)$. So, $\epsilon$ is the ratio of the internal non-mass energy to the mass.

\paragraph{Stress-energy tensor}

The stress-energy tensor \(T^{\mu\nu}\) is a \((2,0)\) tensor, whose \(\mu, \nu\) component is the flow of the \(\mu\)-th component of four-momentum \(p^\mu\) through a surface of constant coordinate \(x^\nu\).

Because of our choice of metric signature, the spatial part of the tensor corresponds to the \emph{negative} of the classical continuum-mechanics stress tensor: \(T^{ij} = - \sigma^{ij}\), sice that tensor describes the stresses \emph{on} the ``box'' of fluid \cite[]{Moretti:2016}.

For a gas of non-interacting particles, the stress-energy tensor is very simple: the momentum density is \(\rho u^\mu\), and then to look at the flow through a surface of constant \(x^\nu\) we just need to multiply by \(u^\nu\), so in the Local Rest Frame we have:

\begin{equation}
    T^{\mu\nu} = \rho u^\mu u^\nu = \begin{bmatrix}
    \rho    & 0  &  0 & 0 \\
      0 & 0  & 0  & 0 \\
      0 & 0  & 0  & 0 \\
      0 & 0  & 0  & 0
    \end{bmatrix}
\end{equation}

% Now, the vector $\rho u^\mu$ describes the flux of energy.
% We can then write the equation for the conservation of momentum:
%
% \begin{equation}
%     f^\mu = \nabla_\nu (\rho u^\mu u^\nu)
% \end{equation}

\paragraph{Nonrelativistic fluid mechanics}

Nonrelativistic fluid mechanics are described by the equations:

\begin{subequations}
\begin{align}
    \partial_t \rho + \partial_i (\rho v^i) &= 0 &\text{conservation of mass} \\
    \rho \qty(\partial_t v^i + v^j \partial_j v^i) &= \partial_j \sigma^{ij} &\text{conservation of momentum}  \\
    \rho \partial_t E + v^i \partial_i E &= \partial_i \qty(\sigma^{ij}v_j + \kappa\partial^i T) &\text{conservation of energy}
\end{align}
\end{subequations}

where $\rho$ is the density of the fluid,
$v^i$ are the components of its velocity,
$\sigma^{ij}$ is the classical stress tensor (or, equivalently, the \emph{negative} of the  space-like components of the energy-momentum tensor),
$E$ is the energy of the fluid,
$\kappa$ is the thermal conductivity,
$T$ is the temperature of the fluid.

The nonrelativistic stress tensor can be written as:

\begin{equation}
    \sigma_{ij} = -(p - \xi \partial_k v^k ) \delta_{ij} + 2 \eta \partial_{(i} v_{j)}
\end{equation}

\begin{greenbox}
  I flipped the sign of the term \(\xi \partial_k v^k\) since I think it is a typo in \cite[page 301]{Taub:1978}: the pressure and compression viscosity terms should have opposite signs like in \eqref{eq:components-stress-energy-tensor}, right?
\end{greenbox}

where $p$ is the (isotropic) pressure, $\eta$ the viscosity, $\xi$ is the compression viscosity. We are assuming that the normal stresses are only those exerted by pressure, so the diagonal terms $\sigma_{ii}$ (not summed) must just be $-p$. So, the term $-\xi \partial_k v^k$ must equal $\eta \partial_{(i} v_{i)} = 2\eta \partial_i v_i$ (not summed). Therefore, by isotropy, $\xi = 2\eta/3$.

Note that we are working in Euclidean 3D space, so the metric is the identity and upper and lower indices are equivalent.

The energy is a sum of kinetic and specific energy:

\begin{equation}
    E = v^i v_i /2 + \varepsilon
\end{equation}

where $\varepsilon$ is the specific energy (of a type that is different from kinetic) per unit mass.

\paragraph{Surfaces in space-time and acceleration decomposition}

Following \cite[section 4]{Taub:1978}
We look at 4D space time in 3D space-like slices: if we have a fluid moving with velocity \(u^\mu\), we can look at the solutions of the associated differential equation: \(x^\mu (\xi^i, s)\), where \(\xi^i\) are the 3D coordinates of the starting position and \(s\) is the time at which we look at the solution. Then we can look at the ``starting'' hypersurface \(\Sigma = \qty{x^\mu (\xi^i, 0)}\).

Say we have a curve \(\xi^i(\tau)\) in \(\Sigma\). Then we can look at the two-dimensional surface defined by the evolution of \(\xi^i(\tau)\): \(x^{\mu} (\xi^i(\tau), s) = x^\mu (\tau, s)\). If we define the ``spatial'' tangent vector \(\lambda^\mu = \dv*{x^\mu}{\tau} \), it follows from Schwarz's theorem that:

\begin{equation} \label{eq:schwarz-spacetime-tube}
    \pdv{x^\mu}{\tau}{s} =
    \pdv{x^\mu}{s}{\tau}
    \implies
    \pdv{u^\mu}{\tau} = \pdv{\lambda^\mu}{s}
\end{equation}

Now let us take the spatial vectors of an orthonormal Fermi-Walker transported tetrad \(V^\mu_{(a)}\) as described in \Nameref{par:tetrads}, and express \(\lambda^\mu\) in this frame: its covariant components will be

\begin{equation} \label{eq:tetrad-components-lambda}
    X_{(a)} = V_{(a)\mu} \lambda^\mu
\end{equation}

If we differentiate \eqref{eq:tetrad-components-lambda} wrt \(s\), and use \eqref{eq:schwarz-spacetime-tube} with the fact that \(\dv{}{\tau} = \lambda^\mu \nabla_\mu \), we get:

\begin{subequations}
\begin{align}
    \dv{X_{(a)}}{s} &= \dv{V_{(a)\mu}}{s} \lambda^\mu + V_{(a)\mu} \dv{\lambda^\mu}{s}  \\
    &= V_{(a)}^\rho \cancelto{0}{\lambda^\mu u_\mu} a_\rho - \cancelto{0}{V_{(a)}^\rho  u_\rho} \lambda^\mu a_\mu
    + V_{(a)}^\nu \nabla_\mu u_\nu  \\
    &= V^\rho _{(a)} \lambda^\mu \nabla_\mu u_\rho  \\
    &= \qty(\nabla_\mu u_\rho) V^\rho_{(a)} V^{\mu}_{(b)} X^{(b)}
\end{align}
\end{subequations}

where in the last step we expressed everything wrt the tetrad coordinate system. Therefore, in those cordinates, the evolution of the components \(X^{(a)}\) is linear, and defined by the tetrad components of the two-form \(\nabla_\mu u_\nu\). So, we want to decompose this tensor:

\begin{equation} \label{eq:covariant-acceleration-decomposition}
    \nabla_\sigma u_\tau =
    \underbrace{\omega_{\sigma \tau}}_{\substack{\text{spatial} \\ \text{rotation}}}
    + \underbrace{\sigma_{\sigma\tau}}_{\substack{\text{spatial} \\ \text{shear}}}
    + \underbrace{\frac{1}{3} \theta h_{\sigma\tau}}_{\substack{\text{spatial} \\
    \text{compression}}}
    - \underbrace{ a_\tau u_\sigma}_{\substack{\text{temporal} \\ \text{acceleration}}} \\
\end{equation}

\begin{enumerate}
    \item \(\theta = \nabla_\mu u^\mu\) is the bare trace of the tensor, corresponding to the expansion velocity;
    \item \(a_\mu = u^\nu \nabla_\nu u^\mu\) is the covariant acceleration;
    \item \(\sigma_{\sigma \tau} = \qty(\nabla_{(\mu} u_{\nu)}) h^\mu_\sigma h^\nu _\tau - \frac[i]{1}{3} \theta h_{\sigma \tau} = \nabla_{(\sigma} u_{\tau)} + a_{(\sigma} u_{\tau)} - \frac[i]{1}{3} \theta h_{\sigma \tau} \) is the spatial symmetric trace-free part of the tensor, that is, the shear stress;
    \footnote{A trick for this computation: why would a term like \(u^\mu u_\sigma \nabla_\tau u_\mu\) be zero? We can just multiply by \(1 = - u^\tau u_\tau\) to get \(-u^\mu u_\sigma u_\tau u^\tau \nabla_\tau u_\mu = \sum_\tau u_\sigma u_\tau a_\mu u^\mu = 0\).
    \begin{greenbox}
        Is this right? It seems like we shouldn't be able to do this... The index \(\tau\) in the starting formula was not summed and then it is so this does not sound convincing. Is there another way to see that \(u^\mu u_{(\sigma} \nabla_{\tau)} u_\mu = 0\)?
    \end{greenbox}}
    \item \(\omega_{\sigma \tau} = h^\nu_\sigma h^\mu_\tau \nabla_{[\nu} u_{\mu]} = \partial_{[\tau} u_{\sigma]} + a_{[\tau} u_{\sigma]}\) is the spatial rotation tensor.
\end{enumerate}

We can describe the rotation with a ``vorticity vector'':

\begin{equation}
    \omega^\mu = \frac{1}{2 \sqrt{-g}} \varepsilon^{\mu\nu\sigma\tau} u_\nu \partial_\tau u_{\sigma} = \frac{1}{2} \eta^{\mu\nu\sigma\tau} u_\nu \partial_\tau u_{\sigma}
\end{equation}

where we define the fully antisymmetric, covariant tensor \(\eta^{\mu\nu\sigma\tau} = \varepsilon^{\mu\nu\sigma\tau} / \sqrt{-g} \). With its indices lowered, it is
\(\eta_{\mu\nu\sigma\tau} = - \varepsilon_{\mu\nu\sigma\tau} \sqrt{-g} \).

Note, from \cite[pages 51--52]{Carroll:1997ar}: this is the volume form of the manifold, and it is defined this way since \(g\) is a \emph{tensor density} of weight $-2$, while the bare Levi-Civita symbol is a density of weight $+1$.

\begin{greenbox}
  The signs in \cite[]{Taub:1978} and \cite[]{Carroll:1997ar} seem to disagree though!
\end{greenbox}

By the antisymmetry in the definition we can immediately see that \(\omega^\mu u_\mu=0\). It holds that \(\omega^\mu \equiv 0 \iff u_\mu = \rho \partial_\mu f \) (locally!) for scalar \(\rho\), \(f\), since this is equivalent to \(u_\mu\) being a closed form.

\(\omega_\mu\) and \(\omega_{\mu \nu}\) are ``dual'' in the sense that

\begin{equation}
    \omega _{\sigma\tau} = u^\mu \omega^\nu \eta_{\mu \nu \sigma \tau}
\end{equation}

and some other useful identities can be found in equations 7.5 through 7.7 of \cite[]{Taub:1978}.

\paragraph{Wave velocity}

Following \cite[section 5]{Taub:1978}.

An equation in the form \(\varphi (x^\mu) = 0\) defines a 3D surface \(\Sigma\); its constant-\(x^0\) slices are 2D surfaces. We can decompose \(\partial_\mu \varphi\) in a component proportional to the velocity, \(a u_\mu\), and one which is orthogonal to it, \(W_\mu = h^\sigma_\mu \partial_\sigma \varphi\). Then \(W_\mu W^\mu = h^\sigma_\mu  h_\beta^\mu \partial_\sigma \varphi \partial^\beta \varphi = h^{\sigma\beta} \partial_\sigma \varphi \partial_\beta \varphi\)
by the idempotency of the projector \(h\).
Thus we can see that \(W^2 \geq 0\), therefore it is a spacelike vector. We can define its corresponding unit vector: \(W^\mu = t^\mu \sqrt{W^\nu W_\nu}\).

We can choose a velocity \(v\) such that \(k^\mu = u^\mu - v t^\mu\) is parallel to \(\Sigma\), or \(k^\mu \partial_\mu \varphi = 0\). If this condition is satisfied, then \(v\) is the wave velocity of \(\Sigma\) as measured by an observer with velocity \(u^\mu\).

\begin{figure}[ht]
  \centering
  \incfig{figures/taub_wave_velocity}
  \caption{Wave velocity diagram}
  \label{fig:taub_wave_velocity}
\end{figure}

Now, we can multiply the equation \(k^\mu = u^\mu - v t^\mu\) by \(\partial_\mu \varphi\): we get \(v t^\mu \partial_\mu \varphi = u^\mu \partial_\mu \varphi\). What multiplies \(v\) in the LHS of this equation is the length of the spatial component of \(\partial_\mu \varphi\), or \(\sqrt{h^{\mu\nu} \partial_\mu \varphi \partial_\nu \varphi}\).
Using the definition of \(h ^{\mu \nu} \), we can arrive at

\begin{equation} \label{eq:lorentz-factor-wave-velocity}
    \gamma^{-2} = 1 - v^2 = \frac{g^{\mu\nu} \partial_\mu \varphi \partial_\nu \varphi}{h^{\mu\nu} \partial_\mu \varphi \partial_\nu \varphi}
\end{equation}

The denominator in \eqref{eq:lorentz-factor-wave-velocity} is positive, so we can see that \(1-v^2\) is positive iff \(\partial_\mu \varphi\) is spacelike, and \(v^2 = 1\) iff it is null.

If we are dealing with a timelike surface \(\Sigma\), or equivalently \(\partial_\mu \varphi\) is spacelike, then we define \(n^\mu \propto \partial_\mu \varphi\) such that \(n^\mu n_\mu = 1\) and we find:

\begin{equation}
    v = \frac{u^\mu \partial_\mu \varphi}{\sqrt{h^{\mu\nu} \partial_\mu \varphi \partial_\nu \varphi}}
    = \frac{u^\mu n_\mu}{\sqrt{h^{\mu\nu} n_\nu n_\mu}}
    = \frac{u^\mu n_\mu}{\sqrt{1 + (u^\mu n_\mu)^2}}
\end{equation}

From this it can be shown that \(1-v^2 = (1+(u^\mu n_\mu)^2)^{-1}\), and then \(v \gamma = u^\mu n_\mu\).

\paragraph{Relativistic non-ideal fluid dynamics}

The dynamics of the fluid are described by the conservation of the stress-energy tensor \(\nabla_\mu T ^{\mu \nu} =0\) and the conservation of mass \(\nabla_\mu (\rho_0 u^\mu) =0\).

Any stress-energy tensor can be decomposed in its space and time-like parts in the local rest frame of the fluid:

\begin{equation} \label{eq:stress-energy-tensor-decomposition}
    T_{\mu \nu} = w u_\mu u_\nu + w_\mu u_\nu + u_\mu w_\nu + w _{\mu \nu}
\end{equation}

where

\begin{subequations} \label{eq:components-stress-energy-tensor}
\begin{align}
  w &=  T _{\mu \nu} u^\mu u^\nu = \rho_0 (1 + \varepsilon) = \rho & \text{rest mass} \\
  w_\mu &= T _{\nu \sigma} h^\sigma _\mu u^\nu  = -\kappa h_\mu ^\sigma  \qty(\partial_\sigma T + T a_\sigma) & \text{heat conduction} \\
  w_{\mu \nu} &= T_{\rho \sigma} h^\rho _\mu h^\sigma_\nu = \qty(p - \xi \theta) h_{\mu \nu} - 2 \eta \sigma_{\mu \nu}  & \text{pressure and viscous stresses}
\end{align}
\end{subequations}

For the definition of the acceleration, vorticity etc. see equation \eqref{eq:covariant-acceleration-decomposition}.
As in the nonrelativistic section \(\eta\) is the viscosity, \(\xi\) is the compression viscosity,  \(\kappa\) is the thermal conductivity, \(T\) is the temperature field, \(p\) is the pressure, \(\rho_0\) is the rest mass density while \(\rho = \rho_0 (1 + \varepsilon)\) is the rest energy.

\begin{greenbox}
    The fact that the decomposition can be written in this way is not proven in \cite[]{Taub:1978}, but it seems like a proof can be found in \cite[]{Eckart:1940}, which is locked behind a paywall.
\end{greenbox}

Equivalently, we can write

\begin{subequations} \label{eq:stress-energy-tensor-decomposition-2}
\begin{align}
  T_{\mu\nu} &= T_{\mu\nu}^p && - T_{\mu\nu}^V &&+ T_{\mu\nu}^h  \\
  &= w u_\mu u_\nu + p h_{\mu\nu} &&-\xi \theta h_{\mu\nu} - 2 \eta \sigma_{\mu\nu} &&+2w_{(\mu} u_{\nu)}  \\
  &\text{perfect fluid} && \text{viscous stresses} && \text{heat conduction}
\end{align}
\end{subequations}

It holds \cite[]{Taub:1948} that:

\begin{equation}
0
\leq 3p
\leq \frac{3}{2}p + \sqrt{\qty(\frac{3}{2}p)^2 + \rho_0^2}
\leq \rho = \rho_0 (1 + \varepsilon)
\leq \rho_0 + 3p
\end{equation}

\paragraph{Forces}

If we apply \(h^\sigma_\mu \nabla_\nu\) to the formulation of the stress-energy tensor given in \eqref{eq:stress-energy-tensor-decomposition-2}, that is, look at the spatial components of the conservation equations, we get:

\begin{subequations}
\begin{align}
  h^\sigma_\mu \nabla_\nu \qty((p+\rho) u^\mu u^\nu + p g^{\mu\nu}) &=
  h^\sigma_\mu \nabla_\nu \qty(T^{\mu\nu}_V - T^{\mu\nu}_h)  \\
  (\rho + p) a^\sigma + h^\sigma_\nu \partial^\nu p &= h^\sigma_\mu \nabla_\nu \qty(
  +\xi \theta h^{\mu\nu} + 2 \eta \sigma^{\mu\nu} - 2w^{(\mu} u^{\nu)})  \\
  &= \underbrace{\nabla_\nu T^{\mu\nu}_V - u^\sigma \qty(\frac{\xi \theta^2}{3} +2 \sigma_{\mu\nu} \sigma^{\mu\nu})}_{\substack{\mathscr F ^\sigma _V}}
  \underbrace{ - \nabla_\nu (w^\sigma u^\nu) - w^\nu \nabla_\nu u^\sigma + a_\mu w^\mu u^\sigma  }_{\substack{\mathscr F ^\sigma _h}} \label{eq:full-spatial-equations-of-motion}
\end{align}
\end{subequations}

\begin{greenbox}
  I have not done the whole computation yet. Shouldn't there be an \(\eta\) in front of \(\sigma^2\)? Is it a typo in \cite[]{Taub:1978}?
  The heat part seems to make sense, I'm missing how \(-w^\mu u^\nu \nabla_\nu u_\mu = u_\mu (u^\nu \nabla_\nu w^\mu + w^\nu \nabla_\nu u^\mu)\): my results are

  \begin{subequations}
  \begin{align}
    h^\sigma _\mu \nabla_\nu \qty(2 w^{(\mu} u^{\nu)}) &=  (\delta_\mu ^\sigma + u^\sigma u_\mu) \nabla_\nu \qty(w^\mu u^\nu + u^\mu w^\nu)  \\
    &= \nabla_\nu \qty(w^\sigma u^\nu) + \nabla_\nu (u^\sigma w^\nu)
    + u^\sigma u_\mu \nabla_\nu \qty(w^\mu u^\nu + u^\mu w^\nu)  \\
    &= \nabla_\nu \qty(w^\sigma u^\nu) + \nabla_\nu (u^\sigma w^\nu)
    + u^\sigma u_\mu \qty(\cancelto{0}{ w^\mu \nabla_\nu u^\nu } + u^\nu \nabla_\nu w^\mu + w^\nu \nabla_\nu u^\mu + u^\mu \nabla_\nu w^\nu)  \\
    &= \nabla_\nu \qty(w^\sigma u^\nu) + \nabla_\nu (u^\sigma w^\nu)
    + u^\sigma u_\mu \qty(u^\nu \nabla_\nu w^\mu + w^\nu \nabla_\nu u^\mu) - u^\sigma \nabla_\nu w^\nu  \\
    &= \nabla_\nu \qty(w^\sigma u^\nu) + w^\nu \nabla_\nu u^\sigma
    + u^\sigma u_\mu \qty(u^\nu \nabla_\nu w^\mu + w^\nu \nabla_\nu u^\mu) \\
    &\overset{?}{=} \nabla_\nu \qty(w^\sigma u^\nu) + w^\nu \nabla_\nu u^\sigma
    - u^\sigma w_\mu a^\mu
  \end{align}
  \end{subequations}

  so the first two terms are there, the rest does not look right.
\end{greenbox}

The vectors \(\mathscr F^\sigma _{h, V}\) are relativistic forces on the fluid due respectively to heat flow and viscosity.

In the perfect fluid case the RHS is zero we simply get Euler's equation from \eqref{eq:full-spatial-equations-of-motion}.

\paragraph{The Second Principle in GR}


Because of the conservation of the stress-energy tensor, we have:

\begin{equation} \label{eq:velocity-stress-energy-tensor-identity}
    \nabla_\nu \qty(u_\mu T ^{\mu \nu}) = T^{\mu \nu} \nabla_\nu u_\mu
\end{equation}

Let us also consider the expression of the differential per-unit-rest-mass entropy:

\begin{equation} \label{eq:differential-entropy}
    T\dd{S} = \dd{\varepsilon} + p \dd{\frac{1}{\rho_0}}
\end{equation}

\begin{claim}
    We can mold equation \eqref{eq:velocity-stress-energy-tensor-identity} into a version of the second principle of thermodynamics

    \begin{equation} \label{eq:second-principle-thermodynamics}
        T \nabla_\mu S^\mu = \xi \theta^2 + 2 \eta \sigma_{\mu\nu} \sigma^{\mu\nu} + \frac{w^\mu w_\mu}{\kappa T} \geq 0
    \end{equation}

    where we define \(S^\mu = \rho_0 S u^\mu + w^\mu /T\).
\end{claim}

\begin{proof}
    We will need the decompositions of
    the derivative of velocity \eqref{eq:covariant-acceleration-decomposition},
    of the stress-energy tensor \eqref{eq:stress-energy-tensor-decomposition}, \eqref{eq:components-stress-energy-tensor},
    the expression of differential entropy \eqref{eq:differential-entropy}
    and the conservation of mass \(\nabla_\mu (\rho_0 u^\mu) = 0\).

    First of all, the LHS of \eqref{eq:velocity-stress-energy-tensor-identity} can be greatly simplified by noticing that \(u^\mu w_\mu = u^\mu w_{\mu \nu} = 0\), so it becomes \footnote{One may think to expand \(\nabla_\nu w^\nu\) and I had, bringing along many useless terms, when actually it can be kept this way and will just cancel later on.}

    \vspace{-1cm}

    \begin{subequations}
    \begin{align}
        \nabla_\nu \qty(u_\mu T ^{\mu \nu}) &= \nabla_\nu \left( w \underbrace{u_\mu u^\mu}_{-1} u^\nu + \underbrace{u_\mu w^\mu}_{0} u^\nu + \underbrace{ u_\mu u^\mu}_{-1} w^\nu + \underbrace{u_\mu w ^{\mu \nu}}_{0}\right) \\
         &= \nabla_\nu \qty(- u^\nu \rho_0 (1 + \varepsilon) - w^\nu) \\
         &=  - \rho_0 u^\nu \partial_\nu \varepsilon - \nabla_\nu w^\nu
         % &= - \rho_0 u^\nu \partial_\nu \varepsilon + \kappa \qty( \nabla_\nu \qty(\partial^\nu T + T a^\nu) + \nabla_\nu \qty(u^\nu u_\sigma \qty(\partial^\sigma T + T a^\sigma)))  \\
         % &= - \rho_0 u^\nu \partial_\nu \varepsilon + \kappa \qty(
         % \qty(\nabla^\nu + 2 a^\nu + \theta u^\nu + u^\alpha u^\nu \nabla_\alpha
         % ) \partial_\nu T + T \nabla_\nu a^\nu
         % )
    \end{align}
    \end{subequations}

    In the RHS of \eqref{eq:velocity-stress-energy-tensor-identity} as well many terms are cancelled because they contain contractions of space and timelike indices: we get

    \begin{subequations}
    \begin{align}
      \qty(\nabla_\nu u_\mu) T^{\mu \nu} &=
      \qty(\omega_{\nu \mu} + \sigma_{\nu\mu} + \frac{1}{3} \theta h_{\nu\mu}
      -  a_\mu u_\nu) \qty(w u^\mu u^\nu + w^\mu u^\nu + u^\mu w^\nu + w ^{\mu \nu})  \\
      &= w^{\mu\nu} \qty(\omega_{\mu\nu} + \sigma_{\mu\nu} + \frac{\theta h_{\mu\nu}}{3}) + a_\mu w^\mu  \\
      &= \qty(\qty(p - \xi \theta) h^{\mu \nu} - 2 \eta \sigma^{\mu \nu})
      \qty(\omega_{\mu\nu} + \sigma_{\mu\nu} + \frac{\theta h_{\mu\nu}}{3})
      + a_\mu \qty(-\kappa h^\mu _\sigma  \qty(\partial^\sigma T + T a^\sigma))  \\
      &= \qty(p - \xi \theta) \theta - 2 \eta \sigma_{\mu\nu} \sigma^{\mu\nu}
      - \kappa a_\mu \partial^\mu T - \kappa T a_\mu a^\mu
    \end{align}
    \end{subequations}

    So far, we have:

    \begin{equation} \label{eq:step1-second-principle-proof}
        - \rho_0 u^\nu \partial_\nu \varepsilon %+ \kappa \qty(
    %    \qty(\nabla^\nu + 3 a^\nu + \theta u^\nu + u^\alpha u^\nu \nabla_\alpha
    %    ) \partial_\nu T + T (\nabla_\nu + a_\nu) a^\nu
    %    )
        - \nabla_\nu w^\nu
        = \qty(p - \xi \theta) \theta - 2 \eta \sigma_{\mu\nu} \sigma^{\mu\nu}
        - \kappa a_\mu \partial^\mu T - \kappa T a_\mu a^\mu
    \end{equation}


    Let us rearrange \eqref{eq:step1-second-principle-proof} in a convenient way:

    \begin{equation} \label{eq:step2-second-principle-proof}
        + \rho_0 u^\nu \partial_\nu \varepsilon + p \theta =
        % - \kappa \qty(
        % \qty(\nabla^\nu + 3 a^\nu + \theta u^\nu + u^\alpha u^\nu \nabla_\alpha
        % ) \partial_\nu T + T (\nabla_\nu + a_\nu) a^\nu
        % )
        - \nabla_\nu w^\nu
        + \xi \theta^2 + 2 \eta \sigma_{\mu\nu} \sigma^{\mu\nu}
        + \kappa a_\mu \partial^\mu T + \kappa T a_\mu a^\mu
    \end{equation}


    Now let us consider a quantity we wish to obtain from these manipulations:
    \(T \nabla_\mu S^\mu\). It can be expanded using the continuity equation into:

    \begin{equation} \label{eq:entropy-vector-identity}
        T \nabla_\mu S^\mu
        = T \nabla_\mu \qty(\rho_0 S u^\mu + \frac 1T w^\mu )
        = T \rho_0 u^\mu \partial_\mu S + \nabla_\mu w^\mu - w^\mu \frac{\nabla_\mu T}{T}
    \end{equation}

    Let us turn the differentials in \eqref{eq:differential-entropy} into proper-time derivatives: \(\dd \rightarrow \dv*{}{\tau} = u^\mu \partial_\mu \). Also, we can use the continuity equation to see that \(u^\mu \partial_\mu \rho_0 = - \rho_0 \theta\).
    Then \eqref{eq:differential-entropy} becomes:

    \begin{equation} \label{eq:differential-entropy-explicit}
        T \dv{S}{\tau}  = \dv{\varepsilon}{\tau} - \frac{p}{\rho_0^2} \dv{\rho_0}{\tau}  =\dv{\varepsilon}{\tau} + \frac{p \theta}{\rho_0}
    \end{equation}

    Then we can write the LHS of \eqref{eq:step2-second-principle-proof}, using the identities in
    equation \eqref{eq:entropy-vector-identity} and \eqref{eq:differential-entropy-explicit}:

    \begin{equation} \label{eq:step3-second-principle-proof}
        \rho_0 \qty(u^\nu \partial_\nu \varepsilon + \frac{p \theta}{\rho_0})
        = \rho_0 T u^\nu \partial_\nu S
        = T\nabla_\mu S^\mu - \nabla_\mu w^\mu + \frac{1}{T} w^\mu \nabla_\mu T
    \end{equation}

    Let us substitute \eqref{eq:step3-second-principle-proof} into \eqref{eq:step2-second-principle-proof},
    and then subtract the desired result \eqref{eq:second-principle-thermodynamics}  from the equation: this way, if we get an identity the proof will be complete (this may seem circular, but it is done just for convenience in the algebraic manipulations: to get a more rigorous argument one may just reverse the steps, using the identity \eqref{eq:step-second-principle-proof-after-subtraction} in equation \eqref{eq:step-second-principle-proof-before-subtraction} to get equation \eqref{eq:second-principle-thermodynamics}).

    \begin{subequations}
    \begin{align}
      T\nabla_\mu S^\mu - \nabla_\mu w^\mu + \frac{1}{T} w^\mu \nabla_\mu T &=
      - \nabla_\nu w^\nu
      + \xi \theta^2 + 2 \eta \sigma_{\mu\nu} \sigma^{\mu\nu}
      + \kappa a_\mu \partial^\mu T + \kappa T a_\mu a^\mu
      \label{eq:step-second-principle-proof-before-subtraction} \\
      - \nabla_\mu w^\mu + \frac{1}{T} w^\mu \nabla_\mu T &=
      - \nabla_\nu w^\nu
      + \kappa a_\mu \partial^\mu T + \kappa T a_\mu a^\mu
      - \frac{w^\mu w_\mu}{\kappa T} \label{eq:step-second-principle-proof-after-subtraction} \\
      + \frac{1}{T} w^\mu \nabla_\mu T &=
      + \kappa a_\mu \partial^\mu T + \kappa T a_\mu a^\mu
      - \frac{w^\mu w_\mu}{\kappa T} \label{eq:step4-second-principle-proof}
    \end{align}
    \end{subequations}

    The last term in \eqref{eq:step4-second-principle-proof} looks like:

    \begin{equation} \label{eq:step5-second-principle-proof}
        \frac{w^\mu w_\mu}{\kappa T} = \frac{1}{\kappa T }
        \kappa^2 h^\sigma _\mu h^{\mu\nu} \qty(\partial_\sigma T + T a_\sigma) \qty(\partial_\nu T + T a_\nu)
        = \kappa \qty(\frac{h^{\mu\nu}}{T} \partial_\mu T \partial_\nu T + 2 a^\mu \partial_\mu T + T a_\mu a^\mu)
    \end{equation}

    Inserting the identity in \eqref{eq:step5-second-principle-proof}  and making the last \(w^\mu\) explicit in \eqref{eq:step4-second-principle-proof} we get:

    \begin{subequations}
    \begin{align}
      -\frac{1}{T} \kappa h^\mu _\sigma  \qty(\partial^\sigma T + T a^\sigma) \partial_\mu T  &= + \kappa a_\mu \partial^\mu T + \kappa T a_\mu a^\mu
      -\kappa \qty(\frac{h^{\mu\nu}}{T} \partial_\mu T \partial_\nu T + 2 a^\mu \partial_\mu T + T a_\mu a^\mu) \\
      +\frac{1}{T} h^{\mu\nu} \partial_\nu T  \partial_\mu T + h^{\mu\nu} a_\nu \partial_\mu T  &= - a_\mu \partial^\mu T -  T a_\mu a^\mu
      + \qty(\frac{h^{\mu\nu}}{T} \partial_\mu T \partial_\nu T + 2 a^\mu \partial_\mu T + T a_\mu a^\mu) \\
      0  &= - a_\mu \partial^\mu T -  T a_\mu a^\mu
      + \qty( a^\mu \partial_\mu T + T a_\mu a^\mu)
    \end{align}
    \end{subequations}

    Thus we have proved the equation in \eqref{eq:second-principle-thermodynamics}, the inequality follows directly from the fact that we are considering square moduli of spacelike vectors, and the coefficients such as \(\xi\) are assumed to be positive.
\end{proof}

If we assume that the fluid is in equilibrium (\(\nabla_\mu S^\mu = 0\)) then we must have \(\theta = 0\) (no compression), \(\sigma_{\mu\nu} = 0 \) (no shear stresses), \(w_\mu=0\) (no heat conduction which is not equal to \(-\kappa T a_\mu\)... doesn't have the same ring to it).

\paragraph{Stationarity}

Will skip most of this section. A spacetime is stationary if it admits a timelike Killing vector field? \textcite[]{Ehlers:1971} proved that from a hypothesis of stationarity we get \(\nabla_\mu S^\mu = 0\).

\paragraph{Ideal fluids}

They are fluids with \(\eta=\xi=\kappa=0\), that is, without viscosity (neither compressive nor shear) nor heat transmission.
They are described by the following stress-energy tensor:

\begin{equation}
    T^{\mu\nu} = \rho u^\mu u^\nu + p h^{\mu\nu} = \rho_0 h u^{\mu} u^\nu + p g^{\mu\nu}
\end{equation}

where \(h = (p + \rho) / \rho_0\) is the enthalpy. If our fluid is ideal then the RHS of \eqref{eq:second-principle-thermodynamics} is zero and so is \(w^\mu\),
therefore \(T \nabla_\mu S^\mu = T \nabla_\mu \qty(\rho_0 S u^\mu) = T \rho_0 u^\mu \partial_\mu S\) by the continuity equation. So, unless \(\rho_0\) and \(T\) are not zero, \(S\) is conserved along the world-lines of the fluid.

Also, the RHS of \eqref{eq:full-spatial-equations-of-motion} is zero, therefore we get the Euler equation:

\begin{equation} \label{eq:relativistic-euler}
    (p+\rho) a^\mu = - h^{\mu \nu} \partial_\nu p
\end{equation}

\paragraph{Speed of sound}

Following \cite[]{Yoshida:2011}.
We want to justify the expression \(v_s^2 = \pdv*{p}{\rho} \). We work in Minkowski spacetime, where \(g_{\mu\nu} = \eta_{\mu\nu}\), and with an ideal fluid, for which \(T_{\mu\nu} = (p+ \rho) u_\mu u_\nu + p \eta_{\mu\nu}\). Then, the equations of conservation of mass and momentum read:

\begin{subequations}
\begin{align}
  \rho_0 \partial_\mu u^\mu + u^\mu \partial_\mu \rho_0 &= 0 & \text{mass}  \\
  u^\mu \qty(\partial_\mu \rho - h \partial_\mu \rho_0) &=0 & \text{momentum along } u^\mu  \\
  (p+\rho) a^\mu + h^{\mu \rho} \partial_\rho p &=0 & \text{momentum in the span of } h_{\mu\nu}
\end{align}
\end{subequations}

where \(h\) is the specific enthalpy. If we assume small perturbations \(p \rightarrow p + \delta p\), \(\rho_0 \rightarrow \rho_0 + \delta \rho_0\), \(\rho \rightarrow \rho + \partial \rho\), \(u^\mu \rightarrow (1, \delta u^x, 0, 0)\) (it is \emph{almost} normalized) we get our three equations, up to first order in the perturbations:

\begin{subequations}
\begin{align}
  \partial_x (\delta u^x) &= -\frac{\partial_t (\delta \rho_0)}{\rho_0}  \\
  \partial_t (\delta \rho) - h \partial_t (\delta \rho_0) &= 0  \\
  -(p+ \rho) \partial_t (\delta u^x) &= \partial_x p
\end{align}
\end{subequations}

We manipulate these by differentiating and substituting, in order to eliminate the dependence on \(\delta u^x\) and \(\rho_0\), and get:

\begin{subequations}
\begin{align}
  \partial_t \delta \rho + (p+\rho) \partial_x (\partial u^x) &= 0 \\
  \partial_x \delta p + (p+\rho) \partial_t (\partial u^x) &= 0
\end{align}
\end{subequations}

which simplifies to \(\partial_{tt}^2 \delta \rho - \partial_{xx}^2 \delta p = 0\): in order to get the wave equation in the form \(\qty(v_s^{-2} \partial_{tt}^2 - \partial_{xx}^2 ) \delta p = 0\) we \emph{need} to define \(v_s^2 = \pdv*{p}{\rho}\).

\paragraph{Barotropic flows}

The definition of a barotropic flow is: a flow for which the density is just a function of temperature, \(\rho = \rho (p)\).

\begin{greenbox}
  What is \(d\) in \cite[equation 11.10]{Taub:1978}? Is it a typo? I do not get that equation. Also, I quote: ``Equations 11.3 and 11.5 do not hold. However, it is a consequence of equations 11.3 and...''.

  Insert here: definition of the variable \(s\).
\end{greenbox}

The quantity \(s\) obeys the conservation law \(\nabla_\mu (s u^\mu) = 0\). We define a quantity \(Q = \log ((\rho + p)/s )\), which for \(s=\rho_0\) is the log-enthalpy. Then, the Euler equation \eqref{eq:relativistic-euler} can be substituted by

\begin{equation}
  a^\sigma = - h^{\sigma \mu} \partial_\mu Q
\end{equation}

or, more explicitly:

\begin{equation}
  (\rho + p) a^\mu = -s h^{\mu\nu} \partial_\nu \qty(\frac{\rho + p}{s})
\end{equation}

\begin{greenbox}
  Unproven in \cite[]{Taub:1978}!? where does it come from?
\end{greenbox}

We now define a ``current'' of \(\exp(Q)\): \(V^\mu = \exp(Q) u_\mu \).

Then we define \(\Omega_{\mu\nu} = 2 \nabla_{(\mu} V_{\nu)}\): this satisfies \(\Omega_{\mu\nu} u^\nu = - T \partial_\mu S\) in general, and \(\Omega_{\mu\nu} u^\nu = 0\) in the isentropic case.

\begin{greenbox}
  Right? Underneath equation \cite[11.14]{Taub:1978} what is meant is ``in the isentropic'' case, I think, since then \(\partial_\mu S = 0\)...

  Also, what does the distinction between the two definitions in 11.14 and 11.13 mean? Is 11.13 not a subcase of 11.14?
\end{greenbox}

Then, we define

\begin{equation}
  v^\mu = \frac{1}{2} \eta^{\mu \nu \rho \sigma} u_\nu \Omega_{\rho \sigma}
\end{equation}

which satisfies

\begin{equation}
  3 u_{[\alpha} \Omega_{\beta \gamma]} = v^{\mu} \eta_{\mu \alpha \beta \gamma}
\end{equation}

\begin{greenbox}
  The normalization does not seem right: \(\eta^{\mu\nu\rho\sigma} \eta_{\mu\alpha\beta\gamma} = -\delta^{\nu\rho\sigma}_{\alpha\beta\gamma}\), so when substituting in the result seems off by \(- 3!\)...
\end{greenbox}

We use this result to get an explicit formula for \(\Omega_{\alpha\beta}\) in terms of \(u^\mu\), \(v^\mu\).
This comes out by multiplying by \(u^\gamma\), and is:

\begin{equation}
  \Omega _{\alpha\beta}
  = - v^\mu u^\gamma \eta_{\mu\gamma\alpha\beta} + u_\alpha \Omega_{\beta\gamma} u^\gamma - u_\beta \Omega_{\alpha\gamma} u^\gamma
  = - v^\mu u^\gamma \eta_{\mu\gamma\alpha\beta} + 2T (u_{[\beta} \partial_{\alpha]} S )
\end{equation}

We can also rewrite the perfect-fluid stress-energy tensor as \(T^{\mu\nu} = s V^\mu u^\nu +p g^{\mu\nu}\): then, since the metric has zero covariant derivative, if we look at the projection of \(T^{\mu\nu}\) along a Killing vector field \(\nabla_{(\mu} \xi_{\nu)}=0\) we get

\begin{equation}
  \nabla_\nu \qty(\xi_\mu T^{\mu\nu}) = \nabla_\nu \qty(\xi_\mu s V^\mu u^\nu) = 0
\end{equation}

but in the (barotropic?) case \(s = \rho_0\) we can use the conservation of mass to get \(\rho_0 u^\nu \nabla_\nu (\xi_\mu V^\mu) = 0\), so the quantity \(H = \xi_\mu V^\mu\) is conserved along the worldlines of the fluid.

If we are in the coordinate system which would be the LRF for an observer with velocity \(\xi^\mu\), then the conserved quantity is \(H = V^0 \).

\paragraph{Shocks and conservation laws through boundaries}

Following \cite[section 13]{Taub:1978}.
The differential formulations of the conservation of mass and momentum will not hold at points of non-differentiability, such as through shock waves.
There, they must be replaced by an integral law.
We can reframe our conservation laws as the statements that, for any scalar \(f\) and vector \(\lambda_\mu\) we will have

\begin{subequations} \label{eq:conservation-laws-arbitrary-functions}
\begin{align}
  \nabla_\mu \qty(f \rho_0 u^\mu) &= \rho_0 u^\mu \partial_\mu f  \\
  \nabla_\mu \qty(\lambda_\nu T^{\mu\nu})&= T^{\mu\nu} \nabla_\mu \lambda_\nu
\end{align}
\end{subequations}

We can integrate these on a generic volume \(V\) using Stokes' theorem \eqref{eq:stokes-theorem}, choosing coordinates on the boundary such that the determinant of the induced metric on the submanifold is uniformly 1.

\begin{subequations} \label{eq:conservation-laws-arbitrary-functions-integral}
    \begin{align}
        \int _{\partial V} n_\mu \qty(f \rho_0 u^\mu) \dd[3]{y} &= \int_V \rho_0 u^\mu \partial_\mu f \sqrt{-g}  \dd[4]{x}  \\
        \int_{\partial V} n_\mu \qty(\lambda_\nu T^{\mu\nu}) \dd[3]{y}&= \int _V T^{\mu\nu} \nabla_\mu \lambda_\nu \sqrt{-g}  \dd[4]{x}
    \end{align}
\end{subequations}

Now, to deal with the shock we do the following: take the hypersurface at which the shock happens, enclose it in an infinitesimally thin 4D volume: then, as the volume decreases the RHSs of equations \eqref{eq:conservation-laws-arbitrary-functions-integral} go to 0, therefore the LHSs also must: we can write them as the difference of the integrands at the boundary.
Therefore, introducing the notation \([F] = F_+ - F_-\), selecting one of the two outward vectors and denoting only it as \(n_\mu\) and using the arbitrariness of \(f\), \(\lambda_\nu\) we have:

\begin{equation}
    \qty[n_\mu \rho_0 u^\mu] = n_\mu \qty[\rho_0 u^\mu] = 0
    \qquad
    \text{and}
    \qquad
    \qty[n_\mu T^{\mu\nu}] = n_\mu [T^{\mu\nu}] = 0
\end{equation}

We denote \(m = \rho_0 u^\mu n_\mu\) (at the boundary), and then the first equation is the fact that \(m\) is the same on either side:

\begin{equation} \label{eq:rankine-hugoniot-1}
    \rho_{0+} u^\mu_+ n_\mu =
    \rho_{0-} u^\mu_- n_\mu \defeq m
\end{equation}

The other, for an ideal flud, is

\begin{equation} \label{eq:momentum-conservation-across-boundary}
    m \qty(V^\mu_+ - V^\mu _-) + n^\mu \qty(p_+ - p_-) = 0
\end{equation}

with \(V^\mu = h u^\mu\). Now, consider a set of vectors \(Y^\mu\) such that \(Y^\mu n_\mu = 0\) and \(Y^\mu Y_\mu = 1\). These form a two-parameter family (\(\sim S^2\)), and
by \eqref{eq:momentum-conservation-across-boundary} must satisfy

\begin{equation}
    m \qty(V^\mu_+ - V^\mu _-) Y_\mu = 0
\end{equation}

then
\begin{itemize}
    \item either \(m = 0\): this is called a \emph{slip-stream} discontinuity, because the normal component of the velocity is zero, so the fluid is flowing \emph{along} the boundary, no matter is crossing it;
    \item  or \(V^\mu _+ Y_\mu = V^\mu _- Y_\mu\) for \emph{all} the \(Y^\mu\): this is called a \emph{shock wave}.
\end{itemize}

\begin{greenbox}
  There are \emph{three} independent \(Y^\mu\), right?
\end{greenbox}

In the shock wave case, we must write two (?) more equations to fully recover the \eqref{eq:momentum-conservation-across-boundary}. To do so, we define \(\tau = h/\rho_0\) and use the fact that \(V^2 = -h^2\).
Then, we multiply \eqref{eq:momentum-conservation-across-boundary} by \(V_{\mu}^{+}\) and \(V_{\mu}^{-}\), and use the fact that \(n_\mu V^\mu_{\pm} = m \tau_{\pm}\). We get

\begin{equation} \label{eq:rankine-hugoniot-2}
    h^2_+ - h^2_- - (p_+ - p_-)(\tau_+ - \tau_-) = 0
\end{equation}

then, by multiplying \eqref{eq:momentum-conservation-across-boundary} and using the fact that \(n^\mu\) is timelike we get:

\begin{equation}\label{eq:rankine-hugoniot-3}
    m^2 = \frac{p_+ - p_-}{\tau_+ - \tau_-}
\end{equation}

equations \eqref{eq:rankine-hugoniot-1}, \eqref{eq:rankine-hugoniot-2} and \eqref{eq:rankine-hugoniot-3} are the Rankine-Hugoniot equations.

\begin{greenbox}
  TODO: show that the nonrelativistic limit of these is

  \begin{subequations}
  \begin{align}
    \rho_0 u^i n_i &= \const  \\
    \rho_0 u^i u_i + p &= \const  \\
    h + u^2 /2 &= \const
  \end{align}
  \end{subequations}
\end{greenbox}

\subsection{Perfect fluids' spherical accretion}

We work with the Schwarzschild metric \eqref{eq:schwartzshild-line-element}; we treat a fluid with 4-velocity $u^\mu$ in spherical coordinates, since the problem we are looking at is stationary and spherically symmetric the velocity is:

\begin{equation}
    u^\mu = \begin{pmatrix}
        \gamma^2 / y\\
        yv\\
        0\\
        0
    \end{pmatrix}
\end{equation}

where we define the Lorentz factor as usual, $\gamma = \qty(1-v^2)^{-1/2}$, and $y=\gamma \sqrt{1+2\Phi}$ is the ``energy-at-infinity per unit rest mass'' (see \cite[equation 3]{ThorneFLammmangZytkow:1981feb}).

The conservation of mass holds: if $\rho_0$ is the rest mass density of the fluid, we must have $\nabla_\mu \qty(\rho_0 u^\mu) =0$. This, using the formula for covariant divergence \eqref{eq:covariant-divergence}, yields:

\begin{equation} \label{eq:integral-continuity}
    \dv{}{r} \qty(\rho_0 yvr^2) = 0
\end{equation}

\(\dv*{Q}{r} = 0\) is equivalent to \(\dv*{\log Q }{\log r} = 0 \), therefore we can recast \eqref{eq:integral-continuity} into

\begin{equation} \label{eq:differential-continuity}
  \dv{\log \rho_0}{\log r} +
  \dv{\log yv}{\log r} + 2 = 0
\end{equation}

In the newtonian limit both $\gamma$ and $y$ approach 1; also, the infalling mass rate $\Dot{M}$ at a certain radius is $\rho_0 (r) v(r) 4\pi r^2$. Then, by continuity to the newtonian limit, the quantity which is constant wrt the radius must be $\Dot{M} / (4\pi)$: therefore

\begin{equation}
  \dot{M} = 4 \pi\rho_0 yvr^2
\end{equation}

We also have the Euler equation \eqref{eq:relativistic-euler}.

The only interesting component of this is the radial one, so we need to calculate \(a^1 = u^\mu \nabla_\mu u^1 = \dv*{u^1}{\tau} + \Gamma^1_{\mu \nu} u^\mu u^\nu \). To do this we will need the radial Schwarzschild Christoffel coefficients:

\begin{equation}
  \Gamma^1_{\mu \nu} = \left[\begin{matrix}\frac{M \left(- 2 M + r\right)}{r^{3}} & 0 & 0 & 0\\0 & \frac{M}{r \left(2 M - r\right)} & 0 & 0\\0 & 0 & 2 M - r & 0\\0 & 0 & 0 & \left(2 M - r\right) \sin^{2}{\left(\theta \right)}\end{matrix}\right]
\end{equation}

while the proper-time derivative is \(\dv*{}{\tau} = u^\mu \partial_\mu = yv\partial_1\).
Plugging in the expression for the only relevant component of \(h^{\mu\nu}\), \(h^{11} = g^{11} + u^1 u^1 = (1 + 2 \Phi) (1 + v^2 \gamma^2) = y^2\)
we get, after lengthy computation,

\begin{equation}
  a^1 = y^2 \qty(\gamma^2 v \dv{v}{r} + \frac{M}{(1+ 2 \Phi) r^2})
\end{equation}

Substituting this into the (radial component of the) Euler equation \eqref{eq:relativistic-euler} we get

\begin{subequations}
\begin{align}
  (p + \rho) y^2 \qty(\gamma^2 v \dv{v}{r} + \frac{M}{(1+ 2 \Phi) r^2}) &= - h^{1 1} \partial_1 p = - y^2 \partial_1 p \\
   \gamma^2 v \dv{v}{r} + \frac{M}{(1+ 2 \Phi) r^2} + \frac{1}{p + \rho} \dv{p}{r} &= 0
  \label{eq:ideal-euler}
\end{align}
\end{subequations}

\begin{greenbox}
  In equations 3.12.7, 8 in \cite{Nobili:2000} there is most definitely a sign error: the term proportional to \(M/r^2\) should be positive.
\end{greenbox}

We also have the equation for the variation of the total internal energy, which holds for ideal fluids at constant entropy:

\begin{equation} \label{eq:enthalpy-definition}
    \dv{\rho}{\tau} = \frac{p+\rho}{\rho_0} \dv{\rho_0}{\tau}
    \qquad
    \text{or}
    \qquad
    \pdv{\rho}{\rho_0} = \frac{p + \rho}{\rho_0} \defeq h
\end{equation}

(where we have defined $h$, the specific enthalpy).
From these we can show that

\begin{claim}
The quantity $\gamma h \sqrt{1+2\Phi} = yh$ , is a constant of motion.
\end{claim}

\begin{proof}
First of all, by direct computation it can be shown that

\begin{equation} \label{eq:log-y-conservation}
  \gamma^2 v \dv{v}{r} + \frac{M}{(1+ 2 \Phi) r^2} = \dv{\log y }{r}
\end{equation}

Then, following \textcite[section 6.3]{Gourgoulhon:2006bn} we find that \(\dd{p} = \rho_0 \dd{h}\) in the isentropic case, therefore

\begin{equation} \label{eq:log-h-conservation}
  \frac{1}{\rho + p} \dv{p}{r}  =  \dv{\log h}{r}
\end{equation}

we can substitute the results in \eqref{eq:log-y-conservation} and \eqref{eq:log-h-conservation} into \eqref{eq:ideal-euler}:

\begin{equation}
  \dv{\log h}{r} + \dv{\log y }{r} = \dv{\log (hy) }{r} = 0
\end{equation}

\end{proof}

In the nonrelativistic, weak-field limit this  becomes the classical conservation of density of energy:

\begin{equation}
    \gamma h \sqrt{1+2\Phi} \approx \frac{p}{\rho_0} + \frac{v^2}{2} - \frac{M}{r} + \epsilon = \const
\end{equation}

Also, we can rewrite the last term of the Euler equation \eqref{eq:ideal-euler} using \eqref{eq:enthalpy-definition}  as:

\begin{equation}
  \frac{\partial_1 p}{p + \rho} =
  \frac{\rho_0}{\rho_0} \frac{\partial_1 p}{p + \rho} =
  \frac{1}{\rho_0} \pdv{p}{\rho} \pdv{\rho_0}{\rho}
  \pdv{\rho}{p}   \partial_1 p =
  \frac{v_s^2}{\rho_0} \partial_1\rho_0
\end{equation}

where we define the speed of sound \(v_s^2 = (\pdv*{p}{\rho})_s\) (the index \(s\) means the derivative is to be taken at constant entropy, and for an adiabatic process).

This can be rewritten as

\begin{equation}
  \dd{p} = \pdv{p}{\rho} \pdv{\rho}{\rho_0} \dd{\rho_0} = v_s^2 h \dd{\rho_0}
\end{equation}

\begin{greenbox}
  Somehow in \cite[page 175]{Nobili:2000} this becomes:

  \begin{equation}
    \dv{\log p }{\log r} = \rho_0 v_s^2 \dv{\log \rho_0}{\log r }
  \end{equation}

  but it seems to me it should be

  \begin{equation}
  \dv{\log p }{\log r} = \frac{p+\rho}{p} v_s^2 \dv{\log \rho_0}{\log r }
  \end{equation}
\end{greenbox}

Coming back to \eqref{eq:relativistic-euler},  we get:

\begin{equation}
  \gamma^2 v \dv{v}{r} + \frac{M}{(1+ 2 \Phi) r^2}
  + \frac{v_s^2 }{\rho_0}\dv{\rho_0}{r} = 0
\end{equation}

We can replace every occcurence of \((\partial x) / x \) with \(\partial \log x \):

\begin{subequations}
\begin{align}
  \gamma^2 v^2 \dv{\log v}{r} + \frac{M}{(1+ 2 \Phi) r^2} + v_s^2 \dv{\log \rho_0}{r}  &= 0 \\
  \frac{\gamma^2 - 1}{r}  \dv{\log v}{\log r} + \frac{M}{(y^2/\gamma^2) r^2}
  + \frac{v_s^2}{r} \dv{\log \rho_0}{\log r}  &= 0  \\
  % \frac{1 - (1-v^2)}{1}  \dv{\log v}{\log r} + \frac{M}{(y^2) r}
  % + (1-v^2)\frac{v_s^2}{1} \dv{\log \rho_0}{\log r}  &= 0 \\
  v^2  \dv{\log v}{\log r} + \frac{M}{y^2 r}
  + (1-v^2) v_s^2 \dv{\log \rho_0}{\log r}  &= 0
\end{align}
\end{subequations}

\begin{greenbox}
  Are the eqrefs at page 174 in \cite{Nobili:2000} wrong? Equations 2.7.3, 5 seem not to be related at all...

  Somehow this equation comes up:

  \begin{equation}
    \dv{\log \rho_0 }{r} + \gamma^2 v^2 \dv{\log v }{r}  + \frac{2 v^2}{r} + \frac{M}{(1+2 \Phi) r^2} = 0
  \end{equation}

  and we wind up with the system

  \begin{subequations}
  \begin{align}
    (v^2-v_s^2) \dv{\log \rho_0}{\log r} &= - 2v^2 + \frac{M}{y^2 r}  \\
    (v^2-v_s^2) \dv{\log yv }{\log r} &=  2v_s^2 - \frac{M}{y^2 r}  \\
    \dv{\log p }{\log r} &= \rho_0 v_s^2 \dv{\log \rho_0}{\log r }
  \end{align}
  \end{subequations}

  but I do not see at all how to get to it, it does not seem to work!
  If we add the first two equations we get back \eqref{eq:differential-continuity} which is good but that is the only part which seems to make sense.
\end{greenbox}

\begin{greenbox}
  Typo in \cite[page 175]{Nobili:2000}: shouldn't it be ``il movimento avviene nella direzione opposta (\(v>0, \dot{M}>0)\))''?
\end{greenbox}

\begin{greenbox}
  To add: commentary on the equation system, its boundary conditions \(\rho_\infty\) and \(T_\infty\), and the fact that to make the system regular at \(v = v_s\) the problem becomes uniquely determined.
\end{greenbox}

\subsection{Radiative processes in spherical accretion}

\begin{greenbox}
  Question posed in \cite[Introduction]{ThorneFLammmangZytkow:1981feb}: will a black hole inside a star eat it whole in a free-fall time-scale (around 1 year) or on an Eddington-limited time-scale (around \(10^8\) years)?
  Is this known now?
\end{greenbox}

\paragraph{Eddington luminosity}

It is the characteristic luminosity at which the radiation pressure from the photons moving outward equals the gravitational specific force on the infalling matter.

The gravitational force, in the newtonian  limit, is

\begin{equation}
  F_{\text{grav}} = \frac{GMm}{r^2 }
\end{equation}

The radiation pressure can be given in terms of the luminosity \(L\) (reinserting the units of \(c\) for this) as

\begin{equation}
  P_{\text{rad}} = \frac{L}{c 4 \pi r^2}
\end{equation}

then, the radiative force is given by \(F_{\text{rad}} =  P_{\text{rad}} \kappa m\), where \(m\) is the mass of the test object and \(\kappa\) is the opacity: the per-unit-mass cross-section. We usually assume \(\kappa = \sigma_T/m_p\), that is, that the interaction between radiation and matter is all due to Thompson scattering and the matter is only composed of hydrogen atoms.

Equating the forces, we get our result:

\begin{equation}
 \frac{L_{\text{Edd}}}{M} = \frac{4 \pi c G}{\kappa}
\end{equation}

In the \(\kappa = \sigma_T / m_p \approx \SI{0.04}{\metre^2 / kg}\) case, we get \(L_{\text{Edd}} / M\) to be around \SI{6.32}{\watt\per\kilo\gram} (constants' values from \cite[]{NISTReccomendedConstants:2018}).
If we express this in units of \(L_{\odot} / M_{\odot} \approx \SI{1.93E-04}{\watt\per\kilo\gram}\) \cite[]{SunFactSheet:2018} we get \(L_{\text{Edd}} / M \approx  \num{3.27E+04} L_{\odot} / M_{\odot}\):
the amount of radiation emitted by the Sun is much less than the Eddington limit.

It is, of course, important to note that this is a limit found with many approximations: nonrelativistic gravity, spherical symmetry, only Thompson scattering, only hydrogen.

\paragraph{Cooling function}

From \cite{NobiliTurollaZampieri:1991dec}.

The cooling function \(\Lambda (T)\) is defined by the following relation, which describes the variation in the energy density by radiative processes:

\begin{equation}
    \dv{U}{t} = n^2_b \qty(\Gamma(T) - \Lambda (T))
\end{equation}

where \(U\) is the energy density (measured in \si{\erg\per\cubic\centi\metre}), \(n_b\) is the baryon density (measured in \si{\per\cubic\centi\metre}), while \(\Gamma\) and \(\Lambda\) are the heating and cooling functions, both measured in \si{\erg\cubic\centi\metre\per\second}, see \cite[equation 1]{GnedinHollon:2012}.

The cooling function of the infalling gas is

\begin{figure}
    \centering
    \includegraphics[width=\textwidth]{figures/cooling_function.pdf}
    \caption{Cooling function graph.}
    \label{fig:cooling-function}
\end{figure}

\begin{equation}
    \begin{split}
    \Lambda (T) &= \left(
    \qty(
    \num{1.42e-27}T^{1/2} \qty(
    1 + \num{4.4e-10}T
    ) + \num{6.0e-22}T^{-1/2}
    )^{-1} \right. \\
    & \quad \left. + \num{e25} \qty(\frac{T}{\SI{1.5849e4}{K}})^{-12}
    \right)^{-1} \si{\erg\cubic\centi\metre\per\second}
    \end{split}
\end{equation}

The version of this equation in \textcite[equation 10]{StellingwerfBuff:1982} is similar: the first constant is \(\SI{2.4e-27}{} \) instead of \(\SI{1.42e-27}{} \), and the factor \(\qty(1 + \SI{4.4e-10}{}T)\) is just \(1\).

\subsection{Thorne's PSTF moment formalism}

Following \cite{Thorne:1981feb}.

Given any tensor \(A^{\mu_1 \dots \mu_k}\) we can use the tensor \(h^{\mu\nu}\) to project it into the space-like subspace defined by the velocity \(u^\mu\):

\begin{equation}
    A^{\mu_1 \dots \mu_k} \rightarrow \qty(A^{\mu_1 \dots \mu_k})^P
    = \qty(\prod_i h^{\mu_i}_{\nu_i}) A^{\nu_1 \dots \nu_k}
\end{equation}

Then, we can take the symmetric part of any (?) tensor as outlined in \Nameref{sec:notational-preface}:

\begin{equation}
    A^{\mu_1 \dots \mu_k} \rightarrow \qty(A^{\mu_1 \dots \mu_k})^S
    = A^{(\mu_1 \dots \mu_k)}
\end{equation}

We can select the trace-free part of a projected, symmetric tensor by

\begin{equation}
    A^{\mu_1 \dots \mu_k} \rightarrow \qty(A^{\mu_1 \dots \mu_k})^{TF}
    = \sum _{i=0}   ^{\lfloor k/2 \rfloor}
    (-1)^i \frac{k! (2k-2i-1)!!}{(k-2i)! (2k-1)!! (2i)!!}
    h^{(\alpha_1 \alpha_2} \dots h^{\alpha_{2i-1} \alpha_{2i}}
    A^{\alpha_{2i+1} \dots \alpha_k) \beta_1 \dots \beta_i}\,_{\beta_1 \dots \beta_i}
\end{equation}

To see what this is doing, let us consider its action on a rank-two projected tensor: it is just the subtraction of its trace,

\begin{equation}
    A^{\mu\nu} \rightarrow A^{\mu\nu} - \frac{1}{3} h^{\mu\nu} A^{\rho}_\rho
\end{equation}

Now, let us consider all the unit vectors \(n^\mu\) in the space normal to the velocity, which have \(n_\mu u^\mu = 0\) and \(n^\mu n_\mu = 1\). They span a three-dimensional sphere.

If we have a function \(F\colon S^2 \rightarrow \mathbb R\), we can decompose it into harmonics as such:

\begin{equation}
    F(n) = \sum _{k=0}   ^{\infty}
    \mathscr F_{\alpha_1 \dots \alpha_k} \prod_{i=0}^k n^{\alpha_i}
\end{equation}

Where the PTSF moments \(\mathscr F_{\alpha_1 \dots \alpha_k}\) can be computed as

\begin{equation}
    \mathscr F_{\alpha_1 \dots \alpha_k} =
    \frac{(2k+1)!!}{4 \pi k!} \qty(\int F \prod_{i=0}^k n^{\alpha_i}  \dd{\Omega}  )^{TF}
\end{equation}

In particular, the function we will apply this to is the distribution of EM radiation around the BH. So, let us consider a photon, whose trajectory in spacetime is parametrized as \(\gamma(\xi)\), with a choice of \(\xi\) such that the photon's momentum is

\begin{equation}
    p = \dv{}{\xi}
\end{equation}

Now, our observer has a timelike velocity \(u^\mu\). We can find a spacelike vector \(n^\mu\) corresponding to the space-like part of the movement of the photon, or

\begin{equation}
    p^\mu = (- u^\nu p_\nu) (u^\mu + n^\mu)
\end{equation}

It must hold that \(u^\mu u_\mu = -1 \) while \(n^\mu n_\mu = +1 \) in order for \(p^\mu\) to be null-like.
Now, we define a parameter \(l\) which corresponds to the space distance the photon moved through in this frame (this is \emph{not} covariant!)

\begin{equation}
    l = \int  (-u^\nu p_\nu) \dd{\xi}
\end{equation}

now, \(\dv*{}{l} \) is parallel to \(p\) but it has different length, in fact since \(\dv*{l}{\xi} = (-u^\nu p_\nu) \) it is \(\dv*{}{l} = u + n \).

It holds \cite[eq. 2.17]{Thorne:1981feb}, (with the notation from \eqref{eq:covariant-acceleration-decomposition}),  that

\begin{equation}
  \dv{\nu}{l}  = (u^\mu + n^\mu) \nabla_\mu (-p^\nu u_\nu) = - \nu \qty(n_\mu a^\mu  + \frac{\theta}{3} + n_\mu n_\nu \sigma^{\mu\nu})
\end{equation}

We want to quantify the number density of photons in relation to their momentum. We assume the radiation in unpolarized, therefore for each unit \(h^3\) cell in phase space there can be 2 photons: so we denote the distribution function of the photons as \(2N (x^\mu, p^\mu)\).

It is known that the volume element \(\dd{V}_p = \dd[3]{p} / p^0 \) is Lorentz invariant (see \cite[box 22.5]{MisnerThorneWheeler:1973}).
We can write this using the photons' frequency \(\nu = - p^\mu u_\mu / h\) as \(\dd{V}_p = \nu \dd[]{\Omega} \dd{\nu} \).

Let us define the \emph{specific radiative intensity} as

\begin{equation}
  I_\nu = \frac{\delta E}{\delta A \delta  t \delta \nu \delta \Omega}
  = \frac{h \nu \delta N}{\delta A \delta  t \delta \nu \delta \Omega}
\end{equation}

where \(\delta A\) denotes an infinitesimal area the photons are coming through, \(\delta t\) an infinitesimal time, \(\delta \nu\) an infinitesimal photon frequency, \(\delta \Omega\) an infinitesimal solid angle.

Then, \cite[figure 22.2]{MisnerThorneWheeler:1973} the number density of photons in phase space is

\begin{equation}
  2N(x^\mu, p^\mu) = \frac{\delta N}{V_x V_p} =  \frac{\delta N}{h^3 \nu^2\delta A \delta  t \delta \nu \delta \Omega} = \frac{1}{h^4 \nu^3} I_\nu
\end{equation}

therefore \(I_\nu = 2 N \nu^3 h^4\).

\begin{greenbox}
  This is from \cite[figure 22.2]{MisnerThorneWheeler:1973}, but it seems to conflict with \cite[equation 6.2]{Thorne:1981feb}!
\end{greenbox}

Now, we want to describe the variation of the occupation number \(N\) with respect to the photons' trajectories' parameter \(l\). We encapsulate all possible effects into a source term \(\mathfrak S\):

\begin{equation}
    \mathfrak S \defeq \dv{}{l} 2N(x^\mu, p^\mu) =
    2 \qty(\pdv{N}{x^\mu} \dv{x^\mu}{l} + \pdv{N}{p^i} \dv{p^i}{l}  )
\end{equation}

since the occupation number can be thought of as just a function of the spatial components of the momentum.

\begin{greenbox}
  Why so? Surely \(N = N(\nu)\) also!
\end{greenbox}

Since \(\dv*{}{l} = (n^\mu + u^\mu) \partial_\mu\) and the covariant derivative of \(p^j\) is zero, we can compute

\begin{equation}
  \dv{p^j}{l} = (n^\mu + u^\mu ) \nabla_\mu p^j - \Gamma ^j _{\alpha \beta} p^\alpha (u^\beta + n^\beta)
  = - \Gamma ^j _{\alpha \beta} p^\alpha (u^\beta + n^\beta)
\end{equation}

where the covariant derivative term vanishes since the photon's trajectory is a geodisic.

\paragraph{Moments' definitions}

In units where \(c=h=1\),

\begin{subequations}
\begin{align}
   M _\nu ^{A_k}
   &\defeq \int 2N \frac{\delta (\nu - (-p^\nu u_\nu))}{\nu^{k-2}} \prod_i^k p^{\alpha_i} \dd{V_p} \\
   &= \int \qty(2N \nu^3) \frac{1}{\nu} \delta (\nu +p^\nu u_\nu) \prod_i^k \qty(\frac{p^{\alpha_i}}{\nu}) \qty(\nu \dd[]{\Omega} \dd[]{\nu})  \\
   &= \int  I_\nu \prod_i^k \qty(n^{\alpha_i} + u^{\alpha_i}) \dd[]{\Omega} \label{eq:simplified-moment-definition}
\end{align}
\end{subequations}

This is a general procedure we can use to associate a function \(f\) (in this case we started with \(2N\)) with \(\nu^3\) times the integral \eqref{eq:simplified-moment-definition} (where one might substitute \(I _\nu\) with the function \(f\)).
We need it for the source moments:

\begin{equation}
   S_\nu ^{A_k} = \nu^3 \int S \mathfrak S \prod_i^k (n^{\alpha_i} + u^{\alpha_i}) \dd[]{\Omega}
\end{equation}

\paragraph{Redshift-adapted version}

\textcite[]{Thorne:1981feb} also defines a redshift-adapted version of the moments' definition: if \(R\) is a universal redshift functions, such that \(R (p^\nu u_\nu)\) is conserved along every photon geodesic \(p^\mu \nabla_\mu p^\nu = 0\), that is, \(R\) allows us to calculate the redshift between any two points \(A\), \(B\) which are connected by a geodesic as \(\nu_A / \nu_B = R_B / R_A\).

Then, we define \( M_f ^{A_k} =  M_{\nu} ^{A_k} / R\)

\paragraph{Frequency-integrated version}

The definition is:

\begin{equation}
   M ^{A_k} = \int   M^{A_k} _\nu \dd{\nu}
\end{equation}

and the same is applied to the source moments \(S_\nu^{A_k} \rightarrow S^{A_k}\).

Since this includes the radiation intensity from all frequencies, we have direct interpretations for the first moments:

\begin{subequations}
\begin{align}
   M &= \int  I_\nu \dd{\Omega} \dd{\nu}   & \text{energy density of radiation}  \\
   M^\alpha &= \int I_\nu (n^\alpha + u^\alpha)\dd{\Omega} \dd{\nu}   & (M^0, M^i) = \text{(energy density of radiation, energy flux)}  \\
   M^{\alpha\beta} &= \int I (n^\alpha + u^\alpha)(n^\beta + u^\beta)\dd{\Omega} \dd{\nu}   & \text{stress-energy tensor of radiation}
\end{align}
\end{subequations}

\paragraph{The moment equations}

These can be derived from the transport equation, see \cite[3.14]{Thorne:1981feb}. I present them only in the grey (frequency-integrated) case:

\begin{equation} \label{eq:grey-moment-equations}
  \nabla_\beta M^{A_k \beta} - (k-1) M^{A_k \beta \gamma} (\nabla_ \gamma u_\beta)= S^{A_k}
\end{equation}

Also, the moments (\(M^{A_k}\), but also \(M^{A_k}_\nu\) and \(M^{A_k}_f\)) satisfy the following:

\begin{subequations}
\begin{align}
  \nabla_\beta M^{A_k \beta} &= 0 \\
  u_\beta M^{A_k \beta} &= -M^{A_k} \\
  h_{\beta \gamma} M^{A_k \beta \gamma} &= M^{A_k}
\end{align}
\end{subequations}

So, the \(k\)-th moment contains all the information about the \(l\)-th moments with \(l\leq k\); also, to get lower-order moments we take partial traces onto space- and time-like subspaces: therefore the unique information to the \(k\)-th moment, which is not redundantly expressed in lower-order moments, is in its PFTF part:

\begin{equation}
  \mathscr M ^{A_k} = \qty(M^{A_k}) ^{PSTF}
\end{equation}

The same can be applied to \(M^{A_k}_\nu\) and \(M^{A_k}_f\) and to the moment equations \eqref{eq:grey-moment-equations}. Since we are taking the projection onto the space-like subspaces, we can simplify the expression of the PSTF moments: all the terms which contain at least a four-velocity vanish, therefore:

\begin{equation}
  \mathscr M^{A_k} = \qty(\int I_\nu \prod_i n^{\alpha_i} \dd{\Omega})^{TF}
\end{equation}

\end{document}
