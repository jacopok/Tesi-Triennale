\documentclass[main.tex]{subfiles}
\begin{document}

\subsection{Relativity}

\paragraph{Tensor calculus}

We can define the Dalambertian operator $\square = \nabla_\mu \nabla^\mu = \nabla_\mu \partial^\mu$, which can only act on scalars, and it does so like:

\begin{equation}
    \square A = \nabla_\mu (\partial^\mu A) =  \frac{1}{\sqrt{-g}}\partial_\mu \qty(\sqrt{-g}\partial^\mu A)
\end{equation}

If we differentiate and antisymmetrize (so, take the rotor of) an antisymmetric tensor $F_{[\mu \nu]}$, the Christoffel symbols cancel:

\begin{equation}
    \nabla_{[\mu} F_{\nu\rho]} = \partial_{[\mu} F_{\nu\rho]}
\end{equation}

\paragraph{Lie derivative}

Following \cite[section 6]{Taub:1978}. The Lie derivative of a generic tensor \(T^{\mu_1 \dots \mu_n} _{\nu_1 \dots \nu_m}\) along a vector \(\xi^\mu\) is defined as:

\begin{equation}
    \Lie_{\xi} T^{\mu_1 \dots \mu_n} _{\nu_1 \dots \nu_m} \defeq
    \xi^\rho \nabla_\rho T^{\mu_1 \dots \mu_n} _{\nu_1 \dots \nu_m}
    + \sum_{i=1}^{m} T^{\mu_1 \dots \mu_n} _{\nu_1 \dots \nu_{i-1} \rho \nu_{i+1} \dots \nu_m} \nabla_{\nu_i} \xi^\rho
    - \sum_{i=1}^{n} T^{\mu_1 \dots \nu_{i-1} \rho \nu_{i+1} \dots \mu_n} _{\nu_1 \dots  \nu_m} \nabla_{\rho} \xi^{\mu_i}
\end{equation}

Some special cases are: a scalar \(\Lie _\xi f = \xi^\rho \partial_\rho f \), a vector \(\Lie _\xi u^\mu = \xi^\rho \nabla_\rho u^\mu -u^\rho \nabla_\rho \xi^\mu\), and  an antisymmetric covariant two-tensor \(\omega_{\mu\nu} = \omega_{[\mu \nu]}\) which represents a closed form \(\nabla_{[\mu} \omega_{\nu \rho]} = 0\) we have \(\Lie _\xi \omega_{\mu\nu} =2\nabla_{[\nu} \qty( \omega_{\mu] \rho} \xi^\rho )\)

\paragraph{Pressure inequalities}

It holds \cite[]{Taub:1948} that:

\begin{equation}
0
\leq 3p
\leq \frac{3}{2}p + \sqrt{\qty(\frac{3}{2}p)^2 + \rho_0^2}
\leq \rho = \rho_0 (1 + \varepsilon)
\leq \rho_0 + 3p
\end{equation}

\paragraph{Vorticity vector}

We can describe the rotation with a ``vorticity vector'':

\begin{equation}
    \omega^\mu = \frac{1}{2 \sqrt{-g}} \varepsilon^{\mu\nu\sigma\tau} u_\nu \partial_\tau u_{\sigma} = \frac{1}{2} \eta^{\mu\nu\sigma\tau} u_\nu \partial_\tau u_{\sigma}
\end{equation}

where we define the fully antisymmetric, covariant tensor \(\eta^{\mu\nu\sigma\tau} = \varepsilon^{\mu\nu\sigma\tau} / \sqrt{-g} \). With its indices lowered, it is
\(\eta_{\mu\nu\sigma\tau} = - \varepsilon_{\mu\nu\sigma\tau} \sqrt{-g} \).

Note, from \cite[pages 51--52]{Carroll:1997ar}: this is the volume form of the manifold, and it is defined this way since \(g\) is a \emph{tensor density} of weight $-2$, while the bare Levi-Civita symbol is a density of weight $+1$.

\begin{greenbox}
  The signs in \cite[]{Taub:1978} and \cite[]{Carroll:1997ar} seem to disagree though!
\end{greenbox}

By the antisymmetry in the definition we can immediately see that \(\omega^\mu u_\mu=0\). It holds that \(\omega^\mu \equiv 0 \iff u_\mu = \rho \partial_\mu f \) (locally!) for scalar \(\rho\), \(f\), since this is equivalent to \(u_\mu\) being a closed form.

\(\omega_\mu\) and \(\omega_{\mu \nu}\) are ``dual'' in the sense that

\begin{equation}
    \omega _{\sigma\tau} = u^\mu \omega^\nu \eta_{\mu \nu \sigma \tau}
\end{equation}

and some other useful identities can be found in equations 7.5 through 7.7 of \cite[]{Taub:1978}.

\paragraph{The General Relativity postulates}

We can formalize the mathematical postulates which determine our theory of relativity (see \cite[section 3.2]{HawkingEllis:1973}):

\begin{enumerate}
    \item The laws of physics are tensor laws in a 4-dimensional manifold with metric signature \(+2\);
    \item there exists a stress-energy tensor \(T^{\mu\nu}\) which: transforms as a \((2,0)\) tensor, is symmetric, is conserved (\(\nabla_\mu T^{\mu\nu} = 0\)), is nonzero only in the regions which contain matter or radiation;
    \item Local causality: every physical curve (which represents an object's path) must admit a reparametrization with tangent vector \(u^\mu\) such that \(u^\mu u_\mu = -1\);
    \item global causality: there are no closed timelike curves;
    \item the Einstein Field Equations \eqref{eq:EFE} hold.
\end{enumerate}

\subsection{Fluid dynamics}

\paragraph{Wave velocity}

Following \cite[section 5]{Taub:1978}.

An equation in the form \(\varphi (x^\mu) = 0\) defines a 3D surface \(\Sigma\); its constant-\(x^0\) slices are 2D surfaces. We can decompose \(\partial_\mu \varphi\) in a component proportional to the velocity, \(a u_\mu\), and one which is orthogonal to it, \(W_\mu = h^\sigma_\mu \partial_\sigma \varphi\). Then \(W_\mu W^\mu = h^\sigma_\mu  h_\beta^\mu \partial_\sigma \varphi \partial^\beta \varphi = h^{\sigma\beta} \partial_\sigma \varphi \partial_\beta \varphi\)
by the idempotency of the projector \(h\).
Thus we can see that \(W^2 \geq 0\), therefore it is a spacelike vector. We can define its corresponding unit vector: \(W^\mu = t^\mu \sqrt{W^\nu W_\nu}\).

We can choose a velocity \(v\) such that \(k^\mu = u^\mu - v t^\mu\) is parallel to \(\Sigma\), or \(k^\mu \partial_\mu \varphi = 0\). If this condition is satisfied, then \(v\) is the wave velocity of \(\Sigma\) as measured by an observer with velocity \(u^\mu\).

\begin{figure}[ht]
  \centering
  \incfig{figures/taub_wave_velocity}
  \caption{Wave velocity diagram}
  \label{fig:taub_wave_velocity}
\end{figure}

Now, we can multiply the equation \(k^\mu = u^\mu - v t^\mu\) by \(\partial_\mu \varphi\): we get \(v t^\mu \partial_\mu \varphi = u^\mu \partial_\mu \varphi\). What multiplies \(v\) in the LHS of this equation is the length of the spatial component of \(\partial_\mu \varphi\), or \(\sqrt{h^{\mu\nu} \partial_\mu \varphi \partial_\nu \varphi}\).
Using the definition of \(h ^{\mu \nu} \), we can arrive at

\begin{equation} \label{eq:lorentz-factor-wave-velocity}
    \gamma^{-2} = 1 - v^2 = \frac{g^{\mu\nu} \partial_\mu \varphi \partial_\nu \varphi}{h^{\mu\nu} \partial_\mu \varphi \partial_\nu \varphi}
\end{equation}

The denominator in \eqref{eq:lorentz-factor-wave-velocity} is positive, so we can see that \(1-v^2\) is positive iff \(\partial_\mu \varphi\) is spacelike, and \(v^2 = 1\) iff it is null.

If we are dealing with a timelike surface \(\Sigma\), or equivalently \(\partial_\mu \varphi\) is spacelike, then we define \(n^\mu \propto \partial_\mu \varphi\) such that \(n^\mu n_\mu = 1\) and we find:

\begin{equation}
    v = \frac{u^\mu \partial_\mu \varphi}{\sqrt{h^{\mu\nu} \partial_\mu \varphi \partial_\nu \varphi}}
    = \frac{u^\mu n_\mu}{\sqrt{h^{\mu\nu} n_\nu n_\mu}}
    = \frac{u^\mu n_\mu}{\sqrt{1 + (u^\mu n_\mu)^2}}
\end{equation}

From this it can be shown that \(1-v^2 = (1+(u^\mu n_\mu)^2)^{-1}\), and then \(v \gamma = u^\mu n_\mu\).

\paragraph{Stationarity}

Will skip most of this section. A spacetime is stationary if it admits a timelike Killing vector field? \textcite[]{Ehlers:1971} proved that from a hypothesis of stationarity we get \(\nabla_\mu S^\mu = 0\).


\paragraph{Barotropic flows}

The definition of a barotropic flow is: a flow for which the density is just a function of temperature, \(\rho = \rho (p)\).

\begin{greenbox}
  What is \(d\) in \cite[equation 11.10]{Taub:1978}? Is it a typo? I do not get that equation. Also, I quote: ``Equations 11.3 and 11.5 do not hold. However, it is a consequence of equations 11.3 and...''.

  Insert here: definition of the variable \(s\).
\end{greenbox}

The quantity \(s\) obeys the conservation law \(\nabla_\mu (s u^\mu) = 0\). We define a quantity \(Q = \log ((\rho + p)/s )\), which for \(s=\rho_0\) is the log-enthalpy. Then, the Euler equation \eqref{eq:relativistic-euler} can be substituted by

\begin{equation}
  a^\sigma = - h^{\sigma \mu} \partial_\mu Q
\end{equation}

or, more explicitly:

\begin{equation}
  (\rho + p) a^\mu = -s h^{\mu\nu} \partial_\nu \qty(\frac{\rho + p}{s})
\end{equation}

\begin{greenbox}
  Unproven in \cite[]{Taub:1978}!? where does it come from?
\end{greenbox}

We now define a ``current'' of \(\exp(Q)\): \(V^\mu = \exp(Q) u_\mu \).

Then we define \(\Omega_{\mu\nu} = 2 \nabla_{(\mu} V_{\nu)}\): this satisfies \(\Omega_{\mu\nu} u^\nu = - T \partial_\mu S\) in general, and \(\Omega_{\mu\nu} u^\nu = 0\) in the isentropic case.

\begin{greenbox}
  Right? Underneath equation \cite[11.14]{Taub:1978} what is meant is ``in the isentropic'' case, I think, since then \(\partial_\mu S = 0\)...

  Also, what does the distinction between the two definitions in 11.14 and 11.13 mean? Is 11.13 not a subcase of 11.14?
\end{greenbox}

Then, we define

\begin{equation}
  v^\mu = \frac{1}{2} \eta^{\mu \nu \rho \sigma} u_\nu \Omega_{\rho \sigma}
\end{equation}

which satisfies

\begin{equation}
  3 u_{[\alpha} \Omega_{\beta \gamma]} = v^{\mu} \eta_{\mu \alpha \beta \gamma}
\end{equation}

\begin{greenbox}
  The normalization does not seem right: \(\eta^{\mu\nu\rho\sigma} \eta_{\mu\alpha\beta\gamma} = -\delta^{\nu\rho\sigma}_{\alpha\beta\gamma}\), so when substituting in the result seems off by \(- 3!\)...
\end{greenbox}

We use this result to get an explicit formula for \(\Omega_{\alpha\beta}\) in terms of \(u^\mu\), \(v^\mu\).
This comes out by multiplying by \(u^\gamma\), and is:

\begin{equation}
  \Omega _{\alpha\beta}
  = - v^\mu u^\gamma \eta_{\mu\gamma\alpha\beta} + u_\alpha \Omega_{\beta\gamma} u^\gamma - u_\beta \Omega_{\alpha\gamma} u^\gamma
  = - v^\mu u^\gamma \eta_{\mu\gamma\alpha\beta} + 2T (u_{[\beta} \partial_{\alpha]} S )
\end{equation}

We can also rewrite the perfect-fluid stress-energy tensor as \(T^{\mu\nu} = s V^\mu u^\nu +p g^{\mu\nu}\): then, since the metric has zero covariant derivative, if we look at the projection of \(T^{\mu\nu}\) along a Killing vector field \(\nabla_{(\mu} \xi_{\nu)}=0\) we get

\begin{equation}
  \nabla_\nu \qty(\xi_\mu T^{\mu\nu}) = \nabla_\nu \qty(\xi_\mu s V^\mu u^\nu) = 0
\end{equation}

but in the (barotropic?) case \(s = \rho_0\) we can use the conservation of mass to get \(\rho_0 u^\nu \nabla_\nu (\xi_\mu V^\mu) = 0\), so the quantity \(H = \xi_\mu V^\mu\) is conserved along the worldlines of the fluid.

If we are in the coordinate system which would be the LRF for an observer with velocity \(\xi^\mu\), then the conserved quantity is \(H = V^0 \).

\paragraph{Shocks and conservation laws through boundaries}

Following \cite[section 13]{Taub:1978}.
The differential formulations of the conservation of mass and momentum will not hold at points of non-differentiability, such as through shock waves.
There, they must be replaced by an integral law.
We can reframe our conservation laws as the statements that, for any scalar \(f\) and vector \(\lambda_\mu\) we will have

\begin{subequations} \label{eq:conservation-laws-arbitrary-functions}
\begin{align}
  \nabla_\mu \qty(f \rho_0 u^\mu) &= \rho_0 u^\mu \partial_\mu f  \\
  \nabla_\mu \qty(\lambda_\nu T^{\mu\nu})&= T^{\mu\nu} \nabla_\mu \lambda_\nu
\end{align}
\end{subequations}

We can integrate these on a generic volume \(V\) using Stokes' theorem \eqref{eq:stokes-theorem}, choosing coordinates on the boundary such that the determinant of the induced metric on the submanifold is uniformly 1.

\begin{subequations} \label{eq:conservation-laws-arbitrary-functions-integral}
    \begin{align}
        \int _{\partial V} n_\mu \qty(f \rho_0 u^\mu) \dd[3]{y} &= \int_V \rho_0 u^\mu \partial_\mu f \sqrt{-g}  \dd[4]{x}  \\
        \int_{\partial V} n_\mu \qty(\lambda_\nu T^{\mu\nu}) \dd[3]{y}&= \int _V T^{\mu\nu} \nabla_\mu \lambda_\nu \sqrt{-g}  \dd[4]{x}
    \end{align}
\end{subequations}

Now, to deal with the shock we do the following: take the hypersurface at which the shock happens, enclose it in an infinitesimally thin 4D volume: then, as the volume decreases the RHSs of equations \eqref{eq:conservation-laws-arbitrary-functions-integral} go to 0, therefore the LHSs also must: we can write them as the difference of the integrands at the boundary.
Therefore, introducing the notation \([F] = F_+ - F_-\), selecting one of the two outward vectors and denoting only it as \(n_\mu\) and using the arbitrariness of \(f\), \(\lambda_\nu\) we have:

\begin{equation}
    \qty[n_\mu \rho_0 u^\mu] = n_\mu \qty[\rho_0 u^\mu] = 0
    \qquad
    \text{and}
    \qquad
    \qty[n_\mu T^{\mu\nu}] = n_\mu [T^{\mu\nu}] = 0
\end{equation}

We denote \(m = \rho_0 u^\mu n_\mu\) (at the boundary), and then the first equation is the fact that \(m\) is the same on either side:

\begin{equation} \label{eq:rankine-hugoniot-1}
    \rho_{0+} u^\mu_+ n_\mu =
    \rho_{0-} u^\mu_- n_\mu \defeq m
\end{equation}

The other, for an ideal flud, is

\begin{equation} \label{eq:momentum-conservation-across-boundary}
    m \qty(V^\mu_+ - V^\mu _-) + n^\mu \qty(p_+ - p_-) = 0
\end{equation}

with \(V^\mu = h u^\mu\). Now, consider a set of vectors \(Y^\mu\) such that \(Y^\mu n_\mu = 0\) and \(Y^\mu Y_\mu = 1\). These form a two-parameter family (\(\sim S^2\)), and
by \eqref{eq:momentum-conservation-across-boundary} must satisfy

\begin{equation}
    m \qty(V^\mu_+ - V^\mu _-) Y_\mu = 0
\end{equation}

then
\begin{itemize}
    \item either \(m = 0\): this is called a \emph{slip-stream} discontinuity, because the normal component of the velocity is zero, so the fluid is flowing \emph{along} the boundary, no matter is crossing it;
    \item  or \(V^\mu _+ Y_\mu = V^\mu _- Y_\mu\) for \emph{all} the \(Y^\mu\): this is called a \emph{shock wave}.
\end{itemize}

\begin{greenbox}
  There are \emph{three} independent \(Y^\mu\), right?
\end{greenbox}

In the shock wave case, we must write two (?) more equations to fully recover the \eqref{eq:momentum-conservation-across-boundary}. To do so, we define \(\tau = h/\rho_0\) and use the fact that \(V^2 = -h^2\).
Then, we multiply \eqref{eq:momentum-conservation-across-boundary} by \(V_{\mu}^{+}\) and \(V_{\mu}^{-}\), and use the fact that \(n_\mu V^\mu_{\pm} = m \tau_{\pm}\). We get

\begin{equation} \label{eq:rankine-hugoniot-2}
    h^2_+ - h^2_- - (p_+ - p_-)(\tau_+ - \tau_-) = 0
\end{equation}

then, by multiplying \eqref{eq:momentum-conservation-across-boundary} and using the fact that \(n^\mu\) is timelike we get:

\begin{equation}\label{eq:rankine-hugoniot-3}
    m^2 = \frac{p_+ - p_-}{\tau_+ - \tau_-}
\end{equation}

equations \eqref{eq:rankine-hugoniot-1}, \eqref{eq:rankine-hugoniot-2} and \eqref{eq:rankine-hugoniot-3} are the Rankine-Hugoniot equations.

\begin{greenbox}
  TODO: show that the nonrelativistic limit of these is

  \begin{subequations}
  \begin{align}
    \rho_0 u^i n_i &= \const  \\
    \rho_0 u^i u_i + p &= \const  \\
    h + u^2 /2 &= \const
  \end{align}
  \end{subequations}
\end{greenbox}

\subsection{Radiative processes}

\begin{greenbox}
  Question posed in \cite[Introduction]{ThorneFLammmangZytkow:1981feb}: will a black hole inside a star eat it whole in a free-fall time-scale (around 1 year) or on an Eddington-limited time-scale (around \(10^8\) years)?
  Is this known now?
\end{greenbox}

\subsection{Old adiabatic accretion}


Also, we can rewrite the last term of the Euler equation \eqref{eq:ideal-euler} using \eqref{eq:enthalpy-definition}  as:

\begin{equation}
  \frac{\partial_1 p}{p + \rho} =
  \frac{\rho_0}{\rho_0} \frac{\partial_1 p}{p + \rho} =
  \frac{1}{\rho_0} \pdv{p}{\rho} \pdv{\rho_0}{\rho}
  \pdv{\rho}{p}   \partial_1 p =
  \frac{v_s^2}{\rho_0} \partial_1\rho_0
\end{equation}
where we define the speed of sound \(v_s^2 = (\pdv*{p}{\rho})_s\) (the index \(s\) means the derivative is to be taken at constant entropy, and for an adiabatic process).

This can be rewritten as

\begin{equation}
  \dd{p} = \pdv{p}{\rho} \pdv{\rho}{\rho_0} \dd{\rho_0} = v_s^2 h \dd{\rho_0} \,.
\end{equation}

\begin{greenbox}
  Somehow in \cite[page 175]{Nobili:2000} this becomes:

  \begin{equation}
    \dv{\log p }{\log r} = \rho_0 v_s^2 \dv{\log \rho_0}{\log r }
  \end{equation}

  but it seems to me it should be

  \begin{equation}
  \dv{\log p }{\log r} = \frac{p+\rho}{p} v_s^2 \dv{\log \rho_0}{\log r }
  \end{equation}
\end{greenbox}

Coming back to the Euler equation \eqref{eq:relativistic-euler},  we get:

\begin{equation}
  \gamma^2 v \dv{v}{r} + \frac{M}{(1+ 2 \Phi) r^2}
  + \frac{v_s^2 }{\rho_0}\dv{\rho_0}{r} = 0 \,.
\end{equation}

We can replace every occcurence of \((\partial x) / x \) with \(\partial \log x \):

\begin{subequations}
\begin{align}
  \gamma^2 v^2 \dv{\log v}{r} + \frac{M}{(1+ 2 \Phi) r^2} + v_s^2 \dv{\log \rho_0}{r}  &= 0 \\
  \frac{\gamma^2 - 1}{r}  \dv{\log v}{\log r} + \frac{M}{(y^2/\gamma^2) r^2}
  + \frac{v_s^2}{r} \dv{\log \rho_0}{\log r}  &= 0  \\
  % \frac{1 - (1-v^2)}{1}  \dv{\log v}{\log r} + \frac{M}{(y^2) r}
  % + (1-v^2)\frac{v_s^2}{1} \dv{\log \rho_0}{\log r}  &= 0 \\
  v^2  \dv{\log v}{\log r} + \frac{M}{y^2 r}
  + (1-v^2) v_s^2 \dv{\log \rho_0}{\log r}  &= 0 \,.
\end{align}
\end{subequations}

\begin{greenbox}
  Are the references to equations at page 174 in \cite{Nobili:2000} wrong? Equations 2.7.3, 5 seem not to be related at all...

  Somehow this equation comes up:

  \begin{equation}
    \dv{\log \rho_0 }{r} + \gamma^2 v^2 \dv{\log v }{r}  + \frac{2 v^2}{r} + \frac{M}{(1+2 \Phi) r^2} = 0
  \end{equation}

  and we wind up with the system shown later,
  but I do not see at all how to get to it, it does not seem to work!
  If we add the first two equations we get back \eqref{eq:differential-continuity} which is good but that is the only part which seems to make sense.
\end{greenbox}

The equations can be cast into the system:

\begin{subequations}
\begin{align}
  (v^2-v_s^2) \dv{\log \rho_0}{\log r} &= - 2v^2 + \frac{M}{y^2 r}  \\
  (v^2-v_s^2) \dv{\log yv }{\log r} &=  2v_s^2 - \frac{M}{y^2 r}  \\
  \dv{\log p }{\log r} &= \rho_0 v_s^2 \dv{\log \rho_0}{\log r }
\end{align}
\end{subequations}
which can be solved numerically: we have a critical point when the velocity \(v\) reaches the speed of sound, since there the Jacobian becomes singular.

% \begin{greenbox}
%   Likely typo in \cite[page 175]{Nobili:2000}: shouldn't it be ``il movimento avviene nella direzione opposta (\(v>0, \dot{M}>0)\))''?
% \end{greenbox}


\end{document}
