\subsection{Nobili}

\paragraph{Tensor calculus}

The covariant derivative keeps account of the shifting of the basis vectors:

\begin{equation}
    \nabla_\mu A^\nu = \partial_\mu A^\nu + \Gamma^\nu _{\alpha \mu}  A^\alpha
\end{equation}

The rank-3 objects $\Gamma$ are called Christoffel symbols. They are not tensors! they depend on the choice of basis $e_\alpha$, and they satisfy $\nabla _\mu e_\alpha = \Gamma ^\nu _{\mu \alpha} e_\nu$.

If we have the metric, they can be calculated as:

\begin{equation}
    \Gamma^\mu_{\nu \rho} = \frac{1}{2} g^{\mu \alpha} \qty(
    \partial_\rho g_{\alpha \nu} +
    \partial_\nu g_{\alpha \rho} -
    \partial_\alpha g_{\nu \rho}
    ) \label{eq:christoffel-symbols-from-metric}
\end{equation}

This also tells us that they are symmetric in the lower two indices: $\Gamma ^\mu _{\nu \rho} = \Gamma ^\mu _{(\nu \rho)}$.

The divergence of a vector field $A^\mu$ can be calculated as:

\begin{equation}
    \nabla_\mu A^\mu = \frac{1}{\sqrt{-g}}\partial_\mu \qty(\sqrt{-g}A^\mu) \label{eq:covariant-divergence}
\end{equation}

where $g$ is the determinant of the metric.

We can also define the Dalambertian operator $\square = \nabla_\mu \nabla^\mu = \nabla_\mu \partial^\mu$, which can only act on scalars, and it does so like:

\begin{equation}
    \square A = \nabla_\mu (\partial^\mu A) =  \frac{1}{\sqrt{-g}}\partial_\mu \qty(\sqrt{-g}\partial^\mu A)
\end{equation}

If we differentiate and antisymmetryze (so, take the rotor of) an antisymmetric tensor $F_{[\mu \nu]}$, the Christoffel symbols cancel:

\begin{equation}
    \nabla_{[\mu} F_{\nu\rho]} = \partial_{[\mu} F_{\nu\rho]}
\end{equation}

The derivative with respect to proper time is $\dv{}{\tau} = u^\mu \partial_\mu$.

Covariant acceleration is defined as:

\begin{equation}
    a^\nu = u^\mu \nabla_\mu u^\nu
\end{equation}

\paragraph{Geodesics}

If we have a path $x^\mu(\lambda)$, we would like to see if it is a geodesic, that is, if it is stationary with respect to path length. To do this we can stationarize the action corresponding to the lagrangian $\Lagr (x, \Dot{x}) = g_{\mu\nu} \Dot{x}^\mu \Dot{x}^\nu$ (where we use $\Dot{x} = \dv*{x}{\lambda}$). The Lagrange equations then are:

\begin{equation}
    \Ddot{x}^\mu + \Gamma^\mu_{\nu\rho} \Dot{x}^\nu \Dot{x}^\rho = 0
\end{equation}

Where $\Gamma$ are the Christoffel symbols, which can be calculated by differentiating the metric, as shown in \eqref{eq:christoffel-symbols-from-metric}. $\Lagr$ is an integral of these Lagrange equations.

If the parameter $\lambda$ is taken to be the proper time $s$, then the equation is

\begin{equation}
    \dv{u^\mu}{s} + \Gamma^\mu _{\nu\rho} u^\nu u^\rho = 0
\end{equation}

\paragraph{Tetrads and projectors}

We want to work in a reference in which the velocity $u^\mu$ is purely timelike. This can always be found by the equivalence principle. Such a reference is called a tetrad. It also allows the metric to become the Minkowski metric in a neighbourhood of the point we consider.

It is useful to project tensors onto the and space-like and time-like subspaces defined by our tetrad (and we wish to do so in a coordinate-independent manner,  so just taking the 0th and $i $-th components in the tetrad will not suffice). We therefore define the projectors:

\begin{equation}
    h_{\mu \nu} = u_\mu u_\nu + g_{\mu \nu} \qquad \pi_{\mu\nu} = -u_\mu u_\nu
\end{equation}

respectively onto the space- and time-like subspaces.

\paragraph{Metrics}

The simplest one is the Schwarzschild metric. It describes a spherically symmetric object of mass $M$, in spherical coordinates. Defining $\Phi = -M/r$, we have:

\begin{equation}
    \dd{s}^2 = -(1+2\Phi)\dd{t}^2 + \frac{1}{1+2\Phi} \dd{r}^2
    + r^2 \qty(\dd{\theta}^2 + \sin^2\theta \dd{\varphi}) \label{eq:schwartzshild-line-element}
\end{equation}

or, equivalently,

\begin{equation}
    g_{\mu\nu} =  \diag{-(1+2\Phi),\, \frac{1}{1+2\Phi},\, r^2,\, r^2 \sin ^2 \theta }
\end{equation}

We can see that it approaches the flat metric $\eta_{\mu\nu} = \diag{-, +, +, +}$ in the limit $M\rightarrow 0$. Its determinant is $g = -r^4 \sin^2 \theta$.

\paragraph{Fluid mechanics}

In usual relativistic single-body mechanics, we use the 4-velocity $u^\mu$ and the corresponding 4-momentum $p^\mu = m u^\mu$. The 0-th component of this vector is the energy of the body, while the $i$-th components are its momentum: we then have $p^\mu p_\mu = m^2 = E^2 - \abs{p}^2$.

When dealing with a continuum, we will have a certain density of particle per unit of volume, we call this $n$. The current of particles is then $N^\mu = n u^\mu$. If these particles have a certain rest mass $m_0$, we can then define the vector $\rho_0 u^\mu = m_0 n u^\mu = m_0 N^\mu$.

This satisfies a conservation equation: $\nabla_\mu(\rho_0 u^\mu) = 0$.

Particles in a fluid can have three kinds of energy we concern ourselves with: mass, kinetic energy and other forms of energy (thermal, chemical, nuclear\dots).
We can always perform a change of coordinates to bring us to a frame in which the kinetic energy is zero. We write the sum of the other two forms of energy as $\rho = \rho_0 (1+\epsilon)$. So, $\epsilon$ is the ratio of the internal non-mass energy to the mass.

Now, the vector $\rho u^\mu$ describes the flux of energy.
We can then write the equation for the conservation of momentum:

\begin{equation}
    f^\mu = \nabla_\nu (\rho u^\mu u^\nu)
\end{equation}

\paragraph{Spherical accretion}

We work with the Schwarzschild metric \eqref{eq:schwartzshild-line-element}; we treat a fluid with 4-velocity $u^\mu$ in spherical coordinates, since the problem we are looking at is stationary and spherically symmetric the velocity is:

\begin{equation}
    u^\mu = \begin{pmatrix}
        \gamma^2 / y\\
        yv\\
        0\\
        0
    \end{pmatrix}
\end{equation}

where we define $\gamma = \qty(1-v^2)^{-1/2}$, $y=\gamma \sqrt{1+2\Phi}$.

The conservation of mass holds: if $\rho_0$ is the rest mass density of the fluid, we must have $\nabla_\mu \qty(\rho_0 u^\mu) =0$. This, using the formula for covariant divergence \eqref{eq:covariant-divergence}, yields:

\begin{equation}
    \dv{}{r} \qty(\rho_0 yvr^2) = 0
\end{equation}

In the newtonian limit both $\gamma$ and $y$ approach 1; also, the infalling mass rate $\Dot{M}$ at a certain radius is $\rho_0 (r) v(r) 4\pi r^2$. Then, by continuity to the newtonian limit, the quantity which is constant wrt the radius must be $\Dot{M} / (4\pi)$.

We also have the Euler equation:

\begin{equation}
    (p+\rho) a^\mu = - h^{\mu \nu} \partial_\nu p
\end{equation}

And the equation for ???

\begin{equation}
    \dv{\rho}{\tau} = \frac{p+\rho}{\rho_0} \dv{\rho_0}{\tau}
\end{equation}

From these we can show that the quantity $\gamma \sqrt{1+2\Phi} h$ (where $h = (p+\rho)/\rho_0$ is the specific enthalpy), is a constant of motion.
In the nonrelativistic, weak-field limit this  becomes

\begin{equation}
    \gamma h \sqrt{1+2\Phi} \approx \frac{p}{\rho_0} + \frac{v^2}{2} - \frac{M}{r} + \epsilon = \const
\end{equation}


\subsection{Taub}

This section summarizes my study of A. H. Taub's review of relativistic fluid dynamics, \cite{taub}.

Nonrelativistic fluid mechanics are described by the equations:

\begin{align}
    \partial_t \rho + \partial_i (\rho v^i) =0 \\
    \rho \qty(\partial_t v^i + v^j \partial_j v^i) = \partial_j T^{ij} \\
    \rho \partial_t E + v^i \partial_i E = \partial_i \qty(T^{ij}v_j + \lambda\partial^i T)
\end{align}

where $\rho$ is the density of the fluid,
$v^i$ are the components of its velocity,
$T^{ij}$ is the stress tensor (or, equivalently, the space-like components of the energy-momentum tensor),
$E$ is the energy of the fluid,
$\lambda$ is the thermal conductivity,
$T$ is the temperature of the fluid.

The nonrelativistic stress tensor can be written as:

\begin{equation}
    T_{ij} = -(p + \xi \partial_k v^k ) \delta_{ij} + \eta \partial_{(i} v_{j)}
\end{equation}

where $p$ is the (isotropic) pressure, $\eta$ the viscosity, $\xi$ is the compression viscosity. We are assuming that the normal stresses are only those exerted by pressure, so the diagonal terms $T_{ii}$ (not summed) must just be $-p$. So, the term $-\xi \partial_k v^k$ must equal $\eta \partial_{(i} v_{i)} = 2\eta \partial_i v_i$ (not summed). Therefore, by isotropy, $\xi = 2\eta/3$.

Note that we are working in Euclidean 3D space, so the metric is the identity and upper and lower indices are equivalent.

The energy is a sum of kinetic and specific energy:

\begin{equation}
    E = v^i v_i /2 + \varepsilon
\end{equation}

where $\varepsilon$ is the specific energy (of a type that is different from kinetic) per unit mass.

Testing git commit: $e^{i\pi}+1=0$.

New test for git
Test from OL