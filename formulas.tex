\documentclass[main.tex]{subfiles}
\begin{document}


\subsection{Nobili}

\paragraph{Tensor calculus}

The covariant derivative keeps account of the shifting of the basis vectors:

\begin{equation}
    \nabla_\mu A^\nu = \partial_\mu A^\nu + \Gamma^\nu _{\alpha \mu}  A^\alpha
\end{equation}

The rank-3 objects $\Gamma$ are called Christoffel symbols. They are not tensors! they depend on the choice of basis $e_\alpha$, and they satisfy $\nabla _\mu e_\alpha = \Gamma ^\nu _{\mu \alpha} e_\nu$.

If we have the metric, they can be calculated as:

\begin{equation}
    \Gamma^\mu_{\nu \rho} = \frac{1}{2} g^{\mu \alpha} \qty(
    \partial_\rho g_{\alpha \nu} +
    \partial_\nu g_{\alpha \rho} -
    \partial_\alpha g_{\nu \rho}
    ) \label{eq:christoffel-symbols-from-metric}
\end{equation}

This also tells us that they are symmetric in the lower two indices: $\Gamma ^\mu _{\nu \rho} = \Gamma ^\mu _{\rho \nu}$.

The divergence of a vector field $A^\mu$ can be calculated as:

\begin{equation}
    \nabla_\mu A^\mu = \frac{1}{\sqrt{-g}}\partial_\mu \qty(\sqrt{-g}A^\mu) \label{eq:covariant-divergence}
\end{equation}

where $g$ is the determinant of the metric.

We can also define the Dalambertian operator $\square = \nabla_\mu \nabla^\mu = \nabla_\mu \partial^\mu$, which can only act on scalars, and it does so like:

\begin{equation}
    \square A = \nabla_\mu (\partial^\mu A) =  \frac{1}{\sqrt{-g}}\partial_\mu \qty(\sqrt{-g}\partial^\mu A)
\end{equation}

If we differentiate and antisymmetryze (so, take the rotor of) an antisymmetric tensor $F_{[\mu \nu]}$, the Christoffel symbols cancel:

\begin{equation}
    \nabla_{[\mu} F_{\nu\rho]} = \partial_{[\mu} F_{\nu\rho]}
\end{equation}

The derivative with respect to proper time is $\dv{}{\tau} = u^\mu \partial_\mu$.

\index{covariant acceleration}Covariant acceleration is defined as:

\begin{equation} \label{eq:covariant-acceleration-def}
    a^\nu = u^\mu \nabla_\mu u^\nu
\end{equation}

\paragraph{Curvature}

The curvature of spacetime is fully described by the Riemann curvature tensor, which is a fourth rank tensor: for any generic vector \(V^\mu\),

\begin{equation} \label{eq:riemann-tensor-def}
    R ^{\mu} _{\nu \rho \sigma} V^\nu \defeq [\nabla_\rho, \nabla_\sigma]   V^\mu
\end{equation}

It can be calculated using the Christoffel symbols, and while they are not tensors \(R ^{\mu} _{\nu \rho \sigma}\) is one. This result follows by direct computation from formula \eqref{eq:riemann-tensor-def}.

\begin{equation}
    R ^{\mu} _{\nu \rho \sigma} =
     \partial_\rho \Gamma^\mu_{\nu \sigma}
    -\partial_\sigma \Gamma^\mu_{\nu \rho}
    +\Gamma^\mu_{\rho \lambda} \Gamma ^{\lambda} _{\sigma \nu}
    -\Gamma^\mu_{\sigma \lambda} \Gamma ^{\lambda} _{\rho \nu}
\end{equation}

The Christoffel symbols can be nonzero if we choose certain coordinates even for flat spacetime, but the Riemann tensor is zero iff the spacetime is flat.

\paragraph{Geodesics}

If we have a path $x^\mu(\lambda)$, we would like to see if it is a geodesic, that is, if it is stationary with respect to path length. To do this we can stationarize the action corresponding to the lagrangian $\Lagr (x, \Dot{x}) = g_{\mu\nu} \Dot{x}^\mu \Dot{x}^\nu$ (where we use $\Dot{x} = \dv*{x}{\lambda}$). The Lagrange equations then are:

\begin{equation}
    \Ddot{x}^\mu + \Gamma^\mu_{\nu\rho} \Dot{x}^\nu \Dot{x}^\rho = 0
\end{equation}

Where $\Gamma$ are the Christoffel symbols, which can be calculated by differentiating the metric, as shown in \eqref{eq:christoffel-symbols-from-metric}. $\Lagr$ is an integral of these Lagrange equations.

If the parameter $\lambda$ is taken to be the proper time $s$, then the equation is

\begin{equation}
    \dv{u^\mu}{s} + \Gamma^\mu _{\nu\rho} u^\nu u^\rho = 0
\end{equation}

Notice that this is equivalent to the covariant acceleration \eqref{eq:covariant-acceleration-def} being zero.

\paragraph{Fermi-Walker transport}

Take a general vector field \(V ^{\mu} (s)\) defined along a curve, with its tangent vector \(u^\mu\) whose covariant acceleration is \(a^\mu\).
Then we say that \(V^\mu\) is transported according to Fermi-Walker iff it satisfies

\begin{equation}
    \dot{V}^\mu  = u^\nu \nabla_\nu V^\nu
    = V_\rho \qty(u^\mu a^\rho - a^\mu u^\rho)
\end{equation}

This condition is always satisfied by \(V^\mu = u^\mu\), since \(a^\mu u_\mu = 0\), whether or not the curve is a geodesic. The tangent vector is \emph{parallel} transported only for geodesics.

\paragraph{Tetrads and projectors}

We want to work in a reference in which the velocity $u^\mu$ is purely timelike. This can always be found by the equivalence principle. Such a reference can be completed into what  is called a tetrad, for which the metric becomes the Minkowski metric in a neighbourhood of the point we consider.

We call the velocity \(u^\mu = V^\mu _{(0)}\), and add to it three other vectors \(V^\mu_{(i)}\) such that

\begin{equation}
    g_{\mu\nu} V^\mu _{(\alpha)} V^\nu _{(\beta)} = \eta_{(\alpha) (\beta)}
\end{equation}

where the brackets around the indices denote the fact that they label four vectors, not the components of a tensor.

We can choose the vectors \(V_{(i)}^\mu\) so that they are Fermi-Walker transported along the worldline defined by \(u^\mu\): this allows us to find the relativistic equivalent of a nonrotating frame of reference.

It is useful to project tensors onto the space-like and time-like subspaces defined by our tetrad (and we wish to do so in a coordinate-independent manner,  so just taking the 0th and $i $-th components in the tetrad will not suffice). We therefore define the projectors:

\begin{equation}
    h_{\mu \nu} = u_\mu u_\nu + g_{\mu \nu} \qquad \pi_{\mu\nu} = -u_\mu u_\nu
\end{equation}

respectively onto the space- and time-like subspaces.

\paragraph{Metrics}

The simplest physically relevant one is the Schwarzschild metric. It describes a spherically symmetric object of mass $M$, in spherical coordinates. Defining $\Phi = -M/r$, we have:

\begin{equation}
    \dd{s}^2 = -(1+2\Phi)\dd{t}^2 + \frac{1}{1+2\Phi} \dd{r}^2
    + r^2 \qty(\dd{\theta}^2 + \sin^2\theta \dd{\varphi}) \label{eq:schwartzshild-line-element}
\end{equation}

or, equivalently,

\begin{equation}
    g_{\mu\nu} =  \diag\qty(-(1+2\Phi),\, \frac{1}{1+2\Phi},\, r^2,\, r^2 \sin ^2 \theta )
\end{equation}

We can see that it approaches the flat metric $\eta_{\mu\nu} = \diag\qty(-, +, +, +)$ in the limit $M\rightarrow 0$. Its determinant is $g = -r^4 \sin^2 \theta$.

\paragraph{Fluid mechanics}

In usual relativistic single-body mechanics, we use the 4-velocity $u^\mu$ and the corresponding 4-momentum $p^\mu = m u^\mu$. The 0-th component of this vector is the energy of the body, while the $i$-th components are its momentum: we then have $p^\mu p_\mu = m^2 = E^2 - \abs{p}^2$.

When dealing with a continuum, we will have a certain density of particles per unit of volume, we call this $n$. The current of particles is then $N^\mu = n u^\mu$. If these particles have a certain rest mass $m_0$, we can then define the vector $\rho_0 u^\mu = m_0 n u^\mu = m_0 N^\mu$.

This satisfies a conservation equation: $\nabla_\mu(\rho_0 u^\mu) = 0$.

Particles in a fluid can have three kinds of energy we concern ourselves with: mass, kinetic energy and other forms of energy (thermal, chemical, nuclear\dots).
We can always perform a change of coordinates to bring us to a frame in which the kinetic energy is zero. We write the sum of the other two forms of energy as $\rho = \rho_0 (1+\epsilon)$. So, $\epsilon$ is the ratio of the internal non-mass energy to the mass.

Now, the vector $\rho u^\mu$ describes the flux of energy.
We can then write the equation for the conservation of momentum:

\begin{equation}
    f^\mu = \nabla_\nu (\rho u^\mu u^\nu)
\end{equation}

\paragraph{Ideal fluids}

They are fluids with \(\eta=\xi=\kappa=0\), that is, without viscosity (neither compressive nor shear) nor heat transmission.
They are described by the following stress-energy tensor:

\begin{equation}
    T^{\mu\nu} = \rho u^\mu u^\nu + p h^{\mu\nu}
\end{equation}

\paragraph{Spherical accretion}

We work with the Schwarzschild metric \eqref{eq:schwartzshild-line-element}; we treat a fluid with 4-velocity $u^\mu$ in spherical coordinates, since the problem we are looking at is stationary and spherically symmetric the velocity is:

\begin{equation}
    u^\mu = \begin{pmatrix}
        \gamma^2 / y\\
        yv\\
        0\\
        0
    \end{pmatrix}
\end{equation}

where we define the Lorentz factor as usual, $\gamma = \qty(1-v^2)^{-1/2}$, and $y=\gamma \sqrt{1+2\Phi}$.

The conservation of mass holds: if $\rho_0$ is the rest mass density of the fluid, we must have $\nabla_\mu \qty(\rho_0 u^\mu) =0$. This, using the formula for covariant divergence \eqref{eq:covariant-divergence}, yields:

\begin{equation}
    \dv{}{r} \qty(\rho_0 yvr^2) = 0
\end{equation}

In the newtonian limit both $\gamma$ and $y$ approach 1; also, the infalling mass rate $\Dot{M}$ at a certain radius is $\rho_0 (r) v(r) 4\pi r^2$. Then, by continuity to the newtonian limit, the quantity which is constant wrt the radius must be $\Dot{M} / (4\pi)$.

We also have the Euler equation:

\begin{equation}
    (p+\rho) a^\mu = - h^{\mu \nu} \partial_\nu p
\end{equation}

And the equation for the variation of the total internal energy, which holds for ideal fluids at constant entropy:

\begin{equation}
    \dv{\rho}{\tau} = \frac{p+\rho}{\rho_0} \dv{\rho_0}{\tau}
\end{equation}

From these we can show that the quantity $\gamma h \sqrt{1+2\Phi} $ (where $h = (p+\rho)/\rho_0$ is the specific enthalpy), is a constant of motion.
In the nonrelativistic, weak-field limit this  becomes

\begin{equation}
    \gamma h \sqrt{1+2\Phi} \approx \frac{p}{\rho_0} + \frac{v^2}{2} - \frac{M}{r} + \epsilon = \const
\end{equation}



\subsection{Taub}

This section summarizes my study of A. H. Taub's review of relativistic fluid dynamics, \cite{taub}.

\paragraph{Nonrelativistic}

Nonrelativistic fluid mechanics are described by the equations:

\begin{subequations}
\begin{align}
    \partial_t \rho + \partial_i (\rho v^i) &= 0 \\
    \rho \qty(\partial_t v^i + v^j \partial_j v^i) &= \partial_j T^{ij} \\
    \rho \partial_t E + v^i \partial_i E &= \partial_i \qty(T^{ij}v_j + \kappa\partial^i T)
\end{align}
\end{subequations}

where $\rho$ is the density of the fluid,
$v^i$ are the components of its velocity,
$T^{ij}$ is the stress tensor (or, equivalently, the space-like components of the energy-momentum tensor),
$E$ is the energy of the fluid,
$\kappa$ is the thermal conductivity,
$T$ is the temperature of the fluid.

The nonrelativistic stress tensor can be written as:

\begin{equation}
    T_{ij} = -(p + \xi \partial_k v^k ) \delta_{ij} + \eta \partial_{(i} v_{j)}
\end{equation}

where $p$ is the (isotropic) pressure, $\eta$ the viscosity, $\xi$ is the compression viscosity. We are assuming that the normal stresses are only those exerted by pressure, so the diagonal terms $T_{ii}$ (not summed) must just be $-p$. So, the term $-\xi \partial_k v^k$ must equal $\eta \partial_{(i} v_{i)} = 2\eta \partial_i v_i$ (not summed). Therefore, by isotropy, $\xi = 2\eta/3$.

Note that we are working in Euclidean 3D space, so the metric is the identity and upper and lower indices are equivalent.

The energy is a sum of kinetic and specific energy:

\begin{equation}
    E = v^i v_i /2 + \varepsilon
\end{equation}

where $\varepsilon$ is the specific energy (of a type that is different from kinetic) per unit mass.

\paragraph{Relativistic}

The dynamics of the fluid are described by the conservation of the stress-energy tensor \(\nabla_\mu T ^{\mu \nu} =0\) and the conservation of mass \(\nabla_\mu (\rho u^\mu) =0\).

Any stress-energy tensor can be decomposed in its space and time-like parts in the local rest frame of the fluid:

\begin{equation}
    T_{\mu \nu} = w u_\mu u_\nu + w_\mu u_\nu + u_\mu w_\nu + w _{\mu \nu}
\end{equation}

where

\begin{subequations}
\begin{align}
  w &=  T _{\mu \nu} u^\mu u^\nu = \rho_0 (1 + \varepsilon) \\
  w_\mu &= T _{\nu \sigma} h^\sigma _\mu u^\nu  = -\kappa h_\mu ^\sigma  \qty(\partial_\sigma T + T a_\sigma)\\
  w_{\mu \nu} &= T_{\rho \sigma} h^\rho _\mu h^\sigma_\nu = \qty(p - \xi \theta) h_{\mu \nu} - 2 \eta \sigma_{\mu \nu}
\end{align}
\end{subequations}

with \(\theta = \nabla_\mu u^\mu\), \(a_\mu\) is the covariant acceleration, \(\sigma_{\sigma \tau} = \frac[i]{1}{2} \qty(\nabla_\mu u_\nu + \nabla_\mu u_\nu) h^\mu_\sigma h^\nu _\tau - \frac[i]{1}{3} \theta h_{\sigma \tau}\),
and as in the nonrelativistic section \(\eta\) is the viscosity, \(\xi\) is the compression viscosity,  \(\kappa\) is the thermal conductivity, \(T\) is the temperature field, \(p\) is the pressure, \(\rho_0\) is the rest mass density while \(\rho = \rho_0 (1 + \varepsilon)\) is the rest energy.

\section{Nobili, Turolla, Zampieri}

See \cite{NTZ91}.

The cooling function \(\Lambda (T)\) is defined by the following relation, which describes the variation in the energy density by radiative processes:

\begin{equation}
    \dv{U}{t} = n^2_b \qty(\Gamma(T) - \Lambda (T))
\end{equation}

where \(U\) is the energy density (measured in \si{\erg\per\cubic\centi\metre}), \(n_b\) is the baryon density (measured in \si{\per\cubic\centi\metre}), while \(\Gamma\) and \(\Lambda\) are the heating and cooling functions, both measured in \si{\erg\cubic\centi\metre\per\second}, see \cite[equation 1]{Gnedin_2012}.

The cooling function of the infalling gas is

\begin{equation}
    \begin{split}
    \Lambda (T) &= \left(
    \qty(
    \num{1.42e-27}T^{1/2} \qty(
    1 + \num{4.4e-10}T
    ) + \num{6.0e-22}T^{-1/2}
    )^{-1} \right. \\
    & \quad \left. + \num{e25} \qty(\frac{T}{\SI{1.5849e4}{K}})^{-12}
    \right)^{-1} \si{\erg\per\cubic\centi\metre\per\second}
    \end{split}
\end{equation}

The version of this equation in Stellingwerf and Buff is similar: the first constant is \(\SI{2.4e-27}{} \) instead of \(\SI{1.42e-27}{} \), and the factor \(\qty(1 + \SI{4.4e-10}{}T)\) is just \(1\).

\section{Thorne's PSTF moment formalism}

Following \cite{thorne1981}.

Given any tensor \(A^{\mu_1 \dots \mu_k}\) we can use the tensor \(h^{\mu\nu}\) to project it into the space-like subspace defined by the velocity \(u^\mu\):

\begin{equation}
    A^{\mu_1 \dots \mu_k} \rightarrow \qty(A^{\mu_1 \dots \mu_k})^P
    = \qty(\prod_i h^{\mu_i}_{\nu_i}) A^{\nu_1 \dots \nu_k}
\end{equation}

Then, we can take the symmetric part of any (?) tensor as outlined in \Nameref{sec:notational-preface}:

\begin{equation}
    A^{\mu_1 \dots \mu_k} \rightarrow \qty(A^{\mu_1 \dots \mu_k})^S
    = A^{(\mu_1 \dots \mu_k)}
\end{equation}

We can select the trace-free part of a projected, symmetric tensor by

\begin{equation}
    A^{\mu_1 \dots \mu_k} \rightarrow \qty(A^{\mu_1 \dots \mu_k})^{TF}
    = \sum _{i=0}   ^{\lfloor k/2 \rfloor}
    (-1)^i \frac{k! (2k-2i-1)!!}{(k-2i)! (2k-1)!! (2i)!!}
    h^{(\alpha_1 \alpha_2} \dots h^{\alpha_{2i-1} \alpha_{2i}}
    A^{\alpha_{2i+1} \dots \alpha_k) \beta_1 \dots \beta_i}\,_{\beta_1 \dots \beta_i}
\end{equation}

To see what this is doing, let us consider its action on a rank-two tensor:

\begin{equation}
    A^{\mu\nu} \rightarrow A^{\mu\nu} - \frac{1}{3} h^{\mu\nu} A^{\rho}_\rho
\end{equation}

Now, let us consider all the unit vectors \(n^\mu\) in the space normal to the velocity, which have \(n_\mu u^\mu = 0\) and \(n^\mu n_\mu = 1\). They span a three-dimensional sphere.

If we have a function \(F\colon S^2 \rightarrow \mathbb R\), we can decompose it into harmonics as such:

\begin{equation}
    F(n) = \sum _{k=0}   ^{\infty}
    \mathscr F_{\alpha_1 \dots \alpha_k} \prod_{i=0}^k n^{\alpha_i}
\end{equation}

Where the PTSF moments \(\mathscr F_{\alpha_1 \dots \alpha_k}\) can be computed as

\begin{equation}
    \mathscr F_{\alpha_1 \dots \alpha_k} =
    \frac{(2k+1)!!}{4 \pi k!} \qty(\int F \prod_{i=0}^k n^{\alpha_i}  \dd{\Omega}  )^{TF}
\end{equation}

Now, consider a photon, whose trajectory in spacetime is parametrized as \(\gamma(\xi)\), with a choice of \(\xi\) such that the photon's momentum is

\begin{equation}
    p = \dv{}{\xi}
\end{equation}

Now, our observer has a timelike velocity \(u^\mu\). We can find a spacelike vector \(n^\mu\) corresponding to the space-like part of the movement of the photon, or

\begin{equation}
    p^\mu = (- u^\nu p_\nu) (u^\mu + n^\mu)
\end{equation}

Now, we define a parameter \(l\) which corresponds to the space distance the photon moved through in this frame (this is \emph{not} covariant!)

\begin{equation}
    l = \int  (-u^\nu p_\nu) \dd{\xi}
\end{equation}

now, \(\dv*{}{l} \) is parallel to \(p\) but it has different length, in fact since \(\dv*{l}{\xi} = (-u^\nu p_\nu) \) it is \(\dv*{}{l} = u + n \).

\end{document}
