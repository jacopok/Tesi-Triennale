\documentclass[main.tex]{subfiles}
\begin{document}

% \subsection{Minkowski spacetime}
%
% how we describe events in special relativity
%
% The mathematical framework of differential geometry: coordinate systems, comoving tetrads (viewed through a fluid-dynamics point of view: introducing the velocity field of a fluid)
%
% Covariant differentiation, parallel transport, Fermi-Walker transport
%
% \subsection{General relativity}
%
% Curvature: the Riemann tensor, the Einstein field equations, their resolution in the spherically symmetric case: the Schwarzschild metric, the approximations we will make (the fluid not being self-gravitating)

\subsection{Special relativity}

Special Relativity is a theory which satisfies the following axioms \cite[]{Lechner:2014}:

\begin{enumerate}
    \item space and time are homogeneous (i.\ e.\ shift-invariant), space is isotropic (i.\ e.\ rotation-invariant);
    \item the speed of light is the same in every inertial reference frame;
    \item all the laws of physics are written in the same way in every inertial reference frame.
\end{enumerate}

In special relativity, instead of having vectors in 3D space and a time scalar coordinate, we denote events as points in 4D spacetime, which is an intrinsically flat semi-Riemannian manifold with metric signature \((-, +, +, +)\), with coordinates such as \(x^\mu = (t, x, y, z)\).
This is called Minkowski flat spacetime.

This difference is not just semantic: the spacetime formalism is needed because the axioms are equivalent to the conservation of the spacetime interval \(\dd{s}^2 = \eta_{\mu\nu} \dd{x}^\mu\dd{x}^\nu\), and the only transformations between inertial reference frames which leave it invariant often \emph{mix} time and space:
they are  represented in 4D spacetime as \(x^\mu \rightarrow \Lambda\indices{^\mu_\alpha} x^\alpha + a^\mu\), with the \(\Lambda\indices{_\mu^\alpha}\) being \((1,1)\) tensors which satisfy \(\Lambda\indices{^\mu_\alpha} \Lambda\indices{^\nu_\beta} \eta_{\mu\nu} = \eta_{\alpha \beta}\), and \(a^\mu\) being a generic constant 4-vector.

The flat metric allows us to compute the lengths of vectors, and since it is indefinite there are nonzero vectors of positive, negative and zero spacetime length. These are respectively called spacelike, timelike and null-like.

We define the proper time \(\tau\) by \(\dd{\tau} \defeq \sqrt{-\dd{s}^2} \). Unlike coordinate time \(x^0 = t\) this has the advantage of being Lorentz-invariant.

We can define a tensorial velocity by differentiating the position with respect to proper time: \(u^\mu \defeq \dv*{x^\mu}{\tau}\). Defined this way, the so-called 4-velocity trasforms like a tensor. If \(\vec{v}\) is the regular three-velocity and \(v\) is its magnitude, we define \(\gamma = 1/\sqrt{1-v^2} \) and then:
\(u^\mu = \qty(\gamma, \gamma \vec{v})\).

Now, differentiating any function of position looks like \(\dv*{T^{A_k}(x^\mu)}{\tau} = (\nabla_\mu T^{A_k}) \dv*{x^\mu}{\tau} = u^\mu \nabla_\mu T^{A_k} \).

\index{covariant acceleration}
Once we have this, we can define the 4-acceleration:
\begin{equation} \label{eq:covariant-acceleration-def}
    a^\nu = \dv{u^\mu}{\tau}  = u^\mu \nabla_\mu u^\nu \,.
\end{equation}

The 4-velocity is a unit vector: \(u^\mu u_\mu = -1\), and by differentiating this relation we get the often used identity \(u^\mu a_\mu = 0\).

We also define the 4-momentum \(p^\mu = m u^\mu\), where \(m\) is the rest mass of the body at hand.

The 0-th component of the 4-momentum vector is the energy of the body, while the $i$-th components define a new relativistic 3-momentum \(p^i = \gamma m v^i\): we then have $p^\mu p_\mu = m^2 = E^2 - \abs{p}^2$.

\subsection{Differential geometry and tensor calculus}

\paragraph{Metric}

The metric tensor \(g_{\mu\nu}\) is a symmetric \((0,2)\) tensor which defines a scalar product at every point in our manifold: \(x \cdot y = g_{\mu\nu} x^\mu y^\nu\).
It is not intrinsic to the manifold.
By integrating the velocity vector we can find the lengths of curves \(x^\mu(\lambda)\):
\begin{equation}
    L = \int \sqrt{g_{\mu\nu} \dv{x^\mu}{\lambda} \dv{x^\nu}{\lambda} }  \dd{\lambda} \,.
\end{equation}

For a flat spacetime we use the Minkowski metric \(\eta_{\mu\nu}\). In general, in the presence of matter the manifold will be intrinsically curved (the meaning of this will be discussed in section \ref{par:curvature}), so there will not be a coordinate transformation to cast \(g_{\mu\nu}\) in the form \(\eta_{\mu\nu}\). If we choose a certain point \(P\), however, it is possible to find a transformation in order to impose the conditions \(g_{\mu\nu}(P)=\eta_{\mu\nu}(P)\), \(\partial_\rho g_{\mu\nu}(P) = 0\) \cite[pages 49--50]{Carroll:1997ar}.

The spacetime interval \(\dd{s}^2\), defined with \(g_{\mu\nu}\) instead of \(\eta_{\mu\nu}\), is still an invariant scalar.

\paragraph{Tensor calculus}

An object such as \(\partial_\mu A^\nu\) (the matrix of the partial derivatives with respect to the coordinates of some vector \(A^\nu\)) does not in general transform as a tensor.

Because of this, we wish to define a new kind of derivative, which is \emph{covariant}, that is, which transforms as a tensor. There is not an intrinsic way to do this, but for any choice of covariant derivative one makes it can be written as
\begin{equation}
    \nabla_\mu A^\nu = \partial_\mu A^\nu + \Gamma^\nu _{\alpha \mu}  A^\alpha \,.
\end{equation}
%
where the rank-3 objects $\Gamma$ are called Christoffel symbols.
They are not tensors: they depend on the choice of basis $e_\alpha$.

A specific covariant derivative can be chosed by imposing the condition of it being torsion-free: $\Gamma^\nu_{\alpha \mu} = \Gamma ^\nu _{\mu \alpha}$ and compatible with the metric: \(\nabla_\sigma g_{\mu\nu} = 0\) . This allows us to have a well-defined unique covariant derivative. If we make this assumption, the Christoffel symbols can be calculated as:
%
\begin{equation}  \label{eq:christoffel-symbols-from-metric}
    \Gamma^\mu_{\nu \rho} = \frac{1}{2} g^{\mu \alpha} \qty(
    \partial_\rho g_{\alpha \nu} +
    \partial_\nu g_{\alpha \rho} -
    \partial_\alpha g_{\nu \rho}
    )\,.
\end{equation}

The covariant derivative is the same as the coordinate derivative for scalars.

We can define the covariant derivative of higher order tensor analogously; adding a Christoffel symbol for every new index. The symbols corresponding to lower indices have a minus sign: this can be seen by differentiating a scalar such as \(\nabla_\nu (A_\mu B^\mu) \overset{!}{=}  \partial_\nu (A_\mu B^\mu)\) and matching the Christoffel terms.

The divergence of a vector field $A^\mu$ can be calculated as:
\begin{equation}
    \nabla_\mu A^\mu = \frac{1}{\sqrt{-g}}\partial_\mu \qty(\sqrt{-g}A^\mu) \label{eq:covariant-divergence}
\end{equation}
where $g$ is the determinant of the metric.

\paragraph{Stokes' theorem}

Following \cite[]{Unger:2016}.
If we have a manifold of dimension \(n\), and an \(n\)-dimensional region \(V\) of this manifold equipped with coordinates \(x^\mu\) a metric \(g_{\mu\nu}\), with a submanifold boundary \(\partial V\) equipped with coordinates \(y^\mu\) and the induced metric \(h_{\alpha\beta} = \qty(\pdv*{x^\mu}{y^\alpha}) \qty(\pdv*{x^\nu}{y^\beta}) g_{\mu\nu}\), and for which we have a properly oriented normal vector \(n^\mu (y)\); then for any vector \(f^\mu\) we have:

\begin{equation} \label{eq:stokes-theorem}
    \int _{V} \nabla_\mu f^\mu \sqrt{\abs{\det g}}  \dd[n]{x}  = \int _{\partial V} f^\mu  n_\mu \sqrt{\abs{\det h}} \dd[n-1]{y} \,.
\end{equation}

\paragraph{Geodesics}

A path $x^\mu(\lambda)$ is called a \emph{geodesic} if it is stationary with respect to path length.
To check whether a given path is a geodesic we can stationarize the action corresponding to the Lagrangian $\Lagr (x, \Dot{x}) = g_{\mu\nu} \Dot{x}^\mu \Dot{x}^\nu$ (where we use the notation $\Dot{x} = \dv*{x}{\lambda}$). The Lagrange equations then are:

\begin{equation}
    \Ddot{x}^\mu + \Gamma^\mu_{\nu\rho} \Dot{x}^\nu \Dot{x}^\rho = 0
\end{equation}
where $\Gamma^\mu_{\nu \rho}$ are the Christoffel symbols, which can be calculated by differentiating the metric, as shown in \eqref{eq:christoffel-symbols-from-metric}. $\Lagr$ is an integral of these Lagrange equations.

If the parameter $\lambda$ is taken to be the proper time $s$, then the equation is

\begin{equation}
    \dv{u^\mu}{s} + \Gamma^\mu _{\nu\rho} u^\nu u^\rho = 0 \,.
\end{equation}

Notice that this is equivalent to the covariant acceleration \eqref{eq:covariant-acceleration-def} being zero.

\paragraph{Fermi-Walker transport}

Take a general vector field \(V ^{\mu} (s)\) defined along a curve, with the curve's tangent vector \(u^\mu\) whose covariant acceleration is \(a^\mu\).
Then we say that \(V^\mu\) is transported according to Fermi-Walker if it satisfies

\begin{equation} \label{eq:fermi-walker-transport}
    \dot{V}^\mu  = u^\nu \nabla_\nu V^\mu
    = 2 V_\rho u^{[\mu} a^{\rho]} \,.
\end{equation}

This condition is always satisfied by \(V^\mu = u^\mu\), since \(a^\mu u_\mu = 0\), whether or not the curve is a geodesic: the tangent vector is always \emph{Fermi-Walker} transported, but it is \emph{parallel} transported only for geodesics.

The justification of this definition is the fact that we want the transformations of our tetrad to be infinitesimal Lorentz boosts, which are generated by antisymmetric tensors, and we want to prohibit any rotations in the plane orthogonal to \(a^\mu\) and \(u^\mu\).

\paragraph{Tetrads and projectors} \label{par:tetrads}

We want to work in a reference in which the velocity $u^\mu$ is purely timelike. This can always be found by the equivalence principle. Such a reference can be completed into what  is called a tetrad, for which the metric becomes the Minkowski metric in a neighbourhood of the point we consider.

We call the velocity \(u^\mu = V^\mu _{(0)}\) and complement it with three other vectors \(V^\mu_{(i)}\) such that

\begin{equation}
    g_{\mu\nu} V^\mu _{(\alpha)} V^\nu _{(\beta)} = \eta_{(\alpha) (\beta)}
\end{equation}
where the brackets around the indices denote the fact that they label four vectors, not the components of a tensor.

We can choose the vectors \(V_{(i)}^\mu\) so that they are Fermi-Walker transported along the worldline defined by \(u^\mu\): this allows us to find the relativistic equivalent of a non-rotating frame of reference.

It is useful to project tensors onto the space-like and time-like subspaces defined by our tetrad (and we wish to do so in a coordinate-independent manner,  so just taking the 0th and $i $-th components in the tetrad will not suffice). We therefore define the projectors:

\begin{equation}
    h_{\mu \nu} = u_\mu u_\nu + g_{\mu \nu} \qquad \qquad \pi_{\mu\nu} = -u_\mu u_\nu
\end{equation}
respectively onto the space- and time-like subspaces defined by the four-velocity.

\paragraph{Killing vector fields}

Following \cite[section 25.2, page 650]{MisnerThorneWheeler:1973}.
Suppose there is a certain direction (which, for simplicity, we assume to be along one of our coordinate axes) along which the metric is preserved: an \(\widetilde{\alpha}\) such that \(\partial_{\widetilde{\alpha}} g_{\mu\nu} =0\).

Then the metric properties of curves along the manifold are unchanged if we shift their coordinate representation by a constant along the \(\widetilde{\alpha}\) coordinate axis.

Let us call the direction of this translation \(\xi^\mu = \delta^\mu_{\widetilde{\alpha}}\) if we use this coordinate system. It can be shown by direct computation that

\begin{equation} \label{eq:killing-vector-identity}
    \nabla_{\nu} \xi_\mu = \frac{1}{2} \qty(\partial_{\widetilde{\alpha}}
    g_{\mu\nu} + \partial_\nu g_{\mu\widetilde{\alpha}} -
    \partial_\mu g_{\nu \widetilde{\alpha}})
\end{equation}
but by hypothesis the first term on the RHS of \eqref{eq:killing-vector-identity} is zero, therefore we have shown that \(\nabla_{\nu} \xi_\mu = \nabla_{[\nu} \xi_{\mu]}\) in this coordinate frame, but since this is a covariant equation it extends to every other one.

This can equivalently be stated by writing \(\nabla_{(\nu} \xi_{\mu)}=0\): this is called \emph{Killing's equation}. This is useful since: given a geodesic \(x^\mu(\lambda)\), for which we define \(u^\mu = \dv*{x^\mu}{\lambda} \), it must be the case that \(u^\nu \nabla_\nu u^\mu = 0 \). Then, the component of \(u^\mu\) along \(\xi^\mu\) (\(u^{\widetilde{\alpha}} = u^\mu \xi_\mu\)) is conserved:

\begin{equation}
    \dv{}{\lambda} \qty(u^\mu \xi_\mu) = u^\nu \nabla_\nu \qty(u^\mu \xi_\mu)
    = \cancelto{0}{\xi^\mu u^\nu \nabla_\nu u_\mu} + \cancelto{0}{u^\nu u^\mu \nabla_\nu \xi_\mu} \equiv 0 \,.
\end{equation}

\paragraph{Surfaces in space-time and acceleration decomposition}

Following \cite[section 4]{Taub:1978}.
We consider 3D space-like surfaces in 4D space-time: if a fluid is moving with velocity \(u^\mu\), we denote the solutions of the differential equation associated with the vector field as \(x^\mu (\xi^i, s)\), where \(\xi^i\) are the 3D coordinates of the starting position and \(s\) is the time at which we look at the solution. Then the ``starting'' hypersurface is \(\Sigma = \qty{x^\mu (\xi^i, 0)}\).

Suppose we have a curve \(\xi^i(\tau)\) in \(\Sigma\). Then we can define the two-dimensional surface defined by the evolution of \(\xi^i(\tau)\): \(x^{\mu} (\xi^i(\tau), s) \defeq x^\mu (\tau, s)\).
If we also define the ``spatial'' tangent vector \(\lambda^\mu = \dv*{x^\mu}{\tau} \), it follows from Schwarz's theorem that:

\begin{equation} \label{eq:schwarz-spacetime-tube}
    \pdv{x^\mu}{\tau}{s} =
    \pdv{x^\mu}{s}{\tau}
    \implies
    \pdv{u^\mu}{\tau} = \pdv{\lambda^\mu}{s} \,.
\end{equation}

Now let us take the spatial vectors of an orthonormal Fermi-Walker transported tetrad \(V^\mu_{(a)}\) as described in \Nameref{par:tetrads}, and express \(\lambda^\mu\) in this frame: its covariant components will be

\begin{equation} \label{eq:tetrad-components-lambda}
    X_{(a)} = V_{(a)\mu} \lambda^\mu \,.
\end{equation}

If we differentiate \eqref{eq:tetrad-components-lambda} with respect to \(s\), and use \eqref{eq:schwarz-spacetime-tube} with the fact that \(\dv{}{\tau} = \lambda^\mu \nabla_\mu \), we get:

\begin{subequations}
\begin{align}
    \dv{X_{(a)}}{s} &= \dv{V_{(a)\mu}}{s} \lambda^\mu + V_{(a)\mu} \dv{\lambda^\mu}{s}  \\
    &= V_{(a)}^\rho \cancelto{0}{\lambda^\mu u_\mu} a_\rho - \cancelto{0}{V_{(a)}^\rho  u_\rho} \lambda^\mu a_\mu
    + V_{(a)}^\nu \nabla_\mu u_\nu  \\
    &= V^\rho _{(a)} \lambda^\mu \nabla_\mu u_\rho  \\
    &= \qty(\nabla_\mu u_\rho) V^\rho_{(a)} V^{\mu}_{(b)} X^{(b)}
\end{align}
\end{subequations}
where in the last step we expressed everything with respect to the tetrad coordinate system. Therefore, in those coordinates, the evolution of the components \(X^{(a)}\) is linear, and defined by the tetrad components of the two-form \(\nabla_\mu u_\nu\). So, we want to decompose this tensor:

\begin{equation} \label{eq:covariant-acceleration-decomposition}
    \nabla_\sigma u_\tau =
    \underbrace{\omega_{\sigma \tau}}_{\substack{\text{spatial} \\ \text{rotation}}}
    + \underbrace{\sigma_{\sigma\tau}}_{\substack{\text{spatial} \\ \text{shear}}}
    + \underbrace{\frac{1}{3} \theta h_{\sigma\tau}}_{\substack{\text{spatial} \\
    \text{compression}}}
    - \underbrace{ a_\tau u_\sigma}_{ \text{acceleration}} \\
\end{equation}

\begin{enumerate}
    \item \(\theta = \nabla_\mu u^\mu\) is the bare trace of the tensor, corresponding to the expansion velocity;
    \item \(a_\mu = u^\nu \nabla_\nu u^\mu\) is the covariant acceleration;
    \item \(\sigma_{\sigma \tau} = \qty(\nabla_{(\mu} u_{\nu)}) h^\mu_\sigma h^\nu _\tau - \frac[i]{1}{3} \theta h_{\sigma \tau} = \nabla_{(\sigma} u_{\tau)} + a_{(\sigma} u_{\tau)} - \frac[i]{1}{3} \theta h_{\sigma \tau} \) is the spatial symmetric trace-free part of the tensor, which corresponds to the shear stress;
    \item \(\omega_{\sigma \tau} = h^\nu_\sigma h^\mu_\tau \nabla_{[\nu} u_{\mu]} = \partial_{[\tau} u_{\sigma]} + a_{[\tau} u_{\sigma]}\) is the spatial (antisymmetric, trace-free) rotation tensor.
\end{enumerate}

To verify the formulas given for \(\sigma_{\sigma \tau}\) and \(\omega_{\sigma \tau}\) it is enough to expand the definitions, simplifying the terms which contain products of the 4-acceleration and the 4-velocity; also, the terms such as \(u^\mu u_{\tau}\nabla_{\sigma} u_\mu \) vanish since \( 0= u_{\tau} \nabla_{\sigma} (u^\mu u_\mu) = 2 u^\mu u_{\tau}\nabla_{\sigma} u_\mu\).

\subsection{General Relativity}

\paragraph{The Equivalence Principle}

In the General theory of Relativity we make a stronger claim than that of the axioms of SR, which are only formulated for inertial reference frames.

The \emph{Einstein Equivalence Principle} states \cite[100]{Carroll:1997ar} that in small enough regions of spacetime the laws of physics are those of special relativity, and we cannot detect gravitational effects locally.
The frame of reference in which we must write the laws for them to appear in their special-relativistic form is called the Locally Inertial Reference Frame or Local Rest Frame (LRF).

Unlike special relativity, the transformation laws between different reference frames are not linear, but can be in general be represented as diffeomorphisms.

We model spacetime as a manifold which has intrinsic (basis-independent) curvature; an object which is modelled in newtonian mechanics as being in free fall, accelerated by a gravitational force, is modelled in general relativity as following a geodesic in the manifold.

\paragraph{Curvature} \label{par:curvature}

The intrinsic curvature of spacetime is fully described by the Riemann curvature tensor, which is a fourth rank tensor: for any generic vector \(V^\mu\),

\begin{equation} \label{eq:riemann-tensor-def}
    R ^{\mu} _{\nu \rho \sigma} V^\nu \defeq [\nabla_\rho, \nabla_\sigma]   V^\mu \,.
\end{equation}

It can be calculated using the Christoffel symbols, and while they are not tensors \(R ^{\mu} _{\nu \rho \sigma}\) is one. This result follows by expanding all the covariant derivatives in formula \eqref{eq:riemann-tensor-def}:

\begin{equation}
    R ^{\mu} _{\nu \rho \sigma} =
     \partial_\rho \Gamma^\mu_{\nu \sigma}
    -\partial_\sigma \Gamma^\mu_{\nu \rho}
    +\Gamma^\mu_{\rho \lambda} \Gamma ^{\lambda} _{\sigma \nu}
    -\Gamma^\mu_{\sigma \lambda} \Gamma ^{\lambda} _{\rho \nu}
\end{equation}

The Christoffel symbols can be nonzero if we choose certain coordinates even for flat spacetime, but the Riemann tensor is zero if and only if the spacetime is flat.

The Riemann tensor satisfies the following identities \cite[eqs. 8.45 and 8.76]{MisnerThorneWheeler:1973}:

\vspace{-.5cm}

\begin{subequations}
\begin{align}
  \nabla _{[\lambda} R_{\mu\nu]\rho \sigma} &= 0 \label{eq:bianchi-identities}  \\
  R_{\mu\nu\rho\sigma} &= R_{[\mu\nu][\rho\sigma]} = R_{[\rho\sigma][\mu\nu]}  \\
  R_{[\mu\nu\rho\sigma]} &= 0 = R_{\mu[\nu\rho\sigma]} \,.
\end{align}
\end{subequations}

If we define the Ricci tensor \(R_{\mu\nu} = R^\rho_{\mu \rho \nu}\) and the curvature scalar \(R = R_{\mu\nu}g^{\mu\nu}\), we can derive from \eqref{eq:bianchi-identities} the \emph{contracted Bianchi identity}  \(\nabla_\mu R = 2 \nabla_\nu R^{\nu}_{\mu}\), which means that \(\nabla_\mu \qty(R^{\mu\nu} - \frac[i]{1}{2} R g^{\mu\nu}) = 0\).

\paragraph{The Einstein Field Equations}

They describe the way the presence of matter changes the geometry of spacetime.
They involve the \emph{stress-energy tensor} \(T^{\mu\nu}\) which is defined in \Nameref{par:stress-energy-tensor} and the Einstein tensor \(G^{\mu\nu} = R^{\mu\nu} - \frac[i]{1}{2} R g^{\mu\nu}\), which is the only independent tensor satisfying the following properties: it can be constructed from only the Riemann tensor and the metric, it vanishes for flat spacetime and it identically satisfies the conservation laws \(\nabla_\mu G^{\mu\nu} = 0\).

The field equations (EFE) are:

\begin{equation} \label{eq:EFE}
  G^{\mu\nu} = 8 \pi T^{\mu\nu} \,.
\end{equation}

The constant comes by imposing continuity with the newtonian limit, in which we know the gravitational field \(\Phi\) is determined by the matter density \(\rho_0\) according to the Poisson equation \(\partial_i \partial^i \Phi = 4 \pi \rho_0\).

The matter density \(\rho_0\) is substituted in the relativistic formulation by \(T_{00}\) while the gravitational field \(\Phi\) is substituted by a small perturbation in the metric: \(\Phi = -\frac[i]{1}{2}  h_{00}\), with \(g_{\mu \nu} = \eta_{\mu\nu} + h_{\mu\nu}\) (see \cite[eq. 4.46]{Carroll:1997ar}).

The EFE can be written in a more general way by removing the condition that the LHS vanish for flat spacetime, and thus including there a \emph{cosmological constant}  term \(\Lambda g^{\mu\nu}\) with constant \(\Lambda\).
It is unclear whether this term should appear, and what the value of \(\Lambda\) should be.

\paragraph{The Schwarzschild solution}


The EFE are generally very difficult to solve, but they admit analytical solutions in certain special cases.
One of the simplest is that of a central mass \(M\) described with spherical coordinates \((t, r, \theta, \varphi)\) and in the presence of spherical symmetry. Here I present a sketch of the procedure used to derive the metric, following what is done in \textcite[section 7]{Carroll:1997ar}.

We impose the condition that the stress energy tensor be identically zero for radii greater than a certain (arbitrarily small) radius, \(r > r_c\).

Then, we can write down the most general possible spherically symmetric metric, which turns out to be \cite[eq. 7.13]{Carroll:1997ar}:

\begin{equation}
  \dd{s} ^2 = -e^{2 \alpha(r, t)} \dd{t}^2 + e^{2 \beta(r, t)} \dd{r}^2
  + r^2 \qty(\dd{\theta}^2 + \sin^2\theta \dd{\varphi})
\end{equation}
and plug this into the equations \(G_{\mu\nu} = 0 \implies R_{\mu\nu} = 0 \) for all \(r>r_c\).
We get the result that the metric possesses a timelike Killing vector field which is orthogonal to a family of hypersurfaces \(t = \const\): therefore it is \emph{static}, unchanging with time.
\(\alpha\) and \(\beta\) only depend on \(r\), and we can show that

\begin{equation}
  e^{2 \alpha(r)} = e^{-2 \beta(r)} = \qty(1 + \frac{C}{r})
\end{equation}
for some \(C\). By continuity with the weak-field limit, for which we have the newtonian gravitational field \(\Phi = -M/r\) and \(g_{00} = - (1 + 2 \Phi)\), one sets \(C = -2M\).
Keeping the notation \(\Phi = -M/r\) we have:

\begin{equation}
    \dd{s}^2 = -(1+2\Phi)\dd{t}^2 + \frac{1}{1+2\Phi} \dd{r}^2
    + r^2 \qty(\dd{\theta}^2 + \sin^2\theta \dd{\varphi}) \label{eq:schwartzshild-line-element}
\end{equation}
or, equivalently,

\begin{equation}
    g_{\mu\nu} =  \diag\qty(-(1+2\Phi),\, \frac{1}{1+2\Phi},\, r^2,\, r^2 \sin ^2 \theta ) \,.
\end{equation}

We can see that it approaches the spherical-coordinates flat metric $\eta_{\mu\nu} ^{\prime} = \diag\qty(-1, 1, r^2, r^2 \sin ^2 \theta)$ both in the limit $M\rightarrow 0$ and the limit \(r \rightarrow \infty\). Its determinant is $g = -r^4 \sin^2 \theta$, so \(\sqrt{-g} = r^2 \sin \theta \).


\end{document}
