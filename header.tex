% Permette di scrivere tutti i caratteri unicode senza formule strane
\usepackage[utf8]{inputenc}

\usepackage[
backend=biber,
style=alphabetic,
sorting=ynt
]{biblatex}

\addbibresource{bibliography.bib}

\usepackage{textcomp}
\usepackage{multirow}
\usepackage{float}
%\usepackage[caption = false]{subfig}

%Per andare a capo fra pagine con le tabelle
\usepackage{longtable}

% %Per inserire codice
% \usepackage{listings}

%Cose di matematica
\usepackage{mathtools}
\usepackage{commath}

%Per fare l'1 bbold della matrice identità
\usepackage{bbold}
\usepackage{xcolor}

%Utilissimo per il formalismo della meccanica quantistica, matrici,
%parentesi che si ridimensionano...
\usepackage{physics}


%Ridimensiona i margini
\usepackage[margin=1.8cm]{geometry}

%Etichette per le sotto-immagini
\usepackage{caption}
\usepackage{subcaption}

%Blocchi colorati
\usepackage{mdframed}
\allowdisplaybreaks

\usepackage{import}
\usepackage{xifthen}
\usepackage[final]{pdfpages}
\usepackage{transparent}

\newcommand{\incfig}[1]{%
    \def\svgwidth{0.7\columnwidth}
    \import{./figures/}{#1.pdf_tex}
}

\numberwithin{equation}{section}
\usepackage[perpage]{footmisc}


\usepackage{cprotect}

\usepackage{tikz-cd}
\usepackage{amsmath}
\usepackage{amsfonts}
\usepackage{amssymb}
\usepackage{amsthm}
\usepackage{graphicx}
\usepackage{mathrsfs}
%\usepackage[colorinlistoftodos]{todonotes}
\usepackage[colorlinks=true, allcolors=blue, breaklinks=true]{hyperref}
\usepackage{etoolbox}
\appto\UrlBreaks{\do\-}
\usepackage{nameref}
% \usepackage[version=4]{mhchem}
\usepackage{siunitx}
\sisetup{separate-uncertainty=true}
\DeclareSIUnit\erg{erg}
\usepackage{cancel}

%%
%Comandi per le frazioni semi-inline
%%

\usepackage{nicefrac}
\usepackage{ifthen}
\let\oldfrac\frac
\renewcommand{\frac}[3][d]{\ifthenelse{\equal{#1}{d}}{\oldfrac{#2}{#3}}{\nicefrac{#2}{#3}}}

%Font bello per la matematica
\usepackage[sc]{mathpazo}
\linespread{1.05}         % Palladio needs more leading (space between lines)
\usepackage[T1]{fontenc}

\usepackage{tocloft}

%\usepackage{minitoc}


\renewcommand{\H}{\mathcal{H}}
\newcommand{\C}{\mathbb{C}}
\newcommand{\R}{\mathbb{R}}
\newcommand{\N}{\mathbb{N}}
\newcommand{\id}{\mathbb{1}}
\newcommand{\Z}{\mathbb{Z}}
\renewcommand{\P}{\mathbb{P}}
\let\Tr\undefined
\DeclareMathOperator*{\Tr}{Tr}
\DeclareMathOperator{\supp}{supp}
\DeclareMathOperator{\diag}{diag}
\newcommand{\Lagr}{\mathcal{L}}
\DeclareMathOperator{\const}{const}
\DeclareMathOperator{\sign}{sign}

\renewcommand{\var}{\text{var}}
\newcommand{\defeq}{\ensuremath{\stackrel{\text{def}}{=}}}

\newcommand\mybox[1]{%
  \fbox{\begin{minipage}{0.9\textwidth}#1\end{minipage}}}

\newtheorem{claim}{Claim}[section]

\newenvironment{greenbox}{\begin{mdframed}[hidealllines=true,backgroundcolor=green!20,innerleftmargin=3pt,innerrightmargin=3pt]}{\end{mdframed}}

\newenvironment{bluebox}{\begin{mdframed}[hidealllines=true,backgroundcolor=blue!10,innerleftmargin=3pt,innerrightmargin=3pt]}{\end{mdframed}}

\newcommand{\hlc}[2]{%
  \colorbox{#1!40}{$\displaystyle#2$}}


%%
% Definizione dello stile per l'inclusione di codice
%%

% \definecolor{codegreen}{rgb}{0,0.6,0}
% \definecolor{codegray}{rgb}{0.5,0.5,0.5}
% \definecolor{codepurple}{rgb}{0.58,0,0.82}
% \definecolor{backcolour}{rgb}{0.95,0.95,0.92}

% \lstdefinestyle{mystyle}{
%     backgroundcolor=\color{backcolour},
%     commentstyle=\color{codegreen},
%     keywordstyle=\color{magenta},
%     numberstyle=\tiny\color{codegray},
%     stringstyle=\color{codepurple},
%     basicstyle=\ttfamily\footnotesize,
%     breakatwhitespace=false,
%     breaklines=true,
%     captionpos=b,
%     keepspaces=true,
%     numbers=left,
%     numbersep=5pt,
%     showspaces=false,
%     showstringspaces=false,
%     showtabs=false,
%     tabsize=2
% }

%\lstset{style=mystyle}

%% SELECTIVE COMPILING
% \usepackage{xstring}
% \newcommand{\thisuser}{DM}
%
% \newcommand{\selective}[2]{
% \IfSubStr{#1}{\thisuser}{#2}{}%
% }

\usepackage{imakeidx}                      % Indice analitico
\makeindex[intoc]                          % L'indice analitico va nell'indice generale
%\indexsetup{firstpagestyle=empty}          % Niente numero di pagina nella prima dell'indice analitico

%\usepackage[italian]{varioref}             % riferimenti completi della pagina
