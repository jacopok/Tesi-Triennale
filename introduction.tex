\documentclass[main.tex]{subfiles}
\begin{document}

When studying a fluid dynamical problem often one wishes to approximate its behaviour as non-relativistic, when \(v \ll c\).
The problem of (spherical) accretion onto black holes allows for neither, since the velocities from radial infinity to the horizon span both cases.

In section \ref{sec:general-relativity} I will briefly recall the basics of special relativity, differential geometry and general relativity in order to introduce the Schwarzschild metric, which is the geometrical background on which the spherical accretion problem is studied.

In section \ref{sec:fluid-dynamics} I will compare the equations of motion of a non-relativistic fluid to those of a relativistic fluid.

I will use the formalism first introduced by \textcite[]{Eckart:1940} for the decomposition the stress energy tensor, and following \textcite[]{Taub:1978} distinguish in the Euler equation --- the spatial projection of the conservation equations --- the relativistic forces acting on the fluid due to viscosity and to heat transfer.

I will then give a proof of the relativistic Second Principle of thermodynamics, which will justify the definition of an ideal fluid.

I will then derive the equations which govern spherical accretion in the adiabatic case.

In section \ref{sec:radiative-effects} I will introduce and apply the PSTF moments formalism by \textcite[]{Thorne:1981feb} to the spherical accretion problem, still considering the fluid as ideal but introducing energy transfer terms due to radiation.

\end{document}
