\documentclass[a4paper, 11pt]{article}

\usepackage{subfiles}

% Permette di scrivere tutti i caratteri unicode senza formule strane
\usepackage[utf8]{inputenc}

\usepackage{textcomp}
\usepackage{float}
%\usepackage[caption = false]{subfig}


%Cose di matematica
\usepackage{mathtools}
\usepackage{commath}

%Per fare l'1 bbold della matrice identità
\usepackage{bbold}
\usepackage{xcolor}

%Utilissimo per il formalismo della meccanica quantistica, matrici,
%parentesi che si ridimensionano...
\usepackage{physics}

%Etichette per le sotto-immagini
\usepackage{caption}
\usepackage{subcaption}

\usepackage{import}
\usepackage{xifthen}
\usepackage[final]{pdfpages}
\usepackage{transparent}

\newcommand{\incfig}[1]{%
    \def\svgwidth{1\columnwidth}
    \import{./figures/}{#1.pdf_tex}
}

\numberwithin{equation}{section}
\usepackage[perpage]{footmisc}


\usepackage{cprotect}

\usepackage{tikz-cd}
\usepackage{amsmath}
\usepackage{amsfonts}
\usepackage{amssymb}
\usepackage{amsthm}
\usepackage{graphicx}
\usepackage{mathrsfs}
%\usepackage[colorinlistoftodos]{todonotes}
%\usepackage[colorlinks=true, allcolors=blue, breaklinks=true]{hyperref}
\usepackage{etoolbox}
\appto\UrlBreaks{\do\-}
\usepackage{nameref}
% \usepackage[version=4]{mhchem}
\usepackage{siunitx}
\sisetup{separate-uncertainty=true}
\DeclareSIUnit\erg{erg}
\usepackage{cancel}

\usepackage{tensor}

%%
%Comandi per le frazioni semi-inline
%%

\usepackage{nicefrac}
\usepackage{ifthen}
\let\oldfrac\frac
\renewcommand{\frac}[3][d]{\ifthenelse{\equal{#1}{d}}{\oldfrac{#2}{#3}}{\nicefrac{#2}{#3}}}

%Font bello per la matematica
\usepackage[sc]{mathpazo}
\linespread{1.10}         % Palladio needs more leading (space between lines)
\usepackage[T1]{fontenc}

%\usepackage{minitoc}


\renewcommand{\H}{\mathcal{H}}
\newcommand{\C}{\mathbb{C}}
\newcommand{\R}{\mathbb{R}}
\newcommand{\N}{\mathbb{N}}
\newcommand{\id}{\mathbb{1}}
\newcommand{\Z}{\mathbb{Z}}
\renewcommand{\P}{\mathbb{P}}
\let\Tr\undefined
\DeclareMathOperator*{\Tr}{Tr}
\DeclareMathOperator*{\Lie}{\mathscr L}
\DeclareMathOperator{\supp}{supp}
\DeclareMathOperator{\diag}{diag}
\newcommand{\Lagr}{\mathcal{L}}
\DeclareMathOperator{\const}{const}
\DeclareMathOperator{\sign}{sign}
\DeclareMathOperator{\spn}{span}

\renewcommand{\var}{\text{var}}
\newcommand{\defeq}{\ensuremath{\stackrel{\text{def}}{=}}}

\newcommand\mybox[1]{%
  \fbox{\begin{minipage}{0.9\textwidth}#1\end{minipage}}}

\newtheorem{claim}{Claim}[section]

\newenvironment{greenbox}{\begin{mdframed}[hidealllines=true,backgroundcolor=green!20,innerleftmargin=3pt,innerrightmargin=3pt]}{\end{mdframed}}

\newenvironment{bluebox}{\begin{mdframed}[hidealllines=true,backgroundcolor=blue!10,innerleftmargin=3pt,innerrightmargin=3pt]}{\end{mdframed}}

\newcommand{\hlc}[2]{%
  \colorbox{#1!40}{$\displaystyle#2$}}


%%
% Definizione dello stile per l'inclusione di codice
%%

% \definecolor{codegreen}{rgb}{0,0.6,0}
% \definecolor{codegray}{rgb}{0.5,0.5,0.5}
% \definecolor{codepurple}{rgb}{0.58,0,0.82}
% \definecolor{backcolour}{rgb}{0.95,0.95,0.92}

% \lstdefinestyle{mystyle}{
%     backgroundcolor=\color{backcolour},
%     commentstyle=\color{codegreen},
%     keywordstyle=\color{magenta},
%     numberstyle=\tiny\color{codegray},
%     stringstyle=\color{codepurple},
%     basicstyle=\ttfamily\footnotesize,
%     breakatwhitespace=false,
%     breaklines=true,
%     captionpos=b,
%     keepspaces=true,
%     numbers=left,
%     numbersep=5pt,
%     showspaces=false,
%     showstringspaces=false,
%     showtabs=false,
%     tabsize=2
% }

%\lstset{style=mystyle}

%% SELECTIVE COMPILING
% \usepackage{xstring}
% \newcommand{\thisuser}{DM}
%
% \newcommand{\selective}[2]{
% \IfSubStr{#1}{\thisuser}{#2}{}%
% }

% \usepackage{imakeidx}                      % Indice analitico
% \makeindex[intoc]                          % L'indice analitico va nell'indice generale
%\indexsetup{firstpagestyle=empty}          % Niente numero di pagina nella prima dell'indice analitico

%\usepackage[italian]{varioref}             % riferimenti completi della pagina


\title{Relativistic non-ideal flows}
\author{Jacopo Tissino}
\date{2019}

\begin{document}

\includepdf[pages=1]{Frontespizio_Laurea.pdf}

%\renewcommand{\baselinestretch}{1.2}

\begin{abstract}
% After reviewing the basic concepts of general-relativistic fluid mechanics, I will focus on the treatment of non-ideal
% (viscous, thermo-conducting) flows. An application of non-ideal relativistic flows to spherical accretion onto black holes
% (generalized Bondi accretion) will be also discussed.
The problem of stationary, spherically symmetric accretion onto a Schwarzschild black hole is discussed here with the use of a formalism which is completely consistent with Einstein's General theory of Relativity.

The transfer of heat is a significant part of this process, however treating it without approximations has proven difficult.
First, we explore the adiabatic case; then we consider a more general case by assuming that all the heat tranfer happens through electromagnetic radiation.
For the latter we apply a formalism which, roughly speaking, while still being relativistic allows for the decomposition of the radiation into its first moments: energy density and flux.

The numerical analysis of the differential equations the problem can be reduced to shows a bimodal behaviour: a branch of solutions has a much higher efficiency (ratio of luminosity to accretion rate) than another.

In order to treat this problem, first the formalism of general relativity is briefly recalled; then we treat the basics of the relativistic formulation of the fluid dynamical equations, including the relativistic version of the Second Principle of thermodynamics.
\end{abstract}

\setcounter{tocdepth}{2}
% \cftsetindents{paragraph}{5em}{0in}
% The table of contents will only include up to the subsection level in the final document, it is just convenient for drafts to be able to see the paragraph structure.
\tableofcontents

\section{Introduction} \label{sec:introduction}
\subfile{introduction.tex}

\section{Notational preface} \label{sec:notational-preface}
\subfile{conventions.tex}

\section{Relativity} \label{sec:general-relativity}
\subfile{general-relativity-basics.tex}

\section{Fluid dynamics} \label{sec:fluid-dynamics}
\subfile{relativistic-fluid-mechanics.tex}

\section{Radiative effects in spherical accretion} \label{sec:radiative-effects}
\subfile{radiative-effects.tex}

% \section{Extra sections, possibly to remove} \label{sec:formulas}
% \subfile{formulas.tex}

\section{Conclusions} \label{sec:conclusions}
\subfile{conclusions.tex}

\printbibliography[title={Bibliography}]

%\printindex

\end{document}
