\pdfminorversion=4
\documentclass{beamer}
% Permette di scrivere tutti i caratteri unicode senza formule strane
\usepackage[utf8]{inputenc}

\usepackage{textcomp}
\usepackage{float}
%\usepackage[caption = false]{subfig}


%Cose di matematica
\usepackage{mathtools}
\usepackage{commath}

%Per fare l'1 bbold della matrice identità
\usepackage{bbold}
\usepackage{xcolor}

%Utilissimo per il formalismo della meccanica quantistica, matrici,
%parentesi che si ridimensionano...
\usepackage{physics}

%Etichette per le sotto-immagini
\usepackage{caption}
\usepackage{subcaption}

\usepackage{import}
\usepackage{xifthen}
\usepackage[final]{pdfpages}
\usepackage{transparent}

\newcommand{\incfig}[1]{%
    \def\svgwidth{1\columnwidth}
    \import{./figures/}{#1.pdf_tex}
}

\numberwithin{equation}{section}
\usepackage[perpage]{footmisc}


\usepackage{cprotect}

\usepackage{tikz-cd}
\usepackage{amsmath}
\usepackage{amsfonts}
\usepackage{amssymb}
\usepackage{amsthm}
\usepackage{graphicx}
\usepackage{mathrsfs}
%\usepackage[colorinlistoftodos]{todonotes}
%\usepackage[colorlinks=true, allcolors=blue, breaklinks=true]{hyperref}
\usepackage{etoolbox}
\appto\UrlBreaks{\do\-}
\usepackage{nameref}
% \usepackage[version=4]{mhchem}
\usepackage{siunitx}
\sisetup{separate-uncertainty=true}
\DeclareSIUnit\erg{erg}
\usepackage{cancel}

\usepackage{tensor}

%%
%Comandi per le frazioni semi-inline
%%

\usepackage{nicefrac}
\usepackage{ifthen}
\let\oldfrac\frac
\renewcommand{\frac}[3][d]{\ifthenelse{\equal{#1}{d}}{\oldfrac{#2}{#3}}{\nicefrac{#2}{#3}}}

%Font bello per la matematica
\usepackage[sc]{mathpazo}
\linespread{1.10}         % Palladio needs more leading (space between lines)
\usepackage[T1]{fontenc}

%\usepackage{minitoc}


\renewcommand{\H}{\mathcal{H}}
\newcommand{\C}{\mathbb{C}}
\newcommand{\R}{\mathbb{R}}
\newcommand{\N}{\mathbb{N}}
\newcommand{\id}{\mathbb{1}}
\newcommand{\Z}{\mathbb{Z}}
\renewcommand{\P}{\mathbb{P}}
\let\Tr\undefined
\DeclareMathOperator*{\Tr}{Tr}
\DeclareMathOperator*{\Lie}{\mathscr L}
\DeclareMathOperator{\supp}{supp}
\DeclareMathOperator{\diag}{diag}
\newcommand{\Lagr}{\mathcal{L}}
\DeclareMathOperator{\const}{const}
\DeclareMathOperator{\sign}{sign}
\DeclareMathOperator{\spn}{span}

\renewcommand{\var}{\text{var}}
\newcommand{\defeq}{\ensuremath{\stackrel{\text{def}}{=}}}

\newcommand\mybox[1]{%
  \fbox{\begin{minipage}{0.9\textwidth}#1\end{minipage}}}

\newtheorem{claim}{Claim}[section]

\newenvironment{greenbox}{\begin{mdframed}[hidealllines=true,backgroundcolor=green!20,innerleftmargin=3pt,innerrightmargin=3pt]}{\end{mdframed}}

\newenvironment{bluebox}{\begin{mdframed}[hidealllines=true,backgroundcolor=blue!10,innerleftmargin=3pt,innerrightmargin=3pt]}{\end{mdframed}}

\newcommand{\hlc}[2]{%
  \colorbox{#1!40}{$\displaystyle#2$}}


%%
% Definizione dello stile per l'inclusione di codice
%%

% \definecolor{codegreen}{rgb}{0,0.6,0}
% \definecolor{codegray}{rgb}{0.5,0.5,0.5}
% \definecolor{codepurple}{rgb}{0.58,0,0.82}
% \definecolor{backcolour}{rgb}{0.95,0.95,0.92}

% \lstdefinestyle{mystyle}{
%     backgroundcolor=\color{backcolour},
%     commentstyle=\color{codegreen},
%     keywordstyle=\color{magenta},
%     numberstyle=\tiny\color{codegray},
%     stringstyle=\color{codepurple},
%     basicstyle=\ttfamily\footnotesize,
%     breakatwhitespace=false,
%     breaklines=true,
%     captionpos=b,
%     keepspaces=true,
%     numbers=left,
%     numbersep=5pt,
%     showspaces=false,
%     showstringspaces=false,
%     showtabs=false,
%     tabsize=2
% }

%\lstset{style=mystyle}

%% SELECTIVE COMPILING
% \usepackage{xstring}
% \newcommand{\thisuser}{DM}
%
% \newcommand{\selective}[2]{
% \IfSubStr{#1}{\thisuser}{#2}{}%
% }

% \usepackage{imakeidx}                      % Indice analitico
% \makeindex[intoc]                          % L'indice analitico va nell'indice generale
%\indexsetup{firstpagestyle=empty}          % Niente numero di pagina nella prima dell'indice analitico

%\usepackage[italian]{varioref}             % riferimenti completi della pagina

\usetheme{Rochester}
% \usepackage[scaled]{helvet} % ss

\title{Relativistic Non-Ideal Flows}
\author{Laureando: Jacopo Tissino \\
    Relatore: prof.\ Roberto Turolla}
\date{24/09/2019}

\begin{document}

\frame{\titlepage}

\begin{frame}
    \frametitle{The Schwarzschild metric}

    The Schwarzschild metric is given by:

    \begin{equation*}
    \dd{s}^2 = -\qty(1-\frac{2M}{r})\dd{t}^2 + \frac{1}{1-\frac{2M}{r}} \dd{r}^2
    + r^2 \qty(\dd{\theta}^2 + \sin^2\theta \dd{\varphi})\,,
    \end{equation*}
    %
    while the flat metric is:
    %
    \begin{equation*}
        \dd{s}^2 = -\dd{t}^2 + \dd{r}^2 + r^2 \qty(\dd{\theta}^2 + \sin^2\theta \dd{\varphi})\,.
    \end{equation*}
    %
    where \(c = G = 1\).

    % The radial coordinate \(r\) can be defined with the 2-sphere's area: \(A \overset{!}{=} 4 \pi r^2\) since the angular metric is flat.

    % The singularity at \(r = 2M \) is not physical: there are other coordinates in which it disappears and the physical behaviour is revealed, objects \emph{can} actually reach the horizon in finite time.
\end{frame}

% \begin{frame}
%     \frametitle{The Schwarzschild metric}
%     \begin{figure}
%         \includegraphics[width=\textwidth]{figures/Schwarzschild-Space}
%     \end{figure}
%
%     {\tiny Image credit: \url{https://www.physicsforums.com/insights/schwarzschild-metric-part-1-gps-satellites/}.}
%
%     % Circles are spaced further apart on the manifold than their radii's difference
% \end{frame}

\begin{frame}
    \frametitle{Tetrads: the Local Rest Frame}

    % A tetrad is a set of vectors \(V^\mu _{(\alpha)}\) which are Fermi-Walker transported and satisfy:
    % %
    The \emph{comoving} reference frame for spherically symmetric accretion is:
    %
    \begin{align*}
      \hat{e}_t &= u^\mu  \\
      \hat{e}_r &= a^\mu / \sqrt{a^\nu a_\nu}   \\
      \hat{e}_\theta &= \qty(0,0,1/r,0)  \\
      \hat{e}_t &= \qty(0,0,0,1/(r \sin\theta))\,.
    \end{align*}

    This basis satisfies:
    %
    \begin{equation*}
        g_{\mu\nu} \hat{e}^\mu _{(\alpha)} \hat{e}^\nu _{(\beta)} = \eta_{(\alpha) (\beta)}\,.
    \end{equation*}
    %
    % If we assume spherical symmetry and stationarity, we can determine the whole tetrad if we have the normalized radius \(r/2M\) and the velocity \(v\).


\end{frame}

\begin{frame}
    \frametitle{The Local Rest Frame}
        \includegraphics[width=\textwidth]{figures/low_speed}
\end{frame}

\begin{frame}
    \frametitle{The Local Rest Frame}
        \includegraphics[width=\textwidth]{figures/high_speed}
\end{frame}

\begin{frame}
    \frametitle{The stress-energy tensor}

    The component \(T^{\mu\nu}\) is the flux of \(\mu\)-th component of the four-momentum \(p^\mu\) through a surface of constant coordinate \(x^\nu\).

    For an ideal fluid (\(\eta = \xi = \kappa = 0\)) in the Local Rest Frame:

    \begin{equation*}
        T^{\mu\nu}_{\text{ideal fluid}} =
        \begin{bmatrix}
        \rho   &   &   & 0  \\
           & p  &   &  \\
           &   & p  &  \\
          0 &   &   & p
       \end{bmatrix}_{\text{fid}} \,,
    \end{equation*}
    %
    where \(\rho = \rho_0 (1 + \varepsilon)\).
\end{frame}

\begin{frame}
    \frametitle{The radiation moments}

    \begin{subequations}
    \begin{align*}
      w_0 &= \int I \dd{\Omega} & \text{radiation energy density} \\
      w_1 &= \int I \cos \theta \dd{\Omega} & \text{radiation energy flux} \\
      w_2 &= \int I \qty(\cos^2 \theta - \frac{1}{3}) \dd{\Omega} & \text{radiation shear stress.}
    \end{align*}
    \end{subequations}
\end{frame}

\begin{frame}
    \frametitle{The radiation moments}

    \incfig{figures/moments}
\end{frame}

\begin{frame}
    \frametitle{The full stress-energy tensor}

    \begin{equation*}
    T^{\mu\nu} =
    T^{\mu\nu}_{\text{ideal fluid}} +
    \begin{bmatrix}
    w_0   & w_1  & 0  & 0 \\
    w_1   & \frac{1}{3}w_0 + w_2  &  0  & 0 \\
      0 & 0  &  \frac{1}{3}w_0 -\frac{1}{2}w_2 & 0 \\
      0 & 0  &  0 & \frac{1}{3}w_0 -\frac{1}{2}w_2
    \end{bmatrix} _{\text{fid}}
    \end{equation*}

    The equations to solve are:

    \begin{subequations}
    \begin{align*}
      \nabla_\mu T^{\mu\nu} = 0 && \text{2 equations} \\
      \nabla_\mu \qty(\rho_0 u^\mu) = 0&& \text{1 equation} \\
      \text{moments of the photon transfer equation} && \text{2 equations.}
    \end{align*}
    \end{subequations}
\end{frame}

\begin{frame}
    \frametitle{Critical point}

    There is a critical point in the Euler equation at \(v = v_s\):
        \begin{equation*}
        (v^2 - v_s^2) \frac{(yv) ^{\prime}}{yv} - 2 v_s^2 + \frac{M}{y^2 r}
        = -\frac{r}{yv (p + \rho)} \qty((\Gamma-1)s_0 + v s_1)\,.
        \end{equation*}
    %
    where:
    \begin{itemize}
      \item \(y\) is the energy at infinity per unit rest mass,
      \item \(v\) is the velocity of the fluid,
      \item \(v_s = \sqrt{\qty(\pdv*{p}{\rho})_s}\) is the adiabatic speed of sound,
      \item \(s_0\) and \(s_1\) are the source moments,
      \item \(\Gamma\) is the local adiabatic exponent.
    \end{itemize}

    % Also, since we assume \(w_2 = f(\tau) w_1\), there is another singularity at the zeros of:
    % %
    % \begin{equation*}
    %     v^2 - v f(\tau) - \frac{1}{3}
    % \end{equation*}
\end{frame}

\begin{frame}
    \frametitle{Accretion efficiency}
    The Eddington luminosity is attained when the radiation pressure on a test electron-proton pair equals the gravitational pull on it:

    \begin{equation*}
        \frac{L_{\text{Edd}}}{M} = \frac{4 \pi c G m_p}{\sigma_T} \approx \num{3.27e4} \frac{L_{\odot}}{M_{\odot}}  \,.
    \end{equation*}

    In the numerical simulations we fix the accretion rate \(\dot{M}\) and calculate the luminosity \(L = 4 \pi r^2 w_1\).

    We can adimensionalize these with the Eddington luminosity \(L_{\text{Edd}}\)  and accretion rate \(\dot{M}_{\text{Edd}} = L_{\text{Edd}} / c^2\):
    we define \(l = L / L_{\text{Edd}}\) and \(\dot{m} = \dot{M} /\dot{M}_{\text{Edd}}\).
\end{frame}

\begin{frame}
    \frametitle{Accretion efficiency}
    \centering
    \includegraphics[width=0.6\textwidth]{../figures/logl-logm}

    {\tiny Image credit: L. Nobili, R. Turolla, L. Zampieri. In: \emph{ApJ} 383 (Dec. 1991).}
\end{frame}

\end{document}
