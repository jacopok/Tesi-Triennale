\documentclass[main.tex]{subfiles}
\begin{document}


% The influence of radiative heat transfer (and of viscosity) on spherically symmetrical stationary accretion

\subsection{Thorne's PSTF moment formalism}

Following \cite{Thorne:1981feb}.

Given any tensor \(A^{\mu_1 \dots \mu_k}\) we can use the tensor \(h^{\mu\nu}\) to project it into the space-like subspace defined by the velocity \(u^\mu\):

\begin{equation}
    A^{\mu_1 \dots \mu_k} \rightarrow \qty(A^{\mu_1 \dots \mu_k})^P
    = \qty(\prod_i h^{\mu_i}_{\nu_i}) A^{\nu_1 \dots \nu_k}
\end{equation}

Then, we can take the symmetric part of any (?) tensor as outlined in \Nameref{sec:notational-preface}:

\begin{equation}
    A^{\mu_1 \dots \mu_k} \rightarrow \qty(A^{\mu_1 \dots \mu_k})^S
    = A^{(\mu_1 \dots \mu_k)}
\end{equation}

We can select the trace-free part of a projected, symmetric tensor by

\begin{equation}
    A^{\mu_1 \dots \mu_k} \rightarrow \qty(A^{\mu_1 \dots \mu_k})^{TF}
    = \sum _{i=0}   ^{\lfloor k/2 \rfloor}
    (-1)^i \frac{k! (2k-2i-1)!!}{(k-2i)! (2k-1)!! (2i)!!}
    h^{(\alpha_1 \alpha_2} \dots h^{\alpha_{2i-1} \alpha_{2i}}
    A^{\alpha_{2i+1} \dots \alpha_k) \beta_1 \dots \beta_i}\,_{\beta_1 \dots \beta_i}
\end{equation}

To see what this is doing, let us consider its action on a rank-two projected tensor: it is just the subtraction of its trace,

\begin{equation}
    A^{\mu\nu} \rightarrow A^{\mu\nu} - \frac{1}{3} h^{\mu\nu} A^{\rho}_\rho
\end{equation}

Now, let us consider all the unit vectors \(n^\mu\) in the space normal to the velocity, which have \(n_\mu u^\mu = 0\) and \(n^\mu n_\mu = 1\). They span a three-dimensional sphere.

If we have a function \(F\colon S^2 \rightarrow \mathbb R\), we can decompose it into harmonics as such:

\begin{equation}
    F(n) = \sum _{k=0}   ^{\infty}
    \mathscr F_{\alpha_1 \dots \alpha_k} \prod_{i=0}^k n^{\alpha_i}
\end{equation}

Where the PSTF moments \(\mathscr F_{\alpha_1 \dots \alpha_k}\) can be computed as

\begin{equation} \label{eq:PSTF-moments-function-decomposition}
    \mathscr F_{\alpha_1 \dots \alpha_k} =
    \frac{(2k+1)!!}{4 \pi k!} \qty(\int F \prod_{i=0}^k n^{\alpha_i}  \dd{\Omega}  )^{TF}
\end{equation}

In particular, the function we will apply this to is the distribution of EM radiation around the BH. So, let us consider a photon, whose trajectory in spacetime is parameterized as \(\gamma(\xi)\), with a choice of \(\xi\) such that the photon's momentum is
\(
p = \dv*{}{\xi}
\).

Now, our observer has a timelike velocity \(u^\mu\). We can find a spacelike vector \(n^\mu\) corresponding to the space-like part of the movement of the photon, or

\begin{equation}
    p^\mu = (- u^\nu p_\nu) (u^\mu + n^\mu)
\end{equation}

It must hold that \(u^\mu u_\mu = -1 \) while \(n^\mu n_\mu = +1 \) in order for \(p^\mu\) to be null-like.
Now, we define a parameter \(l\) which corresponds to the space distance the photon moved through in this frame (this is \emph{not} covariant!)

\begin{equation}
    l = \int  (-u^\nu p_\nu) \dd{\xi}
\end{equation}

now, \(\dv*{}{l} \) is parallel to \(p\) but it has different length, in fact since \(\dv*{l}{\xi} = (-u^\nu p_\nu) \) it is \(\dv*{}{l} = u + n \).

It holds \cite[eq. 2.17]{Thorne:1981feb}, (with the notation from \eqref{eq:covariant-acceleration-decomposition}),  that

\begin{equation}
  \dv{\nu}{l}  = (u^\mu + n^\mu) \nabla_\mu (-p^\nu u_\nu) = - \nu \qty(n_\mu a^\mu  + \frac{\theta}{3} + n_\mu n_\nu \sigma^{\mu\nu})
\end{equation}

We want to quantify the number density of photons in relation to their momentum. We assume the radiation in unpolarized, therefore for each unit \(h^3\) cell in phase space there can be 2 photons: so we denote the distribution function of the photons as \(2N (x^\mu, p^\mu)\).

It is known that the volume element \(\dd{V}_p = \dd[3]{p} / p^0 \) is Lorentz invariant (see \cite[box 22.5]{MisnerThorneWheeler:1973}).
We can write this using the photons' frequency \(\nu = - p^\mu u_\mu / h\) as \(\dd{V}_p = \nu \dd[]{\Omega} \dd{\nu} \).

Let us define the \emph{specific radiative intensity} as

\begin{equation}
  I_\nu = \frac{\delta E}{\delta A \delta  t \delta \nu \delta \Omega}
  = \frac{h \nu \delta N}{\delta A \delta  t \delta \nu \delta \Omega}
\end{equation}

where \(\delta A\) denotes an infinitesimal area the photons are coming through, \(\delta t\) an infinitesimal time, \(\delta \nu\) an infinitesimal photon frequency, \(\delta \Omega\) an infinitesimal solid angle.

Then, \cite[figure 22.2]{MisnerThorneWheeler:1973} the number density of photons in phase space is

\begin{equation}
  2N(x^\mu, p^\mu) = \frac{\delta N}{V_x V_p} =  \frac{\delta N}{h^3 \nu^2\delta A \delta  t \delta \nu \delta \Omega} = \frac{1}{h^4 \nu^3} I_\nu
\end{equation}

therefore \(I_\nu = 2 N \nu^3 h^4\).

\begin{greenbox}
  This is from \cite[figure 22.2]{MisnerThorneWheeler:1973}, but it seems to conflict with \cite[equation 6.2]{Thorne:1981feb}!
\end{greenbox}

Now, we want to describe the variation of the occupation number \(N\) with respect to the photons' trajectories' parameter \(l\). We encapsulate all possible effects into a source term \(\mathfrak S\):

\begin{equation}
    \mathfrak S \defeq \dv{}{l} 2N(x^\mu, p^\mu) =
    2 \qty(\pdv{N}{x^\mu} \dv{x^\mu}{l} + \pdv{N}{p^i} \dv{p^i}{l}  )
\end{equation}

since the occupation number can be thought of as just a function of the spatial components of the momentum.

\begin{greenbox}
  Why so? Surely \(N = N(\nu) = N(p^0)\) also!
\end{greenbox}

Since \(\dv*{}{l} = (n^\mu + u^\mu) \partial_\mu\) and the covariant derivative of \(p^j\) is zero, we can compute

\begin{equation}
  \dv{p^j}{l} = (n^\mu + u^\mu ) \nabla_\mu p^j - \Gamma ^j _{\alpha \beta} p^\alpha (u^\beta + n^\beta)
  = - \Gamma ^j _{\alpha \beta} p^\alpha (u^\beta + n^\beta)
\end{equation}

where the covariant derivative term vanishes since the photon's trajectory is a geodesic.

\paragraph{Moments' definitions}

In units where \(c=h=1\),

\begin{subequations}
\begin{align}
   M _\nu ^{A_k}
   &\defeq \int 2N \frac{\delta (\nu - (-p^\nu u_\nu))}{\nu^{k-2}} \prod_i^k p^{\alpha_i} \dd{V_p} \\
   &= \int \qty(2N \nu^3) \frac{1}{\nu} \delta (\nu +p^\nu u_\nu) \prod_i^k \qty(\frac{p^{\alpha_i}}{\nu}) \qty(\nu \dd[]{\Omega} \dd[]{\nu})  \\
   &= \int  I_\nu \prod_i^k \qty(n^{\alpha_i} + u^{\alpha_i}) \dd[]{\Omega} \label{eq:simplified-moment-definition}
\end{align}
\end{subequations}

This is a general procedure we can use to associate a function \(f\) (in this case we started with \(2N\)) with \(\nu^3\) times the integral \eqref{eq:simplified-moment-definition} (where one might substitute \(I _\nu\) with the function \(f\)).
We need it for the source moments:

\begin{equation}
   S_\nu ^{A_k} = \nu^3 \int S \mathfrak S \prod_i^k (n^{\alpha_i} + u^{\alpha_i}) \dd[]{\Omega}
\end{equation}

\paragraph{Redshift-adapted version}

\textcite[]{Thorne:1981feb} also defines a redshift-adapted version of the moments' definition: if \(R\) is a universal redshift functions, such that \(R (p^\nu u_\nu)\) is conserved along every photon geodesic \(p^\mu \nabla_\mu p^\nu = 0\), that is, \(R\) allows us to calculate the redshift between any two points \(A\), \(B\) which are connected by a geodesic as \(\nu_A / \nu_B = R_B / R_A\).

Then, we define \( M_f ^{A_k} =  M_{\nu} ^{A_k} / R\)

\paragraph{Frequency-integrated version}

The definition is:

\begin{equation}
   M ^{A_k} = \int   M^{A_k} _\nu \dd{\nu}
\end{equation}

and the same is applied to the source moments \(S_\nu^{A_k} \rightarrow S^{A_k}\).

Since this includes the radiation intensity from all frequencies, we have direct interpretations for the first moments:

\begin{subequations}
\begin{align}
   M &= \int  I_\nu \dd{\Omega} \dd{\nu}   & \text{energy density of radiation}  \\
   M^\alpha &= \int I_\nu (n^\alpha + u^\alpha)\dd{\Omega} \dd{\nu}   & (M^0, M^i) = \text{(energy density of radiation, energy flux)}  \\
   M^{\alpha\beta} &= \int I (n^\alpha + u^\alpha)(n^\beta + u^\beta)\dd{\Omega} \dd{\nu}   & \text{stress-energy tensor of radiation}
\end{align}
\end{subequations}

\paragraph{The moment equations}

These can be derived from the transport equation, see \cite[3.14]{Thorne:1981feb}. I present them only in the grey (frequency-integrated) case:

\begin{equation} \label{eq:grey-moment-equations}
  \nabla_\beta M^{A_k \beta} - (k-1) M^{A_k \beta \gamma} (\nabla_ \gamma u_\beta)= S^{A_k}
\end{equation}

Also, the moments (\(M^{A_k}\), but also \(M^{A_k}_\nu\) and \(M^{A_k}_f\)) satisfy the following:

\begin{subequations}
\begin{align}
  M^{A_k \beta}\,_\beta &= 0 \\
  u_\beta M^{A_k \beta} &= -M^{A_k} \\
  h_{\beta \gamma} M^{A_k \beta \gamma} &= M^{A_k}
\end{align}
\end{subequations}

So, the \(k\)-th moment contains all the information about the \(l\)-th moments with \(l\leq k\); also, to get lower-order moments we take partial traces onto space- and time-like subspaces: therefore the unique information to the \(k\)-th moment, which is not redundantly expressed in lower-order moments, is in its PSTF part:

\begin{equation}
  \mathscr M ^{A_k} = \qty(M^{A_k}) ^{PSTF}
\end{equation}

The same can be applied to \(M^{A_k}_\nu\) and \(M^{A_k}_f\), to the moment equations \eqref{eq:grey-moment-equations} and to the source moments \(S^{A_k} \rightarrow \mathscr S ^{A_k}\). Since we are taking the projection onto the space-like subspaces, we can simplify the expression of the PSTF moments: all the terms which contain at least a four-velocity vanish, therefore:

\begin{equation} \label{eq:trace-free-moments-definition}
  \mathscr M^{A_k} = \qty(\int I \prod_i n^{\alpha_i} \dd{\Omega})^{TF}
\end{equation}

where \(I = \int I_\nu \dd{\nu}\).
The first \emph{PSTF} moments also have physical interpretations:

\begin{subequations}
\begin{align}
   \mathscr M &= \int  I \dd{\Omega}    & \text{energy density of radiation}  \\
   \mathscr M^\alpha &= \int I n^\alpha\dd{\Omega}  & \text{energy flux of radiation}  \\
   \mathscr M^{\alpha\beta} &= \int I n^\alpha n^\beta \dd{\Omega}   & \text{shears in the stress-energy tensor of radiation}
\end{align}
\end{subequations}

We can write the stress-energy tensor \(T^{\mu\nu} = M^{\mu\nu}\) with the PSTF moments (see \cite[eq. 4.9]{Thorne:1981feb}):

\begin{equation} \label{eq:PSTF-stress-energy-tensor-decomposition}
    T^{\mu\nu} = \mathscr M u^\mu u^\nu + 2 \mathscr M ^{(\mu} u^{\nu)}
    + \mathscr M ^{\mu\nu} + \frac{1}{3} \mathscr M h^{\mu\nu}
\end{equation}

we can compare these to \eqref{eq:components-stress-energy-tensor} to get the following identifications:

\begin{subequations}
\begin{align}
  \mathscr M &= w = \rho  \label{eq:identification-PSTF-stress-energy-tensor-1}\\
  \mathscr M ^\mu &= w^\mu = -\kappa h^\mu _\sigma  \qty(\partial^\sigma T + T a^\sigma) \label{eq:identification-PSTF-stress-energy-tensor-2}\\
  \mathscr M^{\mu\nu} + \frac{1}{3} \mathscr M h^{\mu\nu}
  &= (p - \xi \theta)h^{\mu\nu} - 2 \eta \sigma^{\mu\nu} \label{eq:identification-PSTF-stress-energy-tensor-3}
\end{align}
\end{subequations}

but since the photons' paths are geodesics in this case \(\theta = 0\), so for the components proportional to \(h^{\mu\nu}\) of equation \eqref{eq:identification-PSTF-stress-energy-tensor-3} we just get \(\rho = \frac[i]{1}{3} p\), which is what we expect for the photon gas.
For the traceless part of the equation, we get \(\mathscr M ^{\mu\nu} = -2 \eta \sigma^{\mu\nu}\).

\paragraph{The PSTF moment equations}

We want to express the grey moment equations \eqref{eq:grey-moment-equations} in terms of the PSTF moments. This can be done as follows: an expression can be found for the full moments in terms of the PSTF moments in \cite[eq. 4.10c]{Thorne:1981feb}:

\begin{equation}
  M^{A_k} = \sum_{l=0}^k \sum_{j=0}^{\lfloor \frac{k-l}{2} \rfloor}
  \frac{1}{(2j)!! (k-l-2j)!}  \frac{k!}{l!} \frac{(2l+1)!!}{(2l+1+2j)!!}
  \mathscr M^{(A_l} \prod_{i=l+1}^{l+2j-1} h^{\alpha_i \alpha_{i+1}}
  \prod_{x=l+2j+1}^k u^{\alpha_x)}
\end{equation}

where all the indices of the \(\mathscr M\), \(h\) and \(u\) are meant to be symmetrized.

We insert this into the moment equations and expand, making use of the decomposition of the covariant derivative of the 4-velocity \eqref{eq:covariant-acceleration-decomposition}.

Then, we take the PSTF part of the equations. This yields a very complicated expression, so here I record only the implicit formula \cite[eq. 4.11c]{Thorne:1981feb}:

\begin{equation} \label{eq:PSTF-grey-moment-equations}
  \begin{split}
    &\left( \nabla _\beta \mathscr M ^{A_k \beta} + u^\beta \nabla_\beta \mathscr M ^{A_k}
    + \frac{k}{2k+1} \nabla_{\alpha_k} \mathscr M ^{A_{k-1}}
    - (k-1) \mathscr M ^{A_k \beta \gamma} \sigma_{\beta \gamma} \right.
    - (k-1) \mathscr M ^{A_k \beta} a_\beta
    + \frac{4}{3} \mathscr M ^{A_k} \theta \\
    &+ \frac{5k}{2k+3} \mathscr M ^{A_{k-1} \beta} \sigma_\beta^{\alpha_k}
    - k \mathscr M ^{A_{k-1} \beta} \omega_\beta ^{\alpha_k}
    \left.+ \frac{k (k+3)}{2k+1} \mathscr M ^{A_{k-1}} a^{\alpha_k}
    + \frac{(k-1) k (k+2) }{(2k-1) (2k+1)} \mathscr M ^{A_{k-2}} \sigma^{\alpha_{k-1} \alpha_k} \right)^{PSTF} = \mathscr S ^{A_k}
  \end{split}
\end{equation}

\paragraph{How to recover the intensity}

Once one has solved the PSTF grey moment equations, one can compute the intensity from the moments by comparing \eqref{eq:PSTF-moments-function-decomposition} and \eqref{eq:trace-free-moments-definition}:

\begin{equation}
  I = \sum _{k=0}   ^{\infty} \frac{(2k+1)!!}{4 \pi k!} \mathscr M^{A_k} \prod_{i=1}^k n_{\alpha_i}
\end{equation}

\subsection{Generalized Bondi accretion}

\paragraph{Simplifications under assumptions of symmetry}

Instead of treating the general case as is done in \cite[]{Thorne:1981feb}, we describe the specific choices made under the assumption of spherical symmetry, following \cite[]{ThorneFLammmangZytkow:1981feb}.

In the fiducial frame defined in \eqref{eq:fiducial-frame}  (denoted with a subscript ``fid''), we have the following expressions:

\begin{subequations} \label{eq:spherical-coordinates-simplifications}
\begin{align}
  a^\mu &= (0,\dv*{y}{r},0,0)_{\text{fid}} \\
  \theta &= - \frac{1}{r^2} \dv{}{r} \qty(r^2 v y)  \\
  \sigma_{\mu\nu} &= - \dv{}{r} \qty(\frac{vy}{r}) \frac{2r}{3} \begin{bmatrix}
  0   &   &   &  \\
     &  1 &   &  \\
     &   & -1/2  &  \\
     &   &   & -1/2
  \end{bmatrix} _{\text{fid}}
  = \sigma \begin{bmatrix}
  0   &   &   &  \\
     &  1 &   &  \\
     &   & -1/2  &  \\
     &   &   & -1/2
  \end{bmatrix} _{\text{fid}} \\
  \Gamma_{\theta r \theta} &= \Gamma_{\varphi r \varphi} = \frac{y}{r}
\end{align}
\end{subequations}

Now, we can see that the shear has been heavily simplified. This is a specific case of a general statement about the PSTF moments: in the spherically symmetric case, the \(k\)-th PSTF moment only has one independent component. This is because it satisfies the following identities:

\begin{subequations}
\begin{align}
  \mathscr M ^{A_k} &= 0 \text{ if } A_k \text{ contains an odd number of } \theta \text{s or } \varphi \text{s}  \label{eq:id1-pstf-spherically-symmetric}  \\
  \mathscr M ^{A_k \theta \theta} &= \mathscr M ^{A_k \varphi \varphi} = -\frac{1}{2} \mathscr M ^{A_k rr} \label{eq:id2-pstf-spherically-symmetric}
\end{align}
\end{subequations}

Equation \eqref{eq:id1-pstf-spherically-symmetric}  comes from the fact that an odd number of \(\theta\) or \(\varphi\) indices corresponds to and odd number of unit vectors which are integrated on the sphere (see the definition \eqref{eq:trace-free-moments-definition}): therefore the integrand is odd.

Equation \eqref{eq:id2-pstf-spherically-symmetric} comes from two observations:
first of all, the moments corresponding to indices \(\theta\) and \(\varphi\) respectively must be equal because of spherical symmetry; secondly the moments must be traceless, therefore the sum of the \(\theta \theta\), \(\varphi \varphi\) and \(rr\) moments must be zero (for any pair of indices).

So, with these every \(k\)-th moment is fully determined by the component \(\mathscr M ^{r\dots r}\) (\(k\) \(r\)s): therefore we give it a name: \(w_k\).
This fact is analogous to the statement that the only spherically symmetrical one of the spherical harmonics \(Y_{lm}\) is \(Y_{l0}\), therefore as in that case we have only one independent component for every \(l\).

\paragraph{Legendre polynomials complement}

The \(l\)-th Legendre polynomial is:

\begin{equation} \label{eq:legendre-polynomials}
    P_{l}(x)=\frac{1}{2^{l}} \sum_{k=0}^{\lfloor l / 2\rfloor} \frac{(-1)^{k}(2 l-2 k) !}{k !(l-k) !(l-2 k) !} x^{l-2 k}
\end{equation}

We can see that the coefficient of \(x^l\) is \((2l)! / (2^l (l!)^2)\). We can rewrite this making use of the identities \((2n)! = (2n-1)!! (2n)!!\) and \((2n)!! = 2^n n!\),  as:

\begin{equation}
    \frac{(2l)!}{2^l (l!)^2} = \frac{1}{l!} \frac{(2l)!}{(2l)!!} = \frac{(2l-1)!!}{l!} = \frac{(2l+1)!!}{l! (2l+1)}
\end{equation}

which is equation \cite[eq. 5.7d]{Thorne:1981feb}.

\textcite[eqs. 5.6]{Thorne:1981feb} claims that

\begin{equation} \label{eq:thorne-trace-free-integral-claim}
    \int_{-1}^1 I(\mu) P_k(\mu) \qty(\frac{(2k-1)!!}{l!})^{-1} \dd{\mu} = \qty(\int_{-1}^1  I(\mu) \prod_{i=1}^k n^r \dd{\mu})^{TF}
\end{equation}

where \(n^r\) denotes the radial component of a normal vector in spherical coordinates, \(P_k\) is the \(k\)-th Legendre polynomial and \(\mu = \cos \theta\) where \(\theta\) is the azimuthal coordinate of \(n\). \(I(\mu)\) is a generic function.

\paragraph{The scalar moments} \label{par:scalar-moments}

It can be shown, using the identity \eqref{eq:thorne-trace-free-integral-claim}  that the definition of \(w_k\) we gave is equivalent to

\begin{equation}
    w_k = \int_{-1}^1 I(\cos \theta) P_k(\cos \theta) \qty(\frac{(2k-1)!!}{l!})^{-1} 2 \pi\dd{\cos \theta}
\end{equation}

where \(P_k\) is the \(k\)-th Legendre polynomial \eqref{eq:legendre-polynomials}.
Then the first moments are:

\begin{subequations}
\begin{align}
  w_0 &= \int I \dd{\Omega} & \text{radiation energy density} \\
  w_1 &= \int I \cos \theta \dd{\Omega} & \text{radiation energy flux} \\
  w_2 &= \int I \qty(\cos^2 \theta - \frac{1}{3}) \dd{\Omega} & \text{radiation shear stress}
\end{align}
\end{subequations}

We can explicitly write the stress-energy tensor in terms of the \(w_k\) using \eqref{eq:PSTF-stress-energy-tensor-decomposition}:

\begin{equation} \label{eq:radiation-stress-energy-tensor-fiducial}
    T^{\mu\nu} = \begin{bmatrix}
    w_0   & w_1  & 0  & 0 \\
    w_1   & \frac{1}{3}w_0 + w_2  &  0  & 0 \\
      0 & 0  &  \frac{1}{3}w_0 -\frac{1}{2}w_2 &  \\
      0 & 0  &   & \frac{1}{3}w_0 -\frac{1}{2}w_2
  \end{bmatrix} _{\text{fid}}
\end{equation}

\paragraph{The source moments}

We can get an explicit formula for the source moments \(s_k = \mathscr S ^{r\dots r}\) (\(k\) \(r\)s) with the same procedure which was used in paragraph \nameref{par:scalar-moments}:
we get

\begin{equation}
    s_k = \int_{-1}^1 \dv{I}{l} (\cos \theta) P_k(\cos \theta) \qty(\frac{(2k-1)!!}{l!})^{-1} 2 \pi\dd{\cos \theta}
\end{equation}

where \(\dv*{I}{l} = \int \mathfrak S \nu^3 \dd{\nu}\) is the frequency-integrated  source term in the transfer equation.

\begin{greenbox}
  The derivative \(\dv*{I}{l}\) first appears in \cite[]{Thorne:1981feb} without any indices, then in \cite[section 6]{Thorne:1981feb} it gets the index ``source'', while in \cite[eq. 15]{ThorneFLammmangZytkow:1981feb} it appears with an index ``interaction'' (and the derivative becomes a proper time one).

  It is not clear to me whether these all represent the same thing, or what the indices are trying to say.
\end{greenbox}

\paragraph{The simplified moment equations}

It is possible to write equations \eqref{eq:PSTF-grey-moment-equations} explicitly in terms of the \(w_k\) and of derivatives wrt the fiducial basis: one gets \cite[eq. 5.10c]{Thorne:1981feb}

\begin{equation} \label{eq:scalar-PSTF-moment-equations}
  \begin{split}
    \pdv{w_{k+1}}{\hat{r}} &+[(2-k) a+(k+2) b] w_{k+1}+\pdv{w_k}{\hat{t}}+\left[\frac{4}{3} \theta+\frac{5 k(k+1)}{2(2 k-1)(2 k+3)} \sigma\right] w_{k} + \\
    &+\frac{k^{2}}{(2 k-1)(2 k+1)} \pdv{w_{k-1}}{\hat{r}}+\frac{k^{2}[(k+3) a+(1-k) b]}{(2 k-1)(2 k+1)} w_{k-1} +  \\
    &-\frac{3}{2}(k-1) \sigma w_{k+2}+\frac{3(k-1)^{2} k^{2}(k+2)}{2(2 k-3)(2 k-1)^{2}(2 k+1)} \sigma w_{k-2} = s_k
    \end{split}
\end{equation}

where \(a = \dv*{y}{r} = \sqrt{a^\mu a_\mu}\) is the magnitude of the 4-acceleration, \(b = y/r\) is the extrinsic curvature, \(\theta\) is the expansion velocity, \(\sigma\) is the scalar shear --- the largest eigenvalue of the shear matrix. Explicit expressions for these are found in  \eqref{eq:spherical-coordinates-simplifications}.

\textcite[]{NobiliTurollaZampieri:1991dec} only use the first two of the moment equations, so here is how the expression is simplified for \(k=0,1\):
for \(k=0\) we get:

\begin{equation}
    \pdv{w_{1}}{\hat{r}} +2( a+ b) w_{1}+\pdv{w_0}{\hat{t}}+\frac{4}{3} \theta w_{0} + \frac{3}{2} \sigma w_{2} = s_0
\end{equation}

For \(k=1\) we get:

\begin{equation}
    \pdv{w_{2}}{\hat{r}} + (a+3 b) w_{2}+\pdv{w_1}{\hat{t}}+\left[\frac{4}{3} \theta+ \sigma\right] w_{1} +\frac{1}{3} \pdv{w_{0}}{\hat{r}}+\frac{4 a}{3} w_{0} = s_1
\end{equation}

These have to be simplified further to be used: specifically, they can be expressed with respect to \(r\), \(v\), \(y\).

\paragraph{Simplifying the moment equations further}

% Following \textcite[]{ThorneFLammmangZytkow:1981feb}.

Because of the hypothesis of stationarity, we can express the derivatives in \eqref{eq:scalar-PSTF-moment-equations} as:

\begin{subequations}
\begin{align}
  \pdv{}{\hat{t}} &= \frac{\gamma^2}{y} \pdv{}{t} - yv \pdv{}{r} = - yv \pdv{}{r}  \\
  \pdv{}{\hat{r}} &= -v \frac{\gamma^2}{y} \pdv{}{t} + y \pdv{}{r} = y \pdv{}{r}
\end{align}
\end{subequations}

Now, we can make the scalar PSTF moment equations \eqref{eq:scalar-PSTF-moment-equations} fully explicit: denoting derivation with respect to \(r\) with a prime, we have

\begin{subequations}
\begin{align}
  y w_1^{\prime} + 2 \qty(y^{\prime} + \frac{y}{r}) w_1
  -yv w_0^{\prime}
  - \frac{4}{3} \frac{w_0}{r^2} \qty(r^2 v y)^{\prime}
  - w_2 r \qty(\frac{vy}{r})^{\prime} &= s_0  \\
  y w_2^{\prime} + \qty(y^{\prime} + \frac{3y}{r})w_2
  - yv w_1 ^{\prime} - \frac{4}{3r^2} \qty(r^2 vy)^{\prime} w_1
  - \frac{2r}{3} \qty(\frac{vy}{r})^{\prime} w_1
  + \frac{y}{3} w_0^{\prime} + \frac{4}{3} y^{\prime} w_0
  &= s_1
\end{align}
\end{subequations}

We can simplify these by expanding the derivatives of products \((vyr^2)^{\prime}\) and \((vy/r)^{\prime}\):

\begin{subequations} \label{eq:NTZ91-moment-equations-spherical}
\begin{align}
  y w_1^{\prime} + 2 \qty(y^{\prime} + \frac{y}{r}) w_1
  -yv w_0^{\prime}
  - \frac{4}{3} w_0 \qty((v y)^{\prime} + \frac{2vy}{r})
  - w_2 \qty((vy)^{\prime} - \frac{vy}{r}) &= s_0  \\
  y w_2^{\prime} + \qty(y^{\prime} + \frac{3y}{r})w_2
  - yv w_1 ^{\prime}
  - 2 w_1 \qty((vy)' + \frac{vy}{r})
  + \frac{y}{3} w_0^{\prime} + \frac{4}{3} y^{\prime} w_0
  &= s_1
\end{align}
\end{subequations}

Equations \eqref{eq:NTZ91-moment-equations-spherical} are the ones which appear in  \textcite[eq. 4]{NobiliTurollaZampieri:1991dec} and which are reported in equation \eqref{eq:NTZ91-moment-equations-logarithmic}: the only difference is that their primes denote derivatives with respect to \(\log (r / (2M))\).

\begin{subequations} \label{eq:NTZ91-moment-equations-logarithmic}
    \begin{align}
        w_{1}^{\prime}-v w_{0}^{\prime}
        -v w_{2}\left[\frac{(v y)^{\prime}}{v y}-1\right]
        +2 w_{1}\left(1+\frac{y^{\prime}}{y}\right)
        -\frac{4}{3} v w_{0}\left[\frac{(v y)^{\prime}}{v y}+2\right]&=\frac{r s_{0}}{y} \\
        w_{2}^{\prime}-v w_{1}^{\prime}+\frac{1}{3} w_{0}^{\prime} +w_{2}\left(3+\frac{y^{\prime}}{y}\right)-2 v w_{1}\left[\frac{(v y)^{\prime}}{v y}+1\right]+\frac{4}{3} \frac{y^{\prime}}{y} w_{0} &=\frac{r s_{1}}{y}
    \end{align}
\end{subequations}

\paragraph{Some properties of the accretion variables}

From equation \eqref{eq:mass-conservation-integral} we can find an expression \cite[eq. 18a]{ThorneFLammmangZytkow:1981feb} for \(y\) which only depends on \(r\) and constants:

\begin{equation}
  y = \sqrt{y^2 \qty(1 - v^2 + v^2)}
  = \sqrt{\qty(\frac{y^2}{\gamma^2}) + y^2 v^2}
  = \sqrt{\qty(1 - \frac{2M}{r}) + \qty(\frac{\dot{M}}{4 \pi r^2 \rho_0})^2}
\end{equation}

therefore \(v\) can also be expressed in terms of \(r\) and constants:

\begin{equation}
  v = \frac{\dot M}{4 \pi r^2 \rho_0 y(r)}
\end{equation}

\paragraph{The source term}

The source term can be written \cite[eq. 15]{ThorneFLammmangZytkow:1981feb} as

\begin{equation} \label{eq:source-term}
  s_k = \frac{l! (2l+1)}{(2l+1)!!} \int_{-1}^1 \qty(\dv{I}{\tau})_{\text{interaction}} P_k(\mu) 2 \pi \dd{\mu}
\end{equation}

\begin{greenbox}
  Why is this a proper-time derivative of the intensity instead of a derivative wrt to the photon spatial distance parameter \(l\) as in \cite[]{Thorne:1981feb}?
\end{greenbox}

Also, we can write

\begin{equation} \label{eq:opacity-emissivity-definition}
  s_k = \rho_0 (\varepsilon_k - \kappa_k w_k)
\end{equation}

where \(\rho_0\) is the rest mass density, \(w_k\) are the PSTF scalar moments, \(\varepsilon_k\) is the \(k\)-th moment of emissivity and \(\kappa_k\) is the \(k\)-th moment of opacity of the gas.

Equation \eqref{eq:opacity-emissivity-definition} can be taken to be the definition of \(\varepsilon_k\) and \(\kappa_k\); we do the integral in \eqref{eq:source-term} and get constant terms and terms proportional to \(w_k\), which we split in the two terms on the RHS of \eqref{eq:opacity-emissivity-definition}.

\begin{claim}
    If the emission is isotropic, then the emissivity moments \(\varepsilon_k\) are 0 for \(k \geq 1\).
\end{claim}

Because of this, we just call the one moment \(\varepsilon = \varepsilon_0\).

\begin{proof}
    To show this, it would be enough to see that in the isotropic case

    \begin{equation}
        \qty(\int \dv{I}{l}_{\text{source}} \prod_i n^{\alpha_i} \dd{\Omega})^{TF}
        \propto \qty(\int I \prod_i n^{\alpha_i} \dd{\Omega})^{TF}
    \end{equation}

    for \(k \geq 1\) but not for \(k = 0\).

    \begin{greenbox}
      I do not know how to prove it though.
    \end{greenbox}
\end{proof}

The source moments given in \cite[eq. 6]{NobiliTurollaZampieri:1991dec} are:

\begin{subequations}
\begin{align}
  s_0 &= \rho_0 \qty(\varepsilon - w_0 \qty(\kappa_0 - \kappa_{\text{es}} \frac{4 k_B}{m_e} (T - T_\gamma)))  \\
  s_1 &= - \rho_0 w_1 \kappa_1
\end{align}
\end{subequations}

with

\begin{equation}
    T_\gamma = \frac{1}{4 k_B} \frac{\displaystyle \int_0^\infty h \nu w_0 (r, \nu) \dd{\nu}}{\displaystyle \int_0^\infty w_0 (r, \nu) \dd{\nu}}
\end{equation}
%
% \begin{greenbox}
%   I do not really understand where these come from.
% \end{greenbox}

Further, we use the facts that \(\varepsilon / \kappa_0 = a T^4\) if there is thermodynamic equilibrium and that we have the following expression for the emissivity \(\varepsilon\) in terms of the cooling function \(\Lambda (T)\) ( given in \eqref{eq:cooling-function-NTZ91}):

\begin{equation}
    \varepsilon = \frac{\rho_0 \Lambda (T)}{m_p^2}
\end{equation}

to write the source term \(s_0\) as

\begin{equation}
    s_0 = \frac{\rho_0^2 \Lambda (T)}{m_p^2} \qty(1 + w_0 \qty(\kappa_{\text{es}} \frac{4 k_B}{\varepsilon m_e} (T - T_\gamma) - \frac{1}{aT^4}))
\end{equation}

As for the \(s_1\) term, we model \(\kappa_1\) as

\begin{equation}
  \kappa_1 = \kappa_{\text{es}} + \expval{\kappa_{\text{ff}}}
  = \kappa_{\text{es}} + \SI{6.4e22}{\centi\metre\square\per\gram} \rho_0 T^{-7/2}
\end{equation}

where the second term is the conventional approximation of the \emph{Rosseland mean opacity} computed taking into account only free-free transitions: the definition of the RMO is a harmonic mean of the opacities at every frequency, weighted by the derivatives wrt temperature of the radiation intensities at specific frequencies (which are related to the grey zeroth moment by \(w_0 = \int I_\nu \dd{\nu}\)).

\begin{equation}
  \frac{1}{\expval{\kappa_{\text{ff}}}} =
  \frac{ \displaystyle\int _{0}   ^{\infty}  \dv{I_\nu}{T} \frac{1}{\kappa_\nu^{\text{ff}}} \dd{\nu}}
  {  \displaystyle \int _{0}   ^{\infty} \dv{I_\nu}{T} \dd{\nu}}
\end{equation}

\paragraph{How the conservation equations change using the moment equations}

The use of the grey moment equations \eqref{eq:NTZ91-moment-equations-logarithmic} is in simplifying the equations of motion given by \(\nabla_\mu T^{\mu\nu}\).

The full energy momentum tensor of the problem is given, in the fiducial reference frame, by combining an ideal-fluid stress-energy tensor (with pressure \(P\) and rest energy density \(\rho\)) with the one given in \eqref{eq:radiation-stress-energy-tensor-fiducial}:

\begin{equation}
    T^{\mu\nu} =
    T^{\mu\nu}_{\text{radiation}} +
    T^{\mu\nu}_{\text{matter}} =
    \left[\begin{matrix}\rho + w_{0} & w_{1} & 0 & 0\\w_{1} & P + \frac{w_{0}}{3} + w_{2} & 0 & 0\\0 & 0 & P + \frac{w_{0}}{3} - \frac{w_{2}}{2} & 0\\0 & 0 & 0 & P + \frac{w_{0}}{3} - \frac{w_{2}}{2}\end{matrix}\right] _{\text{fid}}
\end{equation}

\begin{greenbox}
  We want to write down the conservation equations. Do we do so in regular spherical coordinates? The transformed stress-energy tensor is really ugly:
\end{greenbox}

The \(t,r\) by \(t, r\) components of the stress-energy tensor in spherical coordinates look like:

\begin{equation} \label{eq:spherical-coordinates-full-stress-energy-tensor}
      \left[
      \begin{matrix}
      \frac{\gamma^{4} \left(\rho - v w_{1} + \frac{v \left(v \left(3 P + w_{0} + 3 w_{2}\right) - 3 w_{1}\right)}{3} + w_{0}\right)}{y^{2}} &
      \gamma^{2} \left(- \frac{v \left(3 P + w_{0} + 3 w_{2}\right)}{3} + v \left(- \rho + v w_{1} - w_{0}\right) + w_{1}\right)\\
      \gamma^{2} \left(- v \left(\rho + w_{0}\right) + \frac{v \left(- 3 P + 3 v w_{1} - w_{0} - 3 w_{2}\right)}{3} + w_{1}\right) &
      y^{2} \left(P - v w_{1} + v \left(v \left(\rho + w_{0}\right) - w_{1}\right) + \frac{w_{0}}{3} + w_{2}\right)
    \end{matrix}
      \right]
\end{equation}

\paragraph{Bernoulli equation simplification}

The ``Bernoulli equation'' can be written as \(\nabla_\mu T^{t\mu} = 0\) in the spherical reference trame --- that is, by projecting the conservation of the stress-energy tensor onto the time-like unit vector in the spherical coordinate system.

% \begin{greenbox}
%   Wait, this is the conservation of the stress-energy tensor projected onto the unit time-directed vector in the \emph{spherical} reference, not the \emph{fiducial} one! If this were not the case, we would have two terms (\(\hat{r}\) and \(\hat{t}\)) in the divergence sum.
%
%   Usually one would expect the Bernoulli equation to be projected along the fluid's 4-velocity.
% \end{greenbox}

Then, applying \eqref{eq:covariant-divergence} and our symmetry assumptions:

\begin{equation}
  0=\nabla_\mu T^{t \mu}
  = \frac{1}{r^2 \sin \theta} \pdv{}{r} \qty(r^2 \sin\theta T^{tr})
  = \frac{1}{r^2} \pdv{}{r} \qty(r^2 T^{tr})
  \implies
  r^2 T^{tr} = \const
\end{equation}

In \cite[before eq. 18c]{ThorneFLammmangZytkow:1981feb} this appears with an additional immaterial factor of \(4 \pi\). The quantity which is conserved is the $(0,1)$ matrix element which appears in \eqref{eq:spherical-coordinates-full-stress-energy-tensor}. We can simplify it by expressing it in terms of the luminosity \(L\) which is \(L = 4 \pi r^2 w_1\): we get

\begin{equation}
  4 \pi r^2 \qty(\gamma^2 (1+ v^2) \frac{L}{4 \pi r^2} - v \gamma^2 \qty(p + \rho + w_2 + \frac{4}{3} w_0)) = \const
\end{equation}

\begin{greenbox}
    I must have done something wrong: the constant multiplying everything is \(y^2\) instead of \(\gamma^2\) in \cite[]{ThorneFLammmangZytkow:1981feb} and \cite[]{NobiliTurollaZampieri:1991dec}.
    I will change it now and figure it out later.
\end{greenbox}

We can also substitute in the expression for \(v = \dot{M} / \qty(4 \pi r^2 \rho_0 y) \), and recognize the expression for the specific enthalpy \(h\).

\begin{equation}
    y^2 (1+v^2) L - y \dot{M} \qty(h + \frac{4w_0/3 + w_2}{\rho_0}) = \const 
\end{equation}


\begin{claim}

These conservation laws can be projected onto \(\hat{t}\) and \(\hat{r}\) and cast into \cite[eq. A7]{NobiliTurollaZampieri:1991dec}:

\begin{subequations} \label{eq:fiducial-projected-conservation}
\begin{align}
    (P+\rho) \dv{y}{r}
    +y \dv{P}{r}
    +\frac{4}{3} w_{0} \dv{y}{r}
    +\frac{1}{3} y \dv{w_0}{r}
    +\frac{1}{y v r^{2}} \dv{}{r}\left(y^{2} v^{2} r^{2} w_{1}\right)
    +\frac{1}{r^{3}} \dv{}{r}\left(r^{3} y w_{2}\right) &=0 \\
    \dv{\rho}{r}
    -\frac{P+\rho}{\rho_{0}}
    \dv{\rho_{0}}{r}
    +\dv{w_0}{r}
    +\frac{4}{3} \frac{w_{0}}{y v r^{2}} \dv{}{r}\left(y v r^{2}\right)
    +\frac{1}{y^{2} v r^{2}} \dv{}{r}\left(y^{2} r^{2} w_{1}\right)
    +w_{2} \frac{r}{y v} \dv{}{r}\left(\frac{y v}{r}\right) &=0
\end{align}
\end{subequations}
\end{claim}

By plugging the moment equations \eqref{eq:NTZ91-moment-equations-logarithmic} into the equations \eqref{eq:fiducial-projected-conservation} we get the simplified conservation equations we will work with, to which we add the continuity equation \eqref{eq:differential-continuity}. Here, as in \cite[]{NobiliTurollaZampieri:1991dec}, primes denote differentiation with respect to \(\log r\). The complete system we need to solve is:

\begin{subequations}
\begin{align}
  (P + \rho) \frac{y'}{y} + P^\prime + \frac{2rM s_1}{y}  &= 0  \\
  \rho^\prime - (P + \rho) \frac{\rho_0^\prime}{\rho_0} + \frac{2rM s_0}{vy}   &=0  \\
  \frac{(vy)^\prime}{vy} + \frac{\rho_0 ^\prime}{\rho_0} + 2 &=0
\end{align}
\end{subequations}

\end{document}
